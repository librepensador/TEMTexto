\documentclass[oneside]{book}%
\usepackage{amsmath}%
%\usepackage{amsfonts}%
%\usepackage{amssymb}%
\usepackage[latin1]{inputenc}
\usepackage{pst-all}
\usepackage{graphicx}
\usepackage{color}
\usepackage{graphics}
\usepackage[spanish]{babel}
\usepackage[adobe-utopia]{mathdesign}
\usepackage{array}
\usepackage{bm}
%\usepackage{cool}
\newcommand{\R}{\mathbb{R}}
\hoffset -0.4in
\setlength{\marginparwidth}{0pt}
\setlength{\textwidth}{14cm}

\title{Notas de Clase del curso de Teor�a Electromagn�tica}
\author{Carlos Pe�uela }
\date{Julio 2009}

\begin{document}
�
\setlength{\parindent}{0pt}
\setlength{\parskip}{3mm}

\maketitle

\tableofcontents

\chapter{An�lisis Vectorial}

En ciencia e ingenier�a usualmente se encuentran cantidades como la temperatura, la masa, el tiempo, etc. que se determinan con s�lo una magnitud. Esas cantidades se denominan escalares. Tambi�n existen otras variables de inter�s, llamadas vectores, que tienen magnitud y direcci�n. Algunos ejemplos son: velocidad, aceleraci�n, fuerza, momentum.

Para distinguir los vectores de los escalares, se sigue la siguiente convenci�n: los vectores se denotan con letras en $\mathbf{negrilla}$ y los escalares sin negrilla. As� $\mathbf{v}$ representa el vector velocidad y $v$ (la magnitud del vector velocidad) la rapidez. 

La representaci�n de un vector $\mathbf{p}$ en $R^3$ como una flecha naciendo en el origen del sistema de coordenadas y terminando en el punto de coordenadas $(p_x,p_y,p_z)$, permite establecer una relaci�n biun�voca entre vectores y tripletas de n�meros reales.
\begin{figure}[ht]
	\centering
		\includegraphics{1g1.1}
	\caption{Representaci�n de un vector en $R^3$.}
	\label{fig:1g1}
\end{figure}

Un vector que ser� particularmente importante es el vector posici�n, denotado por $\mathbf{r}$ y asociado al punto $(x,y,z)$ de $R^3$. 
\begin{equation}
\mathbf{r}\leftrightarrow(x,y,z)\;.
\end{equation}

Al usar $r$ como la magnitud de $\mathbf{r}$, las coordenadas quedan establecidas as�:
\begin{equation}
P_x=r\cos\alpha\qquad,\qquad P_y=r\cos\beta\qquad,\qquad P_z=r\cos\gamma\;.
\end{equation}

Donde $\alpha$, $\beta$, y $\gamma$ son los cosenos directores de $\mathbf{r}$. �ngulos formados entre el vector $\mathbf{r}$ y los semiejes cartesianos positivos $x,y,z$; respectivamente. Las cantidades $p_x$, $p_y$, y $p_z$ son las \textit{componentes} (cartesianas) de $\mathbf{p}$ o las \textit{proyecciones} de $\mathbf{p}$; como se observa en la figura \ref{fig:1g1}. 

\section{Operaciones con vectores}

\subsection{Vectores unitarios}

Por definici�n un vector unitario es un vector que tiene magnitud igual a la unidad. El vector unitario que tiene la misma direcci�n que el vector $\mathbf{p}$ es: 
\begin{equation}
\hat{\mathbf{p}}=\frac{\mathbf{p}}{p}\;.
\end{equation}
Los vectores unitarios de los ejes del sistema de coordenadas cartesiano son denotados as�: 
\begin{equation*}
\begin{aligned}
		&\hat{\mathbf{x}}: \text{Vector unitario para el eje } x=(1,0,0)\;,\\
		&\hat{\mathbf{y}}: \text{Vector unitario para el eje } y=(0,1,0)\;,\\
		&\hat{\mathbf{z}}: \text{Vector unitario para el eje } z=(0,0,1)\;.
\end{aligned}
\end{equation*}

Se puede escribir el vector $\mathbf{p}$ como: 
\begin{equation}
\begin{gathered}
	\hat{\mathbf{p}}=p_x\hat{\mathbf{x}}+p_y\hat{\mathbf{y}}+p_z\hat{\mathbf{z}}\;,\\
	\text{donde}\quad p_x=\hat{\mathbf{p}}\cdot\hat{\mathbf{x}}\quad,\quad p_y=\hat{\mathbf{p}}\cdot\hat{\mathbf{y}}\quad,\quad
			pz=\hat{\mathbf{p}}\cdot\hat{\mathbf{z}}\;,
\end{gathered}
\end{equation}

y por el teorema de Pit�goras: 
\begin{equation}
p=(p_x^2+p_y^2+p_z^2)^{1/2}\;.
\end{equation}

Al denotar un vector con sus componentes respecto de un sistema de coordenadas se presentan infinidad de posibilidades, las m�s simples ser�n tratadas en el cap�tulo \ref{cap_SistCoord}.

Todo vector en $R^3$ se puede escribir como una combinaci�n lineal de los vectores unitarios $\hat{\mathbf{x}}$, $\hat{\mathbf{y}}$ y $\hat{\mathbf{z}}$; que son linealmente independientes. Esto implica que la triada $\hat{\mathbf{x}}$, $\hat{\mathbf{y}}$ y $\hat{\mathbf{z}}$ forma una base en $R^3$. 

\subsection{Operaciones b�sicas}

Definimos la \textit{suma} de los vectores $\mathbf{A}=(A_1,A_2,\ldots,A_n)$ y $\mathbf{B}=(B_1,B_2,\ldots,B_n)$ como: 
\begin{equation}
\mathbf{A}+\mathbf{B}=(A_1+B_1,A_2+B_2,\ldots,A_n+B_n)\;.
\end{equation}

Para cualquier n�mero real $r$ y el vector $\mathbf{A}=(A_1,A_2,\ldots,A_n)$ se define el M�ltiplo escalar como: 
\begin{equation}
r\mathbf{A}=(rA_1,rA_2,\ldots,rA_n)\;.
\end{equation}

Las f�rmulas dadas a continuaci�n son v�lidas para cualquiera $\mathbf{A}$, $\mathbf{B}$, $\mathbf{C}$ en $R^n$ y los n�meros $r$ y $s$ en $R$. 
\begin{equation}\label{1ec8}
\begin{split}
	\mathbf{A}+\mathbf{B}&=\mathbf{B}+\mathbf{A}\;,\\
	(\mathbf{A}+\mathbf{B})+\mathbf{C}&=\mathbf{A}+(\mathbf{B}+\mathbf{C})\;,\\
	\mathbf{A}+0&=\mathbf{A}\;,\\
	\mathbf{A}+(-\mathbf{A})&=0\;,\\
	r(\mathbf{A}+\mathbf{B})&=r\mathbf{A}+r\mathbf{B}\;,\\
	(r+s)\mathbf{A}&=r\mathbf{A}+s\mathbf{A}\;,\\
	r(s\mathbf{A})&=(sr)\mathbf{A}\;,\\
	1\mathbf{A}&=\mathbf{A}\;.
\end{split}
\end{equation}

Estas leyes expresan la conmutatividad, distributividad y asosiatividad de las operaciones suma y m�ltiplo escalar para los vectores. 

\subsubsection{Producto Escalar}

El producto escalar o producto punto es una forma de operar los vectores que tiene mucho inter�s para la F�sica. Se define como: 
\begin{equation}
	\mathbf{A}\cdot\mathbf{B}=A_1B_1+A_2B_2+\cdots+A_nB_n=\sum\limits_i^n{A_i\,B_i}\;.
\end{equation}

Este producto satisface las siguientes propiedades:
\begin{equation}
\begin{aligned}
		&\text{No negatividad:} & &\mathbf{A}\cdot\mathbf{A}\geq0\;.\\
		&\text{Conmutatividad o simetr�a:} & &\mathbf{A}\cdot\mathbf{B}=\mathbf{B}\cdot\mathbf{A}\;.\\
		&\text{Distributividad:} & &(\mathbf{A}+\mathbf{B})\cdot\mathbf{C}=\mathbf{A}\cdot\mathbf{C}+\mathbf{B}\cdot\mathbf{C}\;.\\
		&\text{Homogeneidad:} & &(r\mathbf{A})\cdot\mathbf{B}=r(\mathbf{A}\cdot\mathbf{B})\;.
\end{aligned}
\end{equation}

En particular para $\R^3$, la magnitud de un vector se define con el producto punto de el sobre si mismo, as�: 
\begin{equation}
	|\mathbf{A}| = A = \sqrt{\mathbf{A}\cdot\mathbf{A}}=(A_x^2+A_y^2+A_z^2)^{1/2}\;.
\end{equation}

Las propiedades de la longitud, norma o magnitud de un vector son:\par\smallskip
No negatividad: $|\mathbf{A}|\geq0$.\par
Homogeneidad: $|r\mathbf{A}|=|r||\mathbf{A}|$.\par
Desigualdad triangular: $|\mathbf{A}+\mathbf{B}|\leq|\mathbf{A}|+|\mathbf{B}|$.\par
Desigualdad de Cauchy-Schwarz: $|\mathbf{A}\cdot\mathbf{B}|\leq|\mathbf{A}||\mathbf{B}|$.\par\smallskip

El �ngulo que forma un vector con otro se define usando el producto punto as�: \\
Si $\theta$ es el �ngulo entre $\mathbf{A}$ y $\mathbf{B}$, entonces: 
\begin{equation}
	\cos\theta=\frac{\mathbf{A}\cdot\mathbf{B}}{|\mathbf{A}||\mathbf{B}|}\;.
\end{equation}

\subsubsection{Producto vectorial}


El producto vectorial o producto cruz en $\R^3$ se define como:

\begin{equation}
\begin{aligned}
&\mathbf{A}\times\mathbf{B}=(A_x\hat{\mathbf{x}}+A_y\hat{\mathbf{y}}+A_z\hat{\mathbf{z}})\times(B_x\hat{\mathbf{x}}+B_y\hat{\mathbf{y}}+B_z\hat{\mathbf{z}})=\mathbf{C}\\
&=(A_yB_z-A_zB_y)\,\hat{\mathbf{x}}+(A_zB_x-A_xB_z)\,\hat{\mathbf{y}}+(A_xB_y-A_yB_x)\,\hat{\mathbf{z}}\;,
\end{aligned}
\end{equation}

� 
\begin{equation}
	C_i=A_jB_k+A_kB_j\;,
\end{equation}
con $i$, $j$, $k$ diferentes y en orden \footnote{El orden es por rotaci�n:\\\centerline{\includegraphics{1g2.2}}}.

Una forma de recordar este resultado es utilizando el siguiente pseudodeterminante: 
\begin{equation}
\mathbf{A}\times\mathbf{B}=\mathbf{C}=\begin{vmatrix}
\hat{\mathbf{x}} & \hat{\mathbf{y}} & \hat{\mathbf{z}} \\
A_x & A_y & A_z \\
B_x & B_y & B_z \\
\end{vmatrix}\;.
\end{equation}

Al desarrollar $\mathbf{A}\cdot\mathbf{C}$ y $\mathbf{B}\cdot\mathbf{C}$, se puede demostrar que el vector $\mathbf{C}$ es perpendicular tanto a $\mathbf{A}$ como a $\mathbf{B}$, o dicho de otra forma, $\mathbf{C}$ es perpendicular al plano que forman $\mathbf{A}$ y $\mathbf{B}$. 

De una forma similar, al desarrollar el producto $(\mathbf{A}\times\mathbf{B})\cdot(\mathbf{A}\times\mathbf{B})$ se puede demostrar que: 
\begin{equation}
	|\mathbf{A}\times\mathbf{B}|=|\mathbf{C}|=AB\sen{\theta}\;.
\end{equation}

Donde $\theta$ es el �ngulo que forman $\mathbf{A}$ y $\mathbf{B}$. 

Para este producto se satisfacen las siguientes igualdades: 
\begin{equation}
	\begin{aligned}
		&\text{Distributividad derecha: } & 
				 &\mathbf{A}\times(\mathbf{B}+\mathbf{C})=\mathbf{A}\times\mathbf{B}+\mathbf{A}\times\mathbf{C}\;.\\
		&\text{Distributividad izquierda: } & 
				 &(\mathbf{A}+\mathbf{B})\times\mathbf{C}=\mathbf{A}\times\mathbf{C}+\mathbf{B}\times\mathbf{C}\;.\\
		&\text{Homogeneidad: } & 
				 &\mathbf{A} \times (r\mathbf{B})=r(\mathbf{A}\times\mathbf{B})=(r\mathbf{A})\times\mathbf{B}\;.\\	
		&\text{Anticonmutatividad o antisimetr�a: } & 
				 &\mathbf{A}\times\mathbf{B}=-\mathbf{B}\times\mathbf{A}\;.
	\end{aligned}
\end{equation}

El producto cruz tiene una importante interpretaci�n geom�trica. El paralelogramo formado por $\mathbf{A}$ y $\mathbf{B}$ en la figura \ref{1g3}, $B\sen\theta$ es la altura si $\mathbf{A}$ se considera como una de las bases. 
\begin{figure}[ht]
	\centering
		\includegraphics{1g3.3}
	\caption{Interpretaci�n geom�trica del producto cruz.}
	\label{1g3}
\end{figure}

Luego $|\mathbf{A}\times\mathbf{B}|=AB\sen\theta$, es el �rea del paralelogramo. El vector $\mathbf{A}\times\mathbf{B}$ sugiere tratar el �rea de paralelogramo formado por $\mathbf{A}$ y $\mathbf{B}$ como un vector.

Ahora con la ayuda del triple producto escalar, se puede definir c�mo es la orientaci�n de un conjunto de vectores en $R^3$. Definimos el triple producto escalar como: 
\begin{equation}
	\mathbf{A}\cdot(\mathbf{B}\times\mathbf{C})\;.
\end{equation}

Al recordar que $\mathbf{B}\times\mathbf{C}$ representa el �rea formada por $\mathbf{B}$ y $\mathbf{C}$, se concluye entonces que $\mathbf{A}\cdot(\mathbf{B}\times\mathbf{C})$ representa el volumen del paralelep�pedo definido por $\mathbf{A}$, $\mathbf{B}$ y $\mathbf{C}$. Ver figura \ref{1g4}. 
\begin{figure}[ht]
	\centering
  	\includegraphics{1g4.4}
	\caption{Representaci�n geom�trica del triple producto escalar.}
	\label{1g4}
\end{figure}
Se dice que el conjunto ordenado $\left\{\mathbf{A}, \mathbf{B}, \mathbf{C}\right\}$ de vectores en $R^3$ tiene \textbf{orientaci�n positiva} si $\mathbf{A}\cdot(\mathbf{B}\times\mathbf{C})>0$ o tiene \textbf{orientaci�n negativa} si $\mathbf{A}\cdot(\mathbf{B}\times\mathbf{C})<0$. Se puede demostrar r�pidamente que el sistema
\[
\left\{\hat{\mathbf{x}},\hat{\mathbf{y}},\hat{\mathbf{z}}\right\}=\left\{\left(1,0,0\right);(0,1,0);(0,0,1)\right\}
\]
es un conjunto de vectores orientados positivamente. Los sistemas as� se denominan dextr�giros o de mano derecha, mientras que los de orientaci�n izquierda se llaman sistemas lev�giros o de mano izquierda. 

\section{Campos Vectoriales}

Cuando se tienen funciones vectoriales donde el dominio (conjunto de vectores donde la funci�n est� definida) y la imagen o rango (conjunto de valores que toma la funci�n) son los mismos, a menudo es conveniente representar los vectores del dominio como puntos y los vectores de la imagen como flechas con sus colas en los puntos. De esta forma, por ejemplo, se puede representar f�cilmente la velocidad en cada punto del flujo, como se muestra en la figura \ref{1g5}. 

Considerada de esta manera, la funci�n vectorial se llama \textbf{campo vectorial} y se representa con vectores que nacen de cada punto del espacio dominio. En consecuencia, dominio e imagen deben tener la misma dimensi�n. Si a cada punto del espacio dominio se le asocia no un vector sino un escalar, a la funci�n se le denomina \textbf{campo escalar}.
\begin{figure}[ht]
	\centering
		\includegraphics{1g5.5}
	\caption{Representaci�n de la velocidad de las part�culas de un fluido en una tuber�a.}
	\label{1g5}
\end{figure}

\section{Sistema de coordenadas}\label{cap_SistCoord}

A menudo la representaci�n matem�tica de la realidad f�sica es m�s entendible o f�cil de manipular si se cambia la forma de ubicar los puntos en el espacio. Esto es, al migrar de un sistema de coordenadas a otro mediante una transformaci�n de coordenadas, se puede simplificar el tratamiento matem�tico y as� concentrar la atenci�n en el fen�meno y no en su descripci�n. 

\subsection{Coordenadas curvil�neas generalizadas}
En el sistema de coordenadas rectangulares, el vector $(x_0,y_0,z_0)$ representa un punto que est� en la intersecci�n de los planos $x=x_0$, $y=y_0$ y $z=z_0$. De igual forma, un punto descrito desde un sistema de coordenadas curvil�neo se considera como la intersecci�n de una triada de superficies no necesariamente planas. De ahora en adelante solo se considerar�n los sistemas ortogonales, esto significa que las superficies deben ser mutuamente perpendiculares. Consid�rense que las formas impl�citas de la triada de superficies son: 
\begin{equation}\label{eq:form_impl_coor_curv}
	u_1(x,y,z)=u_1\qquad,\qquad u_2(x,y,z)=u_2\qquad,\qquad u_3(x,y,z)=u_3\;.
\end{equation}

El sistema de ecuaciones en \ref{eq:form_impl_coor_curv} realiza la transformaci�n de coordenadas curvil�neas a coordenadas rectangulares y se conoce como la transformaci�n inversa. La transformaci�n directa es: 
\begin{equation}
	x=x(u_1,u_2,u_3)\qquad,\qquad y=y(u_1,u_2,u_3)\qquad,\qquad z=z(u_1,u_2,u_3)\;.
\end{equation}

En esta �ltima relaci�n se tiene que: 
\begin{equation}\label{1ec22}
\begin{aligned}
	dx&=\frac{\partial x}{\partial u_1}du_1+\frac{\partial x}{\partial u_2}du_2+\frac{\partial x}{\partial u_3}du_3\;,\\
	dy&=\frac{\partial y}{\partial u_1}du_1+\frac{\partial y}{\partial u_2}du_2+\frac{\partial y}{\partial u_3}du_3\;,\\
	dz&=\frac{\partial z}{\partial u_1}du_1+\frac{\partial z}{\partial u_2}du_2+\frac{\partial z}{\partial u_3}du_3\;.\\
\end{aligned}
\end{equation}

Adem�s el elemento diferencial de l�nea en coordenadas rectangulares es: 
\begin{equation}\label{1ec23}
	dr^2=d\mathbf{r}\cdot d\mathbf{r}=dx^2+dy^2+dz^2\;.
\end{equation}

Luego de reemplazar \eqref{1ec22} en \eqref{1ec23} se puede escribir todo en forma condensada as�: 
\begin{equation}
	dr^2=\sum_{i,j}{\frac{\partial\mathbf{r}}{\partial u_i}\cdot}\frac{\partial\mathbf{r}}{\partial u_j}\,du_i\,du_j\;.
\end{equation}

Donde $\mathbf{r}$ es un vector que se�ala cualquier punto en el espacio, $\mathbf{r}=(x,y,z)$. 

En los sistemas ortogonales, los vectores $\frac{\partial\mathbf{r}}{\partial u_i}$ y $\frac{\partial\mathbf{r}}{\partial u_j}$, que son tangentes a los ejes $u_i$ y $u_j$, son perpendiculares si $i\neq j$ y por supuesto su producto punto es nulo ($\frac{\partial\mathbf{r}}{\partial u_i}\cdot\frac{\partial\mathbf{r}}{du_j}=0$). Considerando que $i\neq j$, se definen los factores de escala: 
\begin{equation} 
	h_i=\left| \frac{\partial\mathbf{r}}{\partial u_i} \right|\;,
\end{equation}

y el elemento diferencial $dr$ se puede expresar como: 
\begin{equation}\label{1ec26}
	dr^2=\sum_i{h_i^2\,du_i^2=h_1^2\,du_1^2+h_2^2\,du_2^2+h_3^2\,du_3^2}\;.
\end{equation}

Si se considera s�lo cambios en $r_i$ al mantener las otras variables constantes, se establece la siguiente relaci�n de \eqref{1ec26}. 
\begin{equation}
	dr_i=h_i\,du_i\;.
\end{equation}

Esto implica que el vector diferencial de posici�n es: 
\begin{equation}
d\mathbf{r}=h_1\,du_1\,\hat{\mathbf{u}}_1+h_2\,du_2\,\hat{\mathbf{u}}_2+h_3\,du_3\,\hat{\mathbf{u}}_3
	=\sum_i{h_i\,du_i\,\hat{\mathbf{u}}_i}\;.
\end{equation}

Donde $\hat{\mathbf{u}}_i$ es el vector unitario tangente a la curva formada por el corte de las superficies $u_j=u_{j0}$ y $u_k=u_{k0}$, siendo $u_{j0}$ y $u_{k0}$  los par�metros que definen las superficies. De igual forma, al considerar s�lo el cambio en la variable $u_{i}$, dejando fijas $u_j$ y $u_k$ se obtiene: 
\begin{equation}\label{1ec29}
	\frac{\partial{\mathbf{r}}}{\partial{u_i}}=h_i\,\hat{\mathbf{u}}_i\;.
\end{equation}

Adem�s, ya que el sistema es ortogonal y se elige dextr�giro: 
\begin{equation}
	\begin{aligned}
\hat{\mathbf{u}}_1\cdot\hat{\mathbf{u}}_2&=0 & &, 
&\hat{\mathbf{u}}_2\cdot\hat{\mathbf{u}}_3&=0 & &, 
&\hat{\mathbf{u}}_3\cdot\hat{\mathbf{u}}_1&=0 & &,\\
\hat{\mathbf{u}}_1\times\hat{\mathbf{u}}_2&=\hat{\mathbf{u}}_3 & &, &\hat{\mathbf{u}}_2\times\hat{\mathbf{u}}_3&=\hat{\mathbf{u}}_1 & &, &\hat{\mathbf{u}}_3\times\hat{\mathbf{u}}_1&=\hat{\mathbf{u}}_2 & &.
\end{aligned}
\end{equation}

Tambi�n: 
\begin{equation}
	dr=(h_1^2\,du_1^2+h_2^2\,du_2^2+h_3^2\,du_3^2)^{1/2}\;.
\end{equation}

El elemento diferencial de superficie se define:
\begin{equation}
	d\mathbf{S}_{ij}=\frac{\partial\mathbf{r}}{\partial u_i}\times\frac{\partial\mathbf{r}}{\partial u_j}\,du_i\,du_j\;.
\end{equation}

Y aprovechando \eqref{1ec29}, se tiene: 
\begin{equation}
	d\mathbf{S}_{ij}=h_ih_j\hat{\mathbf{u}}_i\times\hat{\mathbf{u}}_j\,du_i\,du_j=h_ih_j\,du_i\,du_j\hat{\mathbf{u}}_k\;.
\end{equation}
 
El diferencial de superficie general es: 
\begin{equation}
	d\mathbf{S}=h_2h_3\,du_2\,du_3\,\hat{\mathbf{u}}_1+h_1h_3\,du_1\,du_3\,\hat{\mathbf{u}}_2+h_1h_2\,du_1\,du_2\,\hat{\mathbf{u}}_3\;.
\end{equation}

El elemento de volumen se consigue a partir de: 
\begin{equation}
	dV=\frac{\partial\mathbf{r}}{\partial u_1}\cdot\frac{\partial\mathbf{r}}{\partial u_2}\times\frac{\partial\mathbf{r}}{\partial\;. u_3}\,du_i\,du_j\,du_k\;.
\end{equation}

Al utilizar \eqref{1ec29}, se tiene: 
\begin{equation}
\begin{aligned}
dV&=h_1\,\hat{\mathbf{u}}_1\cdot h_2\,\hat{\mathbf{u}}_2\times h_3\,\hat{\mathbf{u}}_3\,du_1\,du_2\,du_3\;,\\
dV&=h_1\,h_2\,h_3\,du_1\,du_2\,du_3\;.
\end{aligned}
\end{equation}

Donde se ha utilizado el hecho que $\hat{\mathbf{u}}_1\cdot \hat{\mathbf{u}}_2\times\hat{\mathbf{u}}_3=1$, por ser un sistema dextr�giro. 
\begin{figure}[ht]
	\centering
		\includegraphics{1g6.6}
	\caption{Sistema de coordenadas generalizadas.}
	\label{1g6}
\end{figure}


\subsection{Coordenadas cartesianas rectangulares}


Como se ha dicho ya, este sistema describe un punto por la tripleta $(x_0,y_0,z_0)$, queriendo significar que el punto est� en la intersecci�n de los planos $x=x_0$, $y=y_0$ y $z=z_0$.
\begin{figure}[ht]
	\centering
		\includegraphics{1g7.7}
	\caption{Sistema de coordenadas cartesianas rectangulares.}
	\label{1g7}
\end{figure}
Al utilizar \eqref{1ec8} y \eqref{1ec29} se deduce que: 
\begin{equation}
	h_1=h_x=1\qquad,\qquad h_2=h_y=1\qquad,\qquad h_3=h_z=1\;.
\end{equation}

Los vectores unitarios en rectangulares llevan la direcci�n y el sentido en el que se desplaza el punto al incrementar la coordenada correspondiente, se ordenan de manera que el sistema sea dextr�giro y siempre tengan direcci�n fija. 
\begin{equation}
\begin{aligned}
	\hat{\mathbf{u}}_1=\hat{\mathbf{x}}=(1,0,0)\;,\\
	\hat{\mathbf{u}}_2=\hat{\mathbf{y}}=(0,1,0)\;,\\
	\hat{\mathbf{u}}_3=\hat{\mathbf{z}}=(0,0,1)\;.\\
\end{aligned}
\end{equation}
\begin{equation}
	\hat{\mathbf{x}}\times\hat{\mathbf{y}}=\hat{\mathbf{z}}\qquad,\qquad \hat{\mathbf{y}}\times\hat{\mathbf{z}}=\hat{\mathbf{x}}\qquad,\qquad\hat{\mathbf{z}}\times\hat{\mathbf{x}}=\hat{\mathbf{y}}\;.
\end{equation}

Los valores de $x$, $y$ y $z$ var�an desde $-\infty$ hasta $\infty$. El diferencial de longitud es:
\begin{equation}
\begin{split}
	d\mathbf{r}=dx\,\hat{\mathbf{x}}+dy\,\hat{\mathbf{y}}+dz\,\hat{\mathbf{z}}\;,\\
	dr=(dx^2+dy^2+dz^2)^{1/2}\;.
\end{split}
\end{equation}

Los diferenciales de superficie con $x=x_0$, $y=y_0$ y $z=z_0$ se ilustran en las figuras.  

7\begin{table}
\centering
\begin{tabular}{|c|c|}\hline
  \includegraphics{1g8.8} &  \includegraphics{1g9.9} \\ \hline
  a) & b) \\ \hline
  \multicolumn{2}{|c|}{\includegraphics{1g10.10}}  \\ \hline
  \multicolumn{2}{|c|}{c)}\\ \hline
\end{tabular}
\caption{Diferenciales de superficie en coordenadas rectangulares con: a) $x=x_0$ , b) $y=y_0$ y c) $z=z_0$.}
\label{tab_GraDifSupRec}
\end{table}

El diferencial de volumen es:
\begin{equation}
	dV=dx\,dy\,dz\;.
\end{equation}

\subsection{Coordenadas cil�ndricas circulares}

Las superficies coordenadas del sistema son: cilindros circunferenciales rectos de radios $\rho$ cuyos ejes coinciden con el eje $z$; semiplanos que contienen al eje $z$ y forman un �ngulo $\phi$ con el semiplano de referencia $xz$; y planos paralelos al plano $xy$. 
\begin{figure}[ht]
	\centering
		\includegraphics{1g11.11}
	\caption{Coordenadas cil�ndricas circulares.}
	\label{1g11}
\end{figure}

El punto $p$ se describe mediante la tripleta $(\rho,\phi,z)$. 

Para pasar de coordenadas cil�ndricas a rectangulares se utilizan las siguientes f�rmulas. 
\begin{equation}
\begin{aligned}
	&\rho=(x^2+y^2)^{1/2}\;. \\
	&\phi=\cos^{-1}\left(\frac{x}{\sqrt{x^2+y^2}}\right)\;. \\
	&z=z\;.
\end{aligned}
\end{equation}

La transformaci�n directa es: 
\begin{equation}
\begin{aligned}
	&x=\rho\cos\phi\;, \\
	&y=\rho\sen\phi\;, \\
	&z=z\;.
\end{aligned}
\end{equation}

Para describir un�vocamente a un punto las coordenadas deben variar as�: 
\begin{equation}
	0\leq\rho<\infty\qquad,\qquad 0\leq\phi<2\pi\qquad,\qquad -\infty< z <\infty\;.
\end{equation}

Existe una ambig�edad con los puntos sobre el eje $z$, pues quedan definidos por su $z$, por $\rho=0$ y cualquier valor de $\phi$. Los factores de escala son: 
\begin{equation}
\begin{aligned}
	& h_1=h_\rho=1\;, \\
	& h_2=h_\phi=\rho\;, \\
	& h_3=h_z=1\;. \\
\end{aligned}
\end{equation}

Los vectores unitarios son:
\begin{equation}
\text{Tomando}\qquad \mathbf{r}=x\hat{\mathbf{x}}+y\hat{\mathbf{y}}+z\hat{\mathbf{z}}=
		\rho\cos\phi\,\hat{\mathbf{x}}+\rho\sen\phi\,\hat{\mathbf{y}}+z\,\hat{\mathbf{z}}\;,\\
\end{equation}
\begin{equation}
\begin{aligned}
\hat{\mathbf{\rho}}&=\frac{1}{h_\rho}\frac{\partial\mathbf{r}}{\partial\rho}=\cos\phi\,\hat{\mathbf{x}}+\sen\phi\,\hat{\mathbf{y}}\;,\\
\hat{\mathbf{\varphi}}&=\frac{1}{h_\phi}\frac{\partial\mathbf{r}}{\partial\phi}=-\sen\phi\,\hat{\mathbf{x}}+\cos\phi\,\hat{\mathbf{y}}\;,\\
\hat{\mathbf{z}}&=\frac{1}{h_z}\frac{\partial\mathbf{r}}{\partial z}=\hat{\mathbf{z}}\;.
\end{aligned}
\end{equation}

El vector unitario $\hat{\mathbf{\rho}}$ es ortogonal al cilindro $\rho=\rho_0$ en la direcci�n en que aumenta $\rho$. El vector unitario $\hat{\mathbf{\varphi}}$ es tangente al cilindro $\rho=\rho_0$ y perpendicular al semiplano $\phi=\phi_0$ en la direcci�n en que aumenta $\phi$. Por �ltimo, el vector $\hat{\mathbf{z}}$ es el mismo vector unitario cartesiano. 

Y las ecuaciones para los unitarios: 
\begin{equation}
\begin{aligned}
	&\hat{\mathbf{x}}=\cos\phi\,\hat{\mathbf{\rho}}-\sen\phi\,\hat{\mathbf{\varphi}}\;, \\
	&\hat{\mathbf{y}}=\sen\phi\,\hat{\mathbf{\rho}}+\cos\phi\,\hat{\mathbf{\varphi}}\;, \\
	&\hat{\mathbf{z}}=\hat{\mathbf{z}}\;.
\end{aligned}
\end{equation}

El vector posici�n es: 
\begin{equation}
\begin{aligned}
\mathbf{r}&=\underbrace{\rho\cos\phi}_{x}\underbrace{\left(\cos\phi\,\hat{\mathbf{\rho}}-\sen\phi\,\hat{\mathbf{\varphi}}\right)}_{\hat{\mathbf{x}}}+\underbrace{\rho\sen\phi}_{y}\underbrace{\left(\sen\phi\,\hat{\mathbf{\rho}}+\cos\phi\,\hat{\mathbf{\varphi}}\right)}_{\hat{\mathbf{y}}}+z\,\hat{\mathbf{z}}\;, \\
	\mathbf{r}&=\rho\hat{\mathbf{\rho}}+z\mathbf{z}\;.
\end{aligned}
\end{equation}

N�tese que $\mathbf{r}$ no depende expl�citamente de $\phi$. La dependencia de $\phi$ est� en $\hat{\mathbf{\rho}}=\cos\phi\,\hat{\mathbf{x}}+\sen\phi\,\hat{\mathbf{x}}$. 

El diferencial de longitud es: 
\begin{equation}
	d\mathbf{r}=d\rho\,\hat{\mathbf{\rho}}+d\phi\,\hat{\mathbf{\varphi}}+dz\,\hat{\mathbf{z}}\;,
\end{equation}
\begin{equation}
	dr=(d\rho^2+\rho^2d\phi^2+dz^2)^{1/2}\;.
\end{equation}

Los diferenciales de superficie con $\rho=\rho_0$, $\phi=\phi_0$ y $z=z_0$ se ilustran en las figuras.

\begin{table}
\centering
\begin{tabular}{|c|c|}\hline
  \includegraphics{1g12.12} &  \includegraphics{1g13.13} \\ \hline
  a) & b) \\ \hline
  \multicolumn{2}{|c|}{\includegraphics{1g14.14}}  \\ \hline
  \multicolumn{2}{|c|}{c)}\\ \hline
\end{tabular}
\caption{Diferenciales de superficie en coordenadas cil�ndricas con: a) $\rho=\rho_0$ , b) $\phi=\phi_0$ y c) $z=z_0$.}
\label{tab_GraDifSupCil}
\end{table}

El diferencial de volumen es:
\begin{equation}
	dV=\rho\,d\rho\,d\phi\,dz\;.
\end{equation}

\subsection{Coordenadas esf�ricas}

En este sistema, cualquier punto se localiza mediante la tripleta $(r,\theta,\phi)$ y significa que el punto $(r_0,\theta_0,\phi_0)$ est� en la intersecci�n de la esfera de radio $r_0$ centrada en el origen; el cono circular recto de �ngulo $\theta_0$ con el eje $z$ y v�rtice en el origen; y el semiplano que contiene al eje $z$ y que forma un �ngulo $\phi_0$ con el semiplano $xz$. Para mayor claridad obs�rvese la figura a) en cuadro  \ref{tab_GraDifSupEsf1}. 

\begin{table}
\centering
\begin{tabular}{|c|c|}\hline
  \includegraphics{1g15.15} &  \includegraphics{1g16.16} \\ \hline
  a) & b) \\ \hline
  \end{tabular}
\caption{a) Punto $P$ en coordenadas esf�ricas. b) Punto $P=(r_0,\phi_0,\theta_0)$.}
\label{tab_GraDifSupEsf1}
\end{table}

La transformaci�n inversa de coordenadas se realiza as�: 
\begin{equation}
	r=\left( x^2+y^2+z^2 \right)^{1/2}\;.
\end{equation}
\begin{equation}
	\theta=\cos^{-1}\left( \frac{z}{\left( x^2+y^2+z^2 \right)^{1/2}} \right)\;.
\end{equation}
\begin{equation}
	\phi=\tan^{-1}\left(\frac{y}{x}\right)\;.
\end{equation}

la transformaci�n directa es:
\begin{equation}
	x=r\sen\theta\cos\phi\;.
\end{equation}
\begin{equation}
	y=r\sen\theta\sen\phi\;.
\end{equation}
\begin{equation}
	z=r\cos\theta\;.
\end{equation}

Para hacer una descripci�n un�voca de los puntos, las coordenadas deben tener la siguiente variaci�n: 
\begin{equation}
	0\leq r<\infty \qquad,\qquad 0\leq\theta\leq\pi\qquad,\qquad 0\leq\phi< 2\pi\qquad\;.
\end{equation}

Existen dos ambig�edades; la primera ocurre con los puntos del eje $z$ que quedan definidos por su valor de $r,\theta=0$ � $\pi$, y $\phi$ puede ser cualquier �ngulo; la segunda est� en el origen al ser determinado por $r=0$, independientemente de los valores de $\theta$ y $\phi$.

Los factores de escala son:
\begin{equation}
	h_1=h_r=1 \qquad,\qquad h_2=h_\theta=r \qquad,\qquad h_3=h_\phi=r\sen\theta\;.
\end{equation}

Los vectores unitarios son: 
\begin{gather}
	\mathbf{r}=r\sen\theta\cos\phi\,\hat{\mathbf{x}}+r\sen\theta\sen\phi\,\hat{\mathbf{y}}+r\cos\theta\,\hat{\mathbf{z}}\;.\\
\hat{\mathbf{r}}=\frac{1}{h_r}\frac{\partial\mathbf{r}}{\partial r}=\sen\theta\cos\phi\,\hat{\mathbf{x}}+\sen\theta\sen\phi\,\hat{\mathbf{y}}+\cos\theta\,\hat{\mathbf{z}}\;.\\
\,\hat{\mathbf{\theta}}=\frac{1}{h_\theta}\frac{\partial\mathbf{r}}{\partial\theta}=\cos\theta\cos\phi\,\hat{\mathbf{x}}+\cos\theta\sen\phi\,\hat{\mathbf{y}}-\sen\theta\,\hat{\mathbf{z}}\;.\\
\,\hat{\mathbf{\varphi}}=\frac{1}{h_\phi}\frac{\partial\mathbf{r}}{\partial\phi}=-\sen\phi\,\hat{\mathbf{x}}+\cos\phi\,\hat{\mathbf{y}}\;.
\end{gather}

El vector unitario $\hat{\mathbf{r}}$ es ortogonal a la esfera $r=r_0$ en la direcci�n en la que aumenta $r$. Asimismo, el vector $\hat{\mathbf{\theta}}$ es tangente a la esfera y perpendicular al cono $\theta=\theta_0$ en la direcci�n en la que aumenta $\theta$. Finalmente, $\hat{\mathbf{\varphi}}$ es tangente tanto a la esfera como al cono y es perpendicular al semiplano $\phi=\phi_0$ en la direcci�n en que aumenta $\phi$. 

Las relaciones para los unitarios son: 
\begin{gather}
\,\hat{\mathbf{x}}=\sen\theta\cos\phi\,\hat{\mathbf{r}}+\cos\theta\cos\phi\,\hat{\mathbf{\theta}}-\sen\phi\,\hat{\mathbf{\varphi}}\;.\\
\,\hat{\mathbf{y}}=\sen\theta\sen\phi\,\hat{\mathbf{r}}+\cos\theta\sen\phi\,\hat{\mathbf{\theta}}+\cos\theta\,\hat{\mathbf{\varphi}}\;.\\
\,\hat{\mathbf{z}}=\cos\theta\,\hat{\mathbf{r}}-\sen\theta\,\hat{\mathbf{\theta}}\;.
\end{gather}

El vector de posici�n es:
\begin{equation}
	\mathbf{r}=r\left[ \sen\theta\left( \cos\phi\,\hat{\mathbf{x}}+\sen\phi\,\hat{\mathbf{y}} \right) +\cos\theta\,\hat{\mathbf{z}} \right]=r\,\hat{\mathbf{r}}\;.
\end{equation}

N�tese que este vector no depende expl�citamente de $\theta$ ni de $\phi$, pero evidentemente depende de esos �ngulos en $\hat{\mathbf{r}}=\sen\theta\cos\phi\hat{\mathbf{x}}+\sen\theta\sen\phi\hat{\mathbf{y}}+\cos\theta\hat{\mathbf{z}}$. 

El diferencial de longitud es:
\begin{equation}\label{1ec69}
\begin{aligned}
	d\mathbf{r}&=dr\,\hat{\mathbf{r}}+r\,d\theta\,\hat{\mathbf{\theta}}+r\,\sen\theta\,d\phi\,\hat{\mathbf{\varphi}}\;,\\
	dr&=(dr^2+r^2\,d\theta^2+r^2\sen^2\theta\,d\phi^2)^{1/2}\;.
\end{aligned}
\end{equation}

En \eqref{1ec69} $dr$, $d\theta$ y $d\phi$ dependen impl�citamente de un par�metro desconocido. 

Los diferenciales de superficie con $r=r_0$, $\theta=\theta_0$ y $\phi=\phi_0$ se ilustran en las siguientes figur

\begin{center}
\begin{table}
\begin{tabular}{|c|c|}\hline
  \includegraphics{1g17.17} &  \includegraphics{1g18.18} \\ \hline
  a) & b) \\ \hline
  \multicolumn{2}{|c|}{\includegraphics{1g19.19}}  \\ \hline
  \multicolumn{2}{|c|}{c)}\\ \hline
\end{tabular}
\caption{Diferenciales de superficie en coordenadas esf�ricas con: a) $r=r_0$ , b) $\theta=\theta_0$ y c) $\phi=\phi_0$}
\label{tab_GraDifSupEsf}
\end{table}
\end{center}

El diferencial del volumen es:
\begin{equation}
	dV=r^2\sen\theta\,d\theta\,d\phi\,dr\;.
\end{equation}

\section{Operadores vectoriales y teoremas fundamentales}
Existen dos grupos de operadores vectoriales; los operadores diferenciales y los operadores integrales. Entre los primeros est�n, por ejemplo, el gradiente ($\nabla$), la divergencia ($\nabla\cdot$), el rotacional ($\nabla\times$) y el laplaciano ($\nabla^2$). Entre los segundos est�n la integral de l�nea y la integral de superficie. Los operadores integrales aplicados a unos casos especiales generan los teoremas de Gauss y de Stokes. 

\subsection{El gradiente ($\nabla)$}

Consid�rese la funci�n $\phi(u_1,u_2,u_3)$ en coordenadas curvil�neas ortogonales. Se sabe que: 
\begin{equation}
\begin{aligned}
	& & d\phi &=\frac{\partial \phi}{\partial u_1}\,du_1+\frac{\partial \phi}{\partial u_2}\,du_2+\frac{\partial \phi}{\partial u_3}\,du_3\;,\\
	&y & d\mathbf{r}&=h_1\,du_1\,\hat{\mathbf{u}}_1+h_2\,du_2\,\hat{\mathbf{u}}_2+h_3\,du_3\,\hat{\mathbf{u}}_3\;.
\end{aligned}
\end{equation}

El $d\phi$ se puede escribir de esta forma:
\begin{equation}
	d\phi=\left( \frac{1}{h_1}\frac{\partial\phi}{\partial u_1}\hat{\mathbf{u}}_1+\frac{1}{h_2}\frac{\partial\phi}{\partial u_2}\hat{\mathbf{u}}_2+\frac{1}{h_3}\frac{\partial\phi}{\partial u_3}\hat{\mathbf{u}}_3 \right)\cdot\left( h_1\,du_1\,\hat{\mathbf{u}}_1+h_2\,du_2\,\hat{\mathbf{u}}_2+h_3\,du_3\,\hat{\mathbf{u}}_3 \right)\;.
\end{equation}

Por definici�n se establece que $\nabla\phi$ es el gradiente de la funci�n $\phi$:
\begin{equation}
	\nabla\phi=\frac{1}{h_1}\frac{\partial\phi}{\partial u_1}\hat{\mathbf{u}}_1+\frac{1}{h_2}\frac{\partial\phi}{\partial u_2}\hat{\mathbf{u}}_2+\frac{1}{h_3}\frac{\partial\phi}{\partial u_3}\hat{\mathbf{u}}_3\;.
\end{equation}

Luego:
\begin{equation}\label{1ec74}
	d\phi=\nabla\phi\cdot d\mathbf{r}\;.
\end{equation}

Si ahora se consideran dos puntos $P$ y $Q$ en la superficie $\phi(u_1,u_2,u_3)=C$ (siendo $C$ una constante) de manera que $P$ est� a una distancia $dr$ de $Q$; el cambio en la funci�n $\phi$ debido al movimiento del punto $P$ a $Q$ es:
\begin{equation}
	d\phi=\nabla\phi\cdot d\mathbf{r}=0\;,
\end{equation}

con lo cual se concluye que $\nabla\phi$ es perpendicular a $d\mathbf{r}$, el vector diferencial que va de $P$ a $Q$, como se observa en la figura \ref{1g20}. 
\begin{figure}[ht]
	\centering
		\includegraphics{1g20.20}
	\caption{Perpendicularidad entre $\nabla\phi$ y $d\mathbf{r}$.}	
	\label{1g20}
\end{figure}

En consecuencia, el vector $\nabla\phi$ es ortogonal a la superficie $\phi(u_1,u_2,u_3)=C$, ya que $d\mathbf{r}$ es un vector tangente a ella. 

Ahora se consideran los puntos $P$ y $Q$ en diferentes superficies $\phi(u_1,u_2,u_3)=C_1$ y $\phi(u_1,u_2,u_3)=C_2$ respectivamente y de nuevo se asigna el vector $d\mathbf{r}$ como el vector que nace en $P$ (en alg�n lugar de $\phi=C_1$) y que termina en $Q$ (en otro lugar de $\phi=C_2$). Tomando fijo el punto $P$ y dejando libre $Q$, el m�ximo valor de $d\phi$ se logra cuando $\nabla\phi$ y $d\mathbf{r}$ son paralelos (ver figura \ref{1g21}), seg�n \eqref{1ec74}. 
\begin{figure}[ht]
	\centering
		\includegraphics{1g21.21}
	\caption{El valor m�ximo de $d\phi$ se logra cuando $\nabla\phi$ y $d\mathbf{r}$ son paralelos.}	
	\label{1g21}
\end{figure}

Se asocia en consecuencia a la direcci�n $\nabla\phi$, la direcci�n de m�ximo crecimiento de $\phi(u_1,u_2,u_3)$.

\subsection{La divergencia $\nabla\cdot$}

Se presentar� el operador divergencia desde la �ptica de su significado f�sico, considerando la divergencia del vector densidad de flujo de masa $\mathbf{J}$\footnote{$\mathbf{J}$ es la cantidad de masa por unidad de �rea por unidad de tiempo.}. Sup�ngase un elemento de volumen en coordenadas curvil�neas como se muestra en la figura \ref{1g22}, por el cual pasa un fluido sin ninguna direcci�n de preferencia. 
\begin{figure}[ht]
	\centering
		\includegraphics{1g22.22}
	\caption{Elemento de volumen en coordenadas curvil�neas.}	
	\label{1g22}
\end{figure}

El fluido circula en direcciones arbitrarias y atraviesa el elemento de volumen. La superficie l�mite en $u_1=u_{10}+\Delta u_1$ se llamar� $S_1$; la superficie en $u_1=u_{10}$ se denominar� $S_1'$; $S_2$ y $S_2'$ estar�n en $u_2=u_{20}+\Delta u_2$ y $u_2=u_{20}$ respectivamente; y $S_3$ y $S_3'$ estar�n en $u_3=u_{30}+\Delta u_3$ y $u_3=u_{30}$. Ver figura \ref{1g23}. 
\begin{figure}[ht]
	\centering
		\includegraphics{1g23.23}
	\caption{Superficies l�mites del elemento de volumen en coordenadas curvil�neas.}	
	\label{1g23}
\end{figure}

El fluido que atraviesa la cara $S_1$ hacia afuera del volumen en la unidad de tiempo es:
\begin{equation}
	\left.\mathbf{J}\cdot\hat{\mathbf{u}}_1\,h_2\,h_3\,\Delta u_2\,\Delta u_3\,\right|_{u_1=u_{10}+\Delta u_1}\;.
\end{equation}

Asimismo para la cara $S'_1$ es:
\begin{equation}
	\left.-\mathbf{J}\cdot \hat{\mathbf{u}}_1\,h_2\,h_3\,\Delta u_2\,\Delta u_3\,\right|_{u_1=\Delta u_1}\;.
\end{equation}

N�tese que las componentes de $\mathbf{J}$ en $\hat{\mathbf{u}}_2$ y $\hat{\mathbf{u}}_3$ no se toman en cuenta para las caras $S_1$ y $S'_1$ ya que son tangenciales a las superficies y no aportan al flujo. El flujo neto de salida en la unidad de tiempo, debido exclusivamente a las superficies $S_1$ y $S'_1$ es:
\begin{equation}\label{1ec78}
	\left.J_1\,h_2\,h_3\,\Delta u_2\,\Delta u_3\,\right|_{u_1=u_{10}+\Delta u_1}-\left.J_1\,h_2\,h_3\,\Delta u_2\,\Delta u_3\,\right|_{u_1=u_{10}}\;.
\end{equation}

Ahora el flujo neto de salida por unidad de tiempo por unidad de volumen, debido a $S_1$ y $S'_1$ es:
\begin{equation}
\left.
\begin{aligned}
	&\text{Rata de flujo}\\
	&\text{de salida}\\
	&\text{por unidad}\\
	&\text{de volumen}
\end{aligned}
\right|_{S_1S'_1}
\begin{aligned}
	=\frac{\left.J_1h_2h_3\Delta u_2 \Delta u_3\right|_{u_1=u_{10}+\Delta u_1} - \left.J_1h_2h_3\Delta u_2 \Delta u_3\right|_{u_1=\Delta u_1}}{h_1h_2h_3\Delta u_1 \Delta u_2 \Delta u_2}
\end{aligned}\;.
\end{equation}

En el paso al l�mite $(\Delta u_1,\Delta u_2,\Delta u_3\rightarrow0)$ se obtiene:
\begin{equation}
\left.
\begin{aligned}
	&\text{Rata de flujo de salida}\\
	&\text{por unidad de volumen}
\end{aligned}
\right|_{S_1S'_1}
\begin{aligned}
	=\frac{1}{h_1h_2h_3}\frac{\partial h_2h_3J_1}{\partial u_1}
\end{aligned}\;.
\end{equation}

En consecuencia, la rata de flujo de salida por unidad de volumen para todo el elemento de volumen (al considerarse todas las caras) es: 
\begin{equation}
	\frac{1}{h_1h_2h_3}\left( \frac{\partial h_2h_3J_1}{\partial u_1}+\frac{\partial h_1h_3J_2}{\partial u_2}+\frac{\partial h_1h_2J_3}{\partial u_3} \right)=\nabla\cdot\mathbf{J}\;.
\end{equation}

A la anterior operaci�n sobre $\mathbf{J}$ se le asigna el nombre de ``divergencia de $\mathbf{J}$'' o $\nabla\cdot\mathbf{J}$. 

Si $\nabla\cdot\mathbf{J}$ evaluada en un punto $\mathbf{r}_0$ es mayor que cero ($\left.\nabla\cdot\mathbf{J}\right|_{\mathbf{r}_0}>0$) entonces del volumen diferencial $\mathbf{r}_0$ est� saliendo m�s fluido del que entra y se afirma que $\mathbf{r}_0$ es un punto fuente de $\mathbf{J}$. Si por el contrario $\nabla\cdot\mathbf{J}$ en $\mathbf{r}_0$ es menor que cero ($\left.\nabla\cdot\mathbf{J}\right|_{\mathbf{r}_0}<0$), al volumen diferencial $\mathbf{r}_0$ est� entrando m�s fluido del que sale y se denomina a $\mathbf{r}_0$ como punto sumidero de $\mathbf{J}$. 

Resumiendo: 
\begin{equation}
\begin{aligned}
	\left.\nabla\cdot\mathbf{J}\right|_{\mathbf{r}_0}=
\end{aligned}
\left\{\begin{aligned}
	&<0;\qquad \mathbf{r}_0\text{ es una fuente de }\mathbf{J}.\\
	&>0;\qquad \mathbf{r}_0\text{ es un sumidero de }\mathbf{J}.\\
	&=0;\qquad \mathbf{r}_0\text{ no es ni fuente ni sumidero de }\mathbf{J}.
\end{aligned}\right.
\end{equation}

El tercer caso se da cuando al volumen diferencial entra la misma cantidad de fluido de la que sale. 

La operaci�n divergencia tambi�n se puede definir utilizando la integral de superficie as�:
\begin{equation}\label{1ec83}
	\nabla\cdot{J}=\lim_{\Delta V\to0}\frac{\oiint\limits_{\mathbf{S}}\mathbf{J}\cdot d\mathbf{S}}{\Delta V}\;.
\end{equation}

N�tese que s�lo hay que interpretar \eqref{1ec78} como el resultado parcial de la integral de superficie sobre la frontera del elemento de volumen, luego el flujo de $\mathbf{J}$ sobre la frontera del elemento de volumen es: 
\begin{equation}
\begin{aligned}
	\oiint\limits_{\mathbf{S}}\mathbf{J}\cdot d\mathbf{S}=
	&\left.J_1h_2h_3\right|_{u_1=-\Delta u_10}-
	\left.J_1h_2h_3\right|_{u_1=0}+
	\left.J_2h_1h_3\right|_{u_2=-\Delta u_2}\\
	&-\left.J_2h_1h_3\right|_{u_2=0}+
	\left.J_3h_1h_2\right|_{u_3=-\Delta u_3}-
	\left.J_3h_1h_2\right|_{u_3=0}\;.
\end{aligned}
\end{equation}

Al sacar el l�mite se concluye \eqref{1ec83}.

\subsection{El teorema de la divergencia}

Para integrar $\mathbf{J}$ sobre una superficie cerrada $S$ se puede utilizar el siguiente procedimiento: sea la superficie cerrada $S$, la frontera del volumen $V$ que est� fragmentado en peque�as porciones $\Delta V_i$. Cada $\Delta V_i$ est� limitado por una superficie $S_i$ como se muestra en la figura \ref{1g24}.
\begin{figure}[ht]
	\centering
		\includegraphics{1g24.24}
	\caption{$S_i$ es la frontera del volumen $\Delta V_i$ que consta de 6 superficies abiertas.}	
	\label{1g24}
\end{figure}

La integral buscada es:
\begin{equation}
	\oiint\limits_{\mathbf{S}}\mathbf{J}\cdot d\mathbf{S}=\lim_{N\to\infty}\sum_{i=1}^{N}\iint\limits_{S_i}\mathbf{J}\cdot d\mathbf{S}_i\;.
\end{equation}

Cada superficie com�n interior entre los vol�menes diferenciales tiene el flujo saliendo de un volumen (contribuci�n positiva), y entrando exactamente al volumen adyacente (contribuci�n negativa). La contribuci�n neta al flujo de la integral de superficie $\iint\limits_{S}\mathbf{J}\cdot d\mathbf{S}$ es cero para todas las superficies interiores. Contribuciones al flujo diferentes de cero, se obtienen �nicamente para aquellas superficies que pertenecen a la superficie exterior $S$ de $V$. Pero de \eqref{1ec83} se sabe que: 
\begin{equation}
	\nabla\cdot\mathbf{J}\;\Delta V_i=\iint\limits_{S_i}\mathbf{J}\cdot d\mathbf{S}_i\;.
\end{equation}

Luego:
\begin{equation}
	\oiint\limits_{\mathbf{S}}\mathbf{J}\cdot d\mathbf{S}=\lim_{N\to\infty}\sum_{i=1}^{N}\nabla\cdot\mathbf{J}\;\Delta V_i=\iiint\limits_{V}\nabla\cdot\mathbf{J}\,dV\;.
\end{equation}

De la adici�n de todas las contribuciones de cada volumen diferencial, se obtiene el teorema de la divergencia. 
\begin{equation}
\oiint\limits_{S}\mathbf{J}\cdot d\mathbf{S}=\iiint\limits_{V}\nabla\cdot\mathbf{J}\,dV\;.
\end{equation}

Este poderoso teorema convierte una integral se superficie en una integral de volumen equivalente. El resultado aqu� expuesto ser� muy �til en el desarrollo de la Teor�a Electromagn�tica. 

\subsection{El rotacional $\nabla\times$}
Se presentar� el rotacional desde su significado f�sico, de manera an�loga como se hizo con la divergencia. Consid�rese un fluido, y asociada a cada part�cula del fluido una fuerza $\mathbf{F}$ que se considera como un campo de fuerzas. Se estudiar� la integral de l�nea sobre una curva cerrada (la circulaci�n de $\mathbf{F}$) as�:
\begin{equation}\label{1ec89}
	C_1=\int\limits_{abcda}\mathbf{F}\cdot d\mathbf{r}=\oint\limits_{C}\mathbf{F}\cdot d\mathbf{r}\;,
\end{equation}

donde \eqref{1ec89} es el trabajo hecho por el campo en la trayectoria cerrada $C$, formada por las l�neas paralelas a las l�neas coordenadas en el plano $u_1=\Delta u_1$.
\begin{figure}[ht]
	\centering
		\includegraphics{1g25.25}
	\caption{Curva cerrada $C$ en coordenadas curvil�neas.}	
	\label{1g25}
\end{figure}
\begin{equation}
\begin{aligned}
	C_1=&\left.\int\limits_{u_{20}}^{u_{20}+\Delta u_2}\mathbf{F}\cdot h_2\,du_2\,\hat{\mathbf{u}}_2\right|_{u_3=\Delta u_{3}}+
	\left.\int\limits_{u_{30}}^{u_{30}+\Delta u_3}\mathbf{F}\cdot h_3\,du_3\,\hat{\mathbf{u}}_3\right|_{u_2=u_{20}+\Delta u_{2}}\\
	&+\left.\int\limits_{u_{20}+\Delta u_2}^{u_{20}}\mathbf{F}\cdot h_2\,du_2\,\hat{\mathbf{u}}_2\right|_{u_3=u_{30}+\Delta u_{3}}+
	\left.\int\limits^{u_{30}}_{u_{30}+\Delta u_3}\mathbf{F}\cdot h_3\,du_3\,\hat{\mathbf{u}}_3\right|_{u_2=u_{20}}\;.
\end{aligned}
\end{equation}

Se observa que:
\begin{equation}
\begin{aligned}
	C_1=&\left.\int\limits_{u_{20}}^{u_{20}+\Delta u_2}\mathbf{F}\cdot h_2\,du_2\,\hat{\mathbf{u}}_2\right|_{u_3=\Delta u_{3}}-
	\left.\mathbf{F}\cdot h_2\,du_2\,\hat{\mathbf{u}}_2\right|_{u_3=u_{30}+\Delta u_{3}}\\
	&+\left.\int\limits^{u_{30}+\Delta u_3}_{u_{30}}\mathbf{F}\cdot h_3\,du_3\,\hat{\mathbf{u}}_2\right|_{u_2=\Delta u_{2}}-
	\left.\mathbf{F}\cdot h_3\,du_3\,\hat{\mathbf{u}}_3\right|_{u_2=u_{20}}\;.
\end{aligned}
\end{equation}

Al evaluar las integrales se obtiene:
\begin{equation}
\begin{aligned}
	C_1\approx 
	&\left.F_3\cdot h_3\,du_3\right|_{u_2=u_{20}+\Delta u_{2}}-
	\left.F_3\cdot h_3\,du_3\right|_{u_2=u_{20}}\\
	&-\left.F_2\cdot h_2\,du_2\right|_{u_3=u_{30}+\Delta u_{3}}-
	\left.F_2\cdot h_2\,du_2\right|_{u_3=u_{30}}\;,
\end{aligned}
\end{equation}

que es igual a:
\begin{equation}
\begin{aligned}
	C_1\approx 
	&\left[\frac{\left.F_3\cdot h_3\,du_3\right|_{u_2=u_{20}+\Delta u_{2}}-
	\left.F_3\cdot h_3\,du_3\right|_{u_2=u_{20}}}{h_2\,h_3\Delta u_2\Delta u_3}\right]h_2\,h_3\,\Delta u_2\Delta u_3 \\
	&-\left[\frac{\left.F_2\cdot h_2\,du_2\right|_{u_3=u_{30}+\Delta u_{3}}-
	\left.F_2\cdot h_2\,du_2\right|_{u_3=u_{30}}}{h_2\,h_3\,\Delta u_2\Delta u_3}\right]h_2\,h_3\,\Delta u_2\Delta u_3\;.
\end{aligned}
\end{equation}

En el l�mite se tiene: 
\begin{equation}
	\lim_{C_1\to0}C_1\approx \frac{1}{h_2\,h_3}\left[ \frac{\partial F_3\,h_3}{\partial u_2}-\frac{\partial F_2\,h_2}{\partial u_3} \right]h_2\,h_3\,\Delta u_2 \Delta u_3\;.
\end{equation}

Y se admite por definici�n que la componente en $u_1$ del rotacional de $\mathbf{F}$ es:
\begin{equation}\label{1ec95}
	\left.\nabla\times\mathbf{F}\right|_1=\nabla\times\mathbf{F}\cdot\hat{\mathbf{u}}_1=
	\lim_{\Delta u_2 \Delta u_3\to0}\frac{\oint\limits_{C}\mathbf{F}\cdot d\mathbf{r}}{h_2\,h_3\,\Delta u_2 \Delta u_3}\;.
\end{equation}

Finalmente: 
\begin{equation}
	\left.\nabla\times\mathbf{F}\,\right|_1=\frac{1}{h_2\,h_3}\left[ \frac{\partial F_3\,h_3}{\partial u_2}-\frac{\partial F_2\,h_2}{\partial u_3} \right]\;.
\end{equation}

El rotacional completo se puede escribir as�: 
\begin{equation}
\nabla\times\mathbf{F}=\frac{1}{h_1h_2h_3}
\begin{vmatrix}
h_1\hat{\mathbf{u}}_1 & h_2\hat{\mathbf{u}}_2 & h_3\hat{\mathbf{u}}_3\\
\partial/\partial u_1 & \partial/\partial u_2 & \partial/\partial u_3\\
h_1F_1 & h_2F_2 & h_3F_3
\end{vmatrix}\;.
\end{equation}

�Qu� significa todo esto?

Si en la figura \ref{1g25} se reemplaza la trayectoria $C$ por una manguera que no oponga resistencia al flujo, el fluido circular� en la direcci�n considerada convencionalmente positiva (es decir, al utilizar la regla de la mano derecha para la orientaci�n de la superficie cuya frontera es la curva $C$) si la componente del rotacional en $u_1$ es tambi�n positiva. Ver figura \ref{1g26}.
\begin{figure}[ht]
	\centering
		\includegraphics{1g26.26}
	\caption{El flujo va en la direcci�n positiva 1234, seg�n la regla de la mano derecha.}	
	\label{1g26}
\end{figure}

El rotacional de $\mathbf{F}$ puede tener componentes en $\hat{\mathbf{u}}_1$, $\hat{\mathbf{u}}_2$ y $\hat{\mathbf{u}}_3$ y puede escribirse como:
\begin{equation}
	\nabla\times\mathbf{F}=\Gamma_1\hat{\mathbf{u}}_1+\Gamma_2\hat{\mathbf{u}}_2+\Gamma_3\hat{\mathbf{u}}_3\;.
\end{equation}

Este rotacional se puede interpretar como un vector perpendicular a un elemento de superficie, que determina la direcci�n del flujo en una manguera colocada en la frontera l�mite del elemento. Debe utilizarse la forma convencional de orientar superficies abiertas por medio de sus vectoress normales y la aplicaci�n de la regla de la mano derecha; adicionalmente, la manguera debe considerarse ideal, es decir, que no oponga resistencia al flujo. Ver figura \ref{1g27}.
\begin{figure}[ht]
	\centering
		\includegraphics{1g27.27}
	\caption{El flujo va en la direcci�n positiva 1234, seg�n la regla de la mano derecha.}	
	\label{1g27}
\end{figure}

\subsection{El teorema de Stokes}

Al integrar la funci�n $\mathbf{F}$ sobre una trayectoria cerrada para calcular su circulaci�n, es posible realizar lo siguiente: una de las superficies sugeridas por $\mathbf{C}$ se subdivide en pedazos. Ver figura \ref{1g28}.
\begin{figure}[ht]
	\centering
		\includegraphics{1g28.28}
	\caption{Partici�n de la curva cerrada $C_l$  l�mite de la superficie $S_l$.}	
	\label{1g28}
\end{figure}

La circulaci�n de $\mathbf{F}$ sobre $C$ es:
\begin{equation}
	\oint\limits_{C}\mathbf{F}\cdot d\mathbf{r}=\lim_{N\to\infty}\sum\limits_{l=1}^{N}\oint\limits_{C}\mathbf{F}\cdot d\mathbf{r}_1\;.
\end{equation}

Donde la trayectoria cerrada $C_l$ es la frontera del elemento de superficie $S_l$. Obs�rvese que las trayectorias $C_{l-1}$ y $C_l$ comparten el camino $12$ y cuando se realiza la suma de las circulaciones. El aporte de la trayectoria $C_{l-1}$ all� se anula con el aporte de $C_l$. S�lo existe un aporte no nulo para las trayectorias que contienen puntos de la frontera de $S$. 

El elemento de superficie $S_l$ se puede escribir como: 
\begin{equation}
d\mathbf{S}_1=h_1\,h_2\,du_1\,du_2\,\hat{\mathbf{u}}_3
	+h_1\,h_3\,du_1\,du_3\,\hat{\mathbf{u}}_2
	+h_2\,h_3\,du_2\,du_3\hat{\mathbf{u}}_1\;.
\end{equation}

De \eqref{1ec95} se sabe que:
\begin{equation}
	\nabla\times\mathbf{F}\cdot\hat{\mathbf{u}}_i\,h_j\,h_k\,\Delta u_j\Delta u_k=\oint\limits_{C_{li}}\mathbf{F}\cdot d\mathbf{r}\;.
\end{equation}

Siendo $C_{li}$ la trayectoria $C_l$ proyectada sobre $u_i=0$, y:
\begin{equation}
	\oint\limits_{C_{l1}}\mathbf{F}\cdot d\mathbf{r}
	+\oint\limits_{C_{l2}}\mathbf{F}\cdot d\mathbf{r}
	+\oint\limits_{C_{l3}}\mathbf{F}\cdot d\mathbf{r}
	=\oint\limits_{C_{l}}\mathbf{F}\cdot d\mathbf{r}\;.
\end{equation}

Se sigue que:
\begin{equation}
\begin{aligned}
	\oint\limits_{C_{l}}\mathbf{F}\cdot d\mathbf{r}&= 
	\nabla\times\mathbf{F}\cdot\hat{\mathbf{u}}_1\,h_2\,h_3\,\Delta u_2\Delta u_3+
	\nabla\times\mathbf{F}\cdot\hat{\mathbf{u}}_2\,h_1\,h_3\,\Delta u_1\Delta u_3+\\
	&\nabla\times\mathbf{F}\cdot\hat{\mathbf{u}}_3\,h_1\,h_2\,\Delta u_1\Delta u_2
	=\nabla\times\mathbf{F}\cdot\Delta\mathbf{S}_l\;.
\end{aligned}
\end{equation}

En consecuencia:
\begin{equation}
	\oint\limits_{C}\mathbf{F}\cdot d\mathbf{r}
	=\lim_{N\to\infty}\sum\limits_{l=1}^{N}{\nabla\times\mathbf{F}\cdot\Delta\mathbf{S}_l}
	=\iint\limits_{S}\nabla\times\mathbf{F}\cdot d\mathbf{S}\;.
\end{equation}

El resultado del teorema de Stokes es:
\begin{equation}
	\oint\limits_{C}\mathbf{F}\cdot d\mathbf{r}=
	\iint\limits_{S}\nabla\times\mathbf{F}\cdot d\mathbf{S}\;.
\end{equation}

Con este �til teorema se pueden realizar integrales de superficie haciendo la integral de l�nea adecuada.

\section{El �ngulo S�lido}

Sea $dS$ un elemento de superficie en $R^3$ de una superficie $S$. Si se unen todos los puntos del contorno de $dS$ con un punto $p$ resulta un cono como se muestra en la figura \ref{1g29}.
\begin{figure}[ht]
	\centering
		\includegraphics{1g29.29}
	\caption{Base del cono $dW$ producto de la proyecci�n de los puntos de la curva l�mite de $dS$ al punto $P$.}	
	\label{1g29}
\end{figure}

Ahora tr�cese una esfera de radio r centrada en $p$. La porci�n de esfera interceptada por el cono es $dW$. El �ngulo s�lido con que se ``ve'' a $dS$ desde $p$ es, por definici�n, $d\Omega=\frac{dW}{r^2}$ y es el valor del �rea de la porci�n de esfera unitaria interceptada por el cono. 

Si $\hat{\mathbf{n}}$ es el vector normal unitario exterior a $dS$ y $\alpha$ el �ngulo formador por $\hat{\mathbf{n}}$ y $\mathbf{r}$, se tiene:
\begin{equation*}
\begin{aligned}
	&d\Omega=\frac{dW}{r^2}=\frac{dS\cos\alpha}{r^2}=\frac{dS}{r^2}\cdot\frac{\mathbf{r}\cdot\hat{\mathbf{n}}}{r}\\
	&d\Omega=\frac{\mathbf{r}\cdot\hat{\mathbf{n}}\,dS}{r^3}=\frac{\mathbf{r}\cdot d\mathbf{S}}{r^3}
\end{aligned}
\end{equation*}

Luego, el �ngulo s�lido subtentido por $S$ en $p$ es:
\begin{equation}
	\Omega=\iint\limits_{S}d\Omega=\iint\limits_{S}\frac{\mathbf{r}\cdot d\mathbf{S}}{r^3}\;.
\end{equation}

\subsection{Ilustraci�n}

Halle el �ngulo s�lido subtendido por una superficie cerrada  

\subsubsection{Origen no contenido en el volumen.}
\label{sec:OrigNoContVol}

\begin{figure}[ht]
	\centering
		\includegraphics{1g30.30}
	\caption{El origen est� fuera de la superficie cerrada $S$.}	
	\label{1g30}
\end{figure}

El �ngulo s�lido subtendido por $S$ desde $O$ es:
\begin{equation}
	\Omega=\iint\limits_{S}\frac{\mathbf{r}\cdot d\mathbf{S}}{r^3}
\end{equation}

Siendo $\mathbf{r}$ un vector que va desde $O$ a cualquier punto de la superficie. Aplicando el teorema de la divergencia,
\begin{equation}
	\iint\limits_{S}\frac{\mathbf{r}\cdot d\mathbf{S}}{r^3}=
	\iiint\limits_{V}\nabla\cdot\frac{\mathbf{r}}{r^3}\,dV
\end{equation}

Luego,

\begin{equation*}
	\begin{aligned}\medskip
		\nabla\cdot\left(\frac{\mathbf{r}}{r^3}\right)=&\nabla\cdot\left( \frac{x\hat{\mathbf{x}}+y\hat{\mathbf{y}}+z\hat{\mathbf{z}}}{(x^2+y^2+z^2)^{3/2}} \right)\\ \medskip
\nabla\cdot\left(\frac{\mathbf{r}}{r^3}\right)=&\frac{1}{(x^2+y^2+z^2)^{3/2}}+\frac{x(-3/2)(2x)}{(x^2+y^2+z^2)^{5/2}}\\ \medskip
+&\frac{1}{(x^2+y^2+z^2)^{3/2}}+\frac{y(-3/2)(2y)}{(x^2+y^2+z^2)^{5/2}}\\ \medskip
+&\frac{1}{(x^2+y^2+z^2)^{3/2}}+\frac{z(-3/2)(2z)}{(x^2+y^2+z^2)^{5/2}}
	\end{aligned}
\end{equation*}

\begin{equation*}
\nabla\cdot\left(\frac{\mathbf{r}}{r^3}\right)=\frac{3(x^2+y^2+z^2)-3x^2-3y^2-3z^2}{(x^2+y^2+z^2)^{5/2}}=0
\end{equation*}

\begin{equation}
	\nabla\cdot\left(\frac{\mathbf{r}}{r^3}\right)=0
\end{equation}
Siempre y cuando el origen ($x=y=z=0$) no sea un punto del volumen. A continuaci�n se sigue que $\Omega=0$.

\subsubsection{Origen contenido en el volumen.}

Ahora consid�rese el caso en que el origen del �ngulo s�lido est� encerrado por $S$. Inicialmente se coloca una esfera centrada en $O$ que est� completamente contenida en el volumen sugerido por $S$. 
\begin{figure}[ht]
	\centering
		\includegraphics{1g31.31}
	\caption{El origen est� dentro de la superficie cerrada $S$.}	
	\label{1g31}
\end{figure}


Se define la regi�n $V^*$ como el volumen exterior a la esfera que est� encerrado por $S$. Es decir, $V^*=V-V'$; siendo $V$ el volumen encerrado por $S$ y $V'$ el volumen encerrado por la esfera $S'$. Sea $S^*$ la superficie l�mite de $V^*: S^*=S+S'$. Se necesita conocer el �ngulo $\Omega$ (�ngulo s�lido subtendido por $S$ desde $O$) y para ello se calcula inicialmente $\Omega^*$ (�ngulo s�lido subtendido por $S^*$). Atendiendo el resultado de (\ref{sec:OrigNoContVol}) ese �ngulo es nulo ($\Omega^*=0$) ya que el volumen $V^*$ no contiene a $O$.


\begin{equation*}
\Omega^*=\iint\limits_{S^*}\frac{\mathbf{r}\cdot d\mathbf{S}}{r^3}=0=\iint\limits_{S}\frac{\mathbf{r}\cdot d\mathbf{S}}{r^3}+\iint\limits_{S'}\frac{\mathbf{r}\cdot d\mathbf{S}}{r^3}
\end{equation*}

Luego:
\begin{equation*}
\begin{aligned}
	\iint\limits_{S}\frac{\mathbf{r}\cdot d\mathbf{S}}{r^3}=&-\iint\limits_{S'}\frac{\mathbf{r}\cdot d\mathbf{S}}{r^3}\\
	\iint\limits_{S}\frac{\mathbf{r}\cdot d\mathbf{S}}{r^3}=&-\iint\limits_{S'}\frac{a\hat{\mathbf{r}}\cdot -a^2\sen\theta\,d\theta\,d\phi\,\hat{\mathbf{r}}}{a^3}=-\iint\limits_{S'}-\sen\theta\,d\theta\,d\phi  \\
	\Omega=&\iint\limits_{S}\frac{\mathbf{r}\cdot d\mathbf{S}}{r^3}=2\pi\int\limits_{0}^{\pi}\sen\theta\,d\theta=4\pi
\end{aligned}
\end{equation*}

Se concluye que el �ngulo s�lido subtendido por una superficie cerrada es: nulo si el origen del �ngulo no est� encerrado por la superficie o $4\pi$ si el origen est� contenido en el volumen sugerido por la superficie. 





\chapter{Carga y corriente.}\label{cap1}

\section{Carga el�ctrica.}
La carga el�ctrica es una propiedad intr�nseca de la materia. Es ampliamente conocido que cargas del mismo signo se repelen y de signos contrarios se atraen. Ese comportamiento est� cuantificado por la ley de Coulomb. 
\begin{equation}
	\mathbf{F}=\frac{q_1\,q_2}{4\pi\epsilon_0}\,\frac{\mathbf{r_1}-\mathbf{r_2}}{|\mathbf{r_1}-\mathbf{r_2}|^3}\;.
\end{equation}

\subsection{Caracter�sticas experimentales.}
Solamente aquello que cumpla estas cuatro caracter�sticas listadas a continuaci�n puede considerarse una carga el�ctrica.

\begin{enumerate}
%\subsubsection{Dualidad de la carga.}
\item \textbf{Dualidad. }
La carga el�ctrica se presenta en dos variedades llamadas convencionalmente positiva y negativa.
%\subsubsection{Conservaci�n de la carga.}
\item \textbf{Conservaci�n.} 
La carga se conserva en todo sistema donde no haya transferencia de masa por sus fronteras. Este hecho se explicita en la ley de Amp�re-Maxwell[HACER referencia cruzada a la ley de Amp�re-Maxwell], que es una de las ecuaciones de Maxwell que se estudiar�n m�s adelante. 

%\subsubsection{Cuantizaci�n de la carga.}
\item \textbf{Cuantizaci�n. }
La carga est� cuantizada, es decir, la carga de un objeto siempre es m�ltiplo entero de una carga elemental, convencionalmente elegida como la carga del electr�n \footnote{Existen cargas de valores fraccionarios de la carga del electr�n, por ejemplo, los quarks, con cargas de $\frac{e}{3}$ o $\frac{-2e}{3}$. Sin embargo, en la naturaleza las combinaciones de quarks que son estables siempre se presentan en m�ltiplos enteros de e.}.  Aun hoy (18 de septiembre de 2007) no hay una justificaci�n de por qu� todas las distribuciones de carga son discretas y m�ltiplos de una carga ``unitaria". Un intento muy imaginativo es la soluci�n de Dirac que sigue sin verificarse experimentalmente, pues aun no se detectan los monopolos magn�ticos\footnote{"Modern Quantum Mechanics" by J.J. Sakuray, 140-143.}.

%\subsubsection{Invariabilidad relativista de la carga.}
\item \textbf{Invariabilidad relativista. }
La carga es un invariante relativista, es decir, la medida de la carga no cambia desde diferentes sistemas de referencia en movimiento.
\end{enumerate}

\subsection{Distribuciones continuas de carga.}
El electromagnetismo es un �rea muy basta de la f�sica. Internamente se puede dividir de m�ltiples maneras. Una de ellas es considerar los sistemas cargados como distribuciones continuas o discretas. 

La realidad muestra que todas las distribuciones de carga son discretas. Sin embargo, en algunos casos, es mucho m�s sencillo considerarlas continuas. Esta aproximaci�n s�lo funciona para conglomerados de muchas part�culas, donde la distancia promedio entre cargas es despreciable respecto de la distancia de observaci�n. A tales distribuciones de carga se les asocia una densidad volum�trica.

Una construcci�n te�rica del electromagnetismo se puede realizar suponiendo que todos los objetos cargados a estudiar son distribuciones volum�tricas. Ello exige que exista un mecanismo para entender como vol�menes cualquier distribuci�n superficial, lineal o discreta. 

\subsubsection{Distribuciones volum�tricas de carga.} 
La distribuci�n volum�trica de carga se define como
\begin{equation}
	\rho(\mathbf{r})=\frac{dQ}{dV}\;, 
\end{equation}

siendo $V$ el volumen del sistema y $Q$ la carga neta contenida en el volumen. $\rho$ se conoce tambi�n como la densidad volum�trica de carga.

Primero que nada, hay que explicar en el contexto de estas aproximaciones, qu� se entender� por distribuci�n superficial, lineal o puntual de carga.

\subsubsection{Distribuciones superficiales de carga.} 
Por definici�n, se acepta que una distribuci�n superficial es un volumen cargado donde una dimensi�n es despreciable respecto de la distancia de observaci�n y se le asocia una densidad superficial de carga $\sigma$ definida como:
\begin{equation}\label{3ec1.3}
	\sigma(\mathbf{r})=\frac{dQ}{dS}\;, 
\end{equation}

siendo $S$ la superficie del sistema y $Q$ la carga neta contenida en esa superficie. 

\subsubsection{Distribuciones lineales de carga.} 
De igual forma, una distribuci�n lineal de carga es un volumen cargado donde dos dimensiones son despreciables respecto de la distancia de observaci�n. La densidad lineal $\lambda$ es la que determina c�mo se distribuye la carga y se define como
\begin{equation}
	\lambda(\mathbf{r})=\frac{dQ}{dr}\;, 
\end{equation}

siendo $\mathbf{r}$ la forma param�trica de la l�nea y $Q$ su carga.

\subsubsection{Distribuciones discretas de carga.} 

Son distribuciones volum�tricas de carga con todas sus dimensiones despreciables respecto a la distancia de observaci�n. Un ejemplo de �stas es la carga puntual, considerada como un volumen cargado con las tres dimensiones despreciables.
 
\subsection{Asociaci�n de densidades volum�tricas de carga a distribuciones no volum�tricas.}\label{AsoDenVolCar}
Normalmente se realiza este tipo de asociaci�n buscando un conjunto m�s compacto de ecuaciones, as� se puede hablar solo de distribuciones volum�tricas y como casos particulares: distribuciones discretas, lineales y superficiales. Esto se realiza utilizando la funci�n delta de Dirac. 

\subsubsection{La funci�n delta de Dirac.}
Se designa con la letra griega min�scula $\delta$ y por definici�n, es:
\begin{equation}\label{ec1.5}
	\int_{a}^{b}{\delta(x)\,dx}=\left\{
	\begin{aligned}
	&1 & &\text{si $0\in (a,b)$}.\\
	&0 & &\text{si $0\notin (a,b)$}.
	\end{aligned}
	\right.
\end{equation}

Se puede demostrar, a partir de \eqref{ec1.5} que:
\begin{equation}
	\int_{a}^{b}{\delta(x-x_0)\,dx}=\left\{
	\begin{aligned}
	&1 & &\text{si $x_0\in(a,b)$}.\\
	&0 & &\text{si $x_0\notin(a,b)$}.
	\end{aligned}
	\right.
\end{equation}

Otras propiedades son: 
\begin{gather}
	\int_{a}^{b}{f(x)\,\delta(x-x_0)\,dx}=\left\{\begin{aligned}f(&x_0) & &\text{si $x_0\in(a,b)$}.\\
		&0 & &\text{si $x_0\notin(a,b)$}.\end{aligned}\right.\label{2ec1.6} \\
		\int_{a}^{b}{f(x)\,\delta'(x-x_0)\,dx}=\left\{\begin{aligned}-f'&(x_0) & &\text{si $x_0\in(a,b)$}.\\
		&0 & &\text{si $x_0\notin(a,b)$}.\end{aligned}\right. \\
		\int_{a}^{b}{f(x)\,\delta^{(n)}(x-x_0)\,dx}=\left\{\begin{aligned}(-1)^{n}&\,f^{(n)}(x_0) & 
		&\text{si $x_0\in(a,b)$}.\\
		&0 & &\text{si $x_0\notin(a,b)$}.\end{aligned}\right.
\end{gather}

\subsubsection{Asociar una densidad volum�trica de carga a una distribuci�n superficial de carga.}
Sea una superficie gen�rica definida por $\mathbf{r}=\mathbf{r}(u_i,u_j)$, siendo $u_k=u_{k0}$ (una constante). Si la densidad de carga es $\sigma=\sigma(u_i,u_j)$, y se requiere hallar una distribuci�n volum�trica de carga equivalente, se puede decir que 
\[
Q=\iint\limits_{S}\sigma\,dS=\iiint\limits_{V}{\rho\,dV}\;, 
\]
donde el volumen $V$ contiene completamente a la superficie $S$. Luego, la integral de volumen se puede reescribir como sigue:
\begin{equation*}
\iint\limits_{S}\sigma\,dS=\iiint\limits_{V}{\rho\,h_k\,du_k\,dS}=
	\iint\limits_{S}\left[\int\limits_{u_k}\rho\,h_k\,du_k\right]\,dS
\end{equation*}

Finalmente, para que se sostenga la igualdad debe ocurrir que, 
\begin{equation*}
	\rho=\frac{\sigma\,\delta(u_k-u_{k0})}{h_k}\;.
\end{equation*}

\subsubsection{Asociar una densidad volum�trica de carga a una distribuci�n lineal.}
Consid�rese una l�nea gen�rica cuya forma param�trica es $\mathbf{r}=\mathbf{r}(u_i)$, siendo $u_j=u_{j0}$ y $u_k=u_{k_0}$. Si la densidad de carga es $\lambda=\lambda(u_i)$, la densidad volum�trica asociada es:
\begin{equation*}
	\rho=\frac{\lambda\,\delta(u_j-u_{j0})\,\delta(u_k-u_{k0})}{h_j\,h_k}\;.
\end{equation*}


\subsubsection{Asociar una densidad volum�trica a una carga puntual.}


Sup�ngase una carga puntual $q_0$. 
\begin{figure}[ht]
  \centering
		\includegraphics{3g1.1}
	\caption{La carga puntual.}
	\label{fig:3g1}
\end{figure}

La densidad asociada est� en la ecuaci�n \eqref{ec10}, donde $\delta{\left(\,\mathbf{r}-\mathbf{r_0}\right)}$ es la funci�n delta de Dirac con argumento vectorial, para el caso $\mathbf{r_0}=x_0\,\mathbf{\hat{x}}+y_0\,\mathbf{\hat{y}}+z_0\,\mathbf{\hat{z}}$.
\begin{equation}\label{ec10}
\rho=q_0\,\delta(x-x_{0})\,\delta(y-y_{0})\,\delta(z-z_{0})=q_0\,\delta{\left(\,\mathbf{r}-\mathbf{r_0}\right)}\;.
\end{equation}


\section{La corriente el�ctrica.}
La corriente el�ctrica se define como el flujo de carga transversal en la unidad de tiempo.
\begin{equation}\label{ec2.1}
	\mathbf{I}=\frac{dQ}{dt}\;.
\end{equation}

En la ecuaci�n \eqref{ec2.1} no se observa expl�citamente la transversalidad. �sta queda manifiesta en la definici�n de la densidad de corriente.

\subsection{Distribuciones de corriente.}
En el cap�tulo anterior se utiliz� la aproximaci�n continua de los sistemas cargados para asociarles densidades volum�tricas. Cuando estos sistemas entran en movimiento, generan corrientes el�ctricas volum�tricas con sus respectivos casos particulares. Teniendo en cuenta el esquema usado para la carga el�ctrica, en la corriente se considerar� tambi�n que todas las distribuciones son volum�tricas. 

\subsubsection{Distribuci�n volum�trica de corriente.}
Una corriente volum�trica es una distribuci�n volum�trica de carga en movimiento y se le asocia una densidad representada con el vector $\mathbf{J}$ que se define en la ecuaci�n \eqref{ec_DefDenCor}. All� el vector $\mathbf{\hat{n}}$ es el vector normal unitario a la superficie $S$ cuyo elemento diferencial es $dS$.

\begin{equation}\label{ec_DefDenCor}
\mathbf{\hat{n}}\cdot\mathbf{J}=\frac{dI}{dS}\,		\quad , \quad \mathbf{J}\,[=]\frac{A}{m^2}\;.
\end{equation}

Utilizando la ecuaci�n \eqref{ec_DefDenCor} se puede dar una definici�n de la corriente donde queda expl�cito su car�cter transversal \eqref{ec_DefCorVol}.

\begin{equation}\label{ec_DefCorVol}
	\mathbf{I}=\iint\limits_{S}{\mathbf{J}\cdot d\mathbf{S}}\;.
\end{equation}

\subsubsection{Distribuci�n superficial de corriente.}

Cuando una distribuci�n volum�trica de carga entra en movimiento, genera una corriente volum�trica. Si la distribuci�n de carga tiene una dimensi�n despreciable, su movimiento puede generar una corriente superficial. �sto s�lo ocurre cuando $\mathbf{v}\cdot\mathbf{\hat{n}}=0$, siendo $\mathbf{v}$ la velocidad de un punto de la superficie cargada y $\mathbf{\hat{n}}$ el vector unitario normal a la superficie en ese mismo punto. 

\begin{figure}[ht]
	\centering
		\includegraphics{3g2.2}
	\caption{Corriente superficial. El vector $\mathbf{v}$ siempre es perpendicular al vector normal a la superficie $\mathbf{\hat{n}}$.}
	\label{g_CorSup}
\end{figure}

A la distribuci�n superficial de corriente se le asocia una densidad superficial $\mathbf{K}$, definida en la ecuaci�n \eqref{ec_DefCorSup}, donde $\hat{\mathbf{t}}$ es un vector tangente unitario a la superficie y perpendicular a la curva $C$, cuya forma param�trica es $\mathbf{r}=\mathbf{r}(u)$, tiene por elemento diferencial $dr=|d\mathbf{r}|$ y est� contenida en la superficie. Ver gr�fico (\ref{g_DefK}).
\begin{equation}\label{ec_DefCorSup}
\hat{\mathbf{t}}\cdot\mathbf{K}=\frac{dI}{dr}\quad,\quad \mathbf{K}\,[=]\frac{A}{m}\;.
\end{equation}
 
\begin{figure}[ht]
	\centering
		\includegraphics{3g3.3}
	\caption{Definici�n de $\mathbf{K}$.}
	\label{g_DefK}
\end{figure}

\subsubsection{Distribuci�n lineal de corriente.}

Una distribuci�n lineal de carga en movimiento puede generar una corriente filamental. �sto ocurre cuando $\mathbf{v}\times\mathbf{\hat{t}}=\textbf{0}$, siendo $\mathbf{v}$ la velocidad de la carga y $\mathbf{\hat{t}}$ el vector tangente unitario a la l�nea. 

A la corriente filamental se le asocia una densidad lineal $\mathbf{I}$, definida en la ecuaci�n \eqref{ec_DefCorLin}, donde $\mathbf{\hat{t}}$ es el vector tangente a la curva y va en la direcci�n del movimiento de la carga positiva.  
\begin{equation}\label{ec_DefCorLin}
\mathbf{I}=I\,\mathbf{\hat{t}}\quad,\quad \mathbf{I}\,[=]A\;.
\end{equation}

\subsection{Asociar distribuciones volum�tricas de corriente a una distribuci�n no volum�trica.}

Aplicando un procedimiento similar al utilizado en la secci�n \ref{AsoDenVolCar}, se pueden asociar distribuciones volum�tricas de corriente a distribuciones no volum�tricas, como se ilustra a continuaci�n. 

\subsubsection{Asociar una distribuci�n volum�trica de corriente a una distribuci�n superficial de corriente.}

A una superficie gen�rica definida por $\mathbf{r}=\mathbf{r}(u_i,u_j)$, siendo $u_k=u_{k0}$ (una constante), con una densidad superficial de corriente $\mathbf{K}=\mathbf{K}(u_i,u_j)$, se le puede asociar una distribuci�n volum�trica de corriente equivalente de la forma
\begin{equation*}
\mathbf{J}=\frac{\mathbf{K}\,\delta(u_k-u_{k0})}{h_k}\;.
\end{equation*}

\subsubsection{Asociar una distribuci�n volum�trica de corriente a una distribuci�n lineal de corriente.}

Siendo $\mathbf{r}=\mathbf{r}(u_i)$ una l�nea gen�rica con $u_j=u_{j0}$ y $u_k=u_{k_0}$, con una densidad lineal de corriente $\mathbf{I}=\mathbf{I}(u_i)$, se le puede asociar una densidad volum�trica de corriente de la forma
\begin{equation*}
\mathbf{I}=\frac{\mathbf{I}\,\delta(u_j-u_{j0})\,\delta(u_k-u_{k0})}{h_j\,h_k}\;.
\end{equation*}


\subsubsection{Relaci�n entre la densidad de corriente y la velocidad.}


Consid�rese un flujo volum�trico de carga con velocidad $\mathbf{v}$. Siendo consistente con las aproximaciones usadas, para un volumen $\Delta V$ se pueden considerar las variaciones de la velocidad despreciables (un campo constante de velocidades). Ver figura (\ref{g_RelDenCorVel}). 

\begin{figure}[ht]
	\centering
		\includegraphics{3g4.4}
	\caption{Relaci�n entre densidad de corriente y velocidad.}
	\label{g_RelDenCorVel}
\end{figure}

Utilizando la definici�n de densidad volum�trica de corriente
\begin{equation}\label{ec_CorVol}
	\mathbf{\hat{n}}\cdot\mathbf{\hat{J}}=\frac{dI}{dS}=\frac{\Delta I}{\Delta S}
		=\frac{\Delta Q}{\Delta s \, \Delta t} \;.
\end{equation}

El volumen abstracto formado por el tr�nsito de carga a trav�s de $\Delta S$ en un tiempo $\Delta t$ se llamar� $\Delta V$. 
\begin{equation}
	\Delta V = \mathbf{v}\Delta t \cdot\Delta S \,\mathbf{\hat{n}}\;.
\end{equation}

La carga alojada en $\Delta V$ que ha pasado por $\Delta S$ en un tiempo $\Delta t$ es:
\begin{equation}
	\Delta Q=\rho\,\Delta V = \rho\,\mathbf{v}\cdot\mathbf{\hat{n}}\,\Delta t \, \Delta S\;.
\end{equation}

Reemplazando en \eqref{ec_CorVol}:
\begin{equation}\label{ec_Dem}
	\mathbf{\hat{n}}\cdot\mathbf{\hat{J}}=
	\frac{\rho\,\mathbf{v}\cdot\mathbf{\hat{n}}\,\Delta t \, \Delta S}{\Delta t \, \Delta S}
	=\rho\,\mathbf{v}\cdot\mathbf{\hat{n}}\;.
\end{equation}

La ecuaci�n \eqref{ec_Dem} se sostiene para cualquier normal $\mathbf{\hat{n}}$, luego 
\begin{equation}
	\boxed{\mathbf{J}=\rho\,\mathbf{v}}\;.
\end{equation}

Mediante un procedimiento similar se puede demostrar que:
\begin{equation}
	\boxed{\mathbf{K}=\sigma\,\mathbf{v}}\qquad,\qquad
	\boxed{\mathbf{I}=\lambda\,\mathbf{v}} \;.
\end{equation}

\subsubsection{Ilustraciones}

\paragraph{Primera ilustraci�n.}


Calcular la corriente que atraviesa el semiplano $XZ\,(x>0)$ de una esfera cargada con una densidad de carga $\sigma=\sigma_0\cos\frac{\theta}{2}$ si �sta gira sobre su propio eje a una velocidad $\boldsymbol{\omega}=\omega_0\,\mathbf{\hat{z}}$, como se ilustra en la figura \ref{g_EsfCar}. %\large{$ultimate.$}
\begin{figure}[ht]
	\centering
		\includegraphics{E3g1.1}
	\caption{Esfera cargada con movimiento rotacional.}
	\label{g_EsfCar}
\end{figure}

La corriente buscada es de la forma:
\begin{equation}\label{ec_DefCor}
	\mathbf{I}=\int\limits_{c}{\mathbf{K}\cdot\mathbf{\hat{t}}\,dr}\;.
\end{equation}

La esfera tiene una velocidad $\mathbf{\omega}=\omega_0\mathbf{\hat{z}}$. El movimiento de esta esfera cargada genera una corriente en la misma direcci�n, cuya densidad de corriente es $\mathbf{K}=\sigma\,\mathbf{v}$, tal que $\mathbf{v}=\mathbf{\omega}\times\mathbf{r}$, en donde $\mathbf{r}$ es la forma param�trica de la esfera ($\mathbf{r}=a\mathbf{\hat{r}}$). Luego, 
\begin{equation}
	\mathbf{v}=\omega_0\mathbf{\hat{z}}\times a\,\mathbf{\hat{r}}=\omega_0\,a\,\sen\theta\,\mathbf{\hat{\varphi}}
\end{equation}

y la densidad de corriente queda:
\begin{equation}
	\mathbf{K}=\sigma\,\omega_0\,a\,\sen\theta\,\mathbf{\hat{\varphi}}
		=\sigma_0\cos\frac{\theta}{2}\,\,\omega_0\,a\,\sen\theta\,\mathbf{\hat{\varphi}}
\end{equation}

Volviendo a la ecuaci�n \eqref{ec_DefCor}: $\mathbf{\hat{t}}$ es el vector unitario perpendicular a la curva de integraci�n y tangente a la superficie que porta la corriente, $dr$ es el elemento diferencial de la curva de integraci�n y $\mathbf{K}$ es la densidad superficial de corriente. Para el caso de estudio $\mathbf{\hat{t}}=\mathbf{\hat{y}}$ y $dr=a\,d\theta$. Reemplazando estos valores en \eqref{ec_DefCor}, se obtiene que:
\begin{equation}\label{ec_SolCor1}
\mathbf{I}=\int\limits_{0}^{\pi}{\left(\sigma_0\cos\frac{\theta}{2}\,\,\omega_0\,a\,\sen\theta\,\mathbf{\hat{y}}\right)
	 \cdot \mathbf{\hat{y}}\,a\,d\theta}=\frac{4}{3}\,a^2\omega_0\,\sigma_0 \;.
\end{equation}

N�tese que $\varphi$ en la superficie de integraci�n es nulo, luego $\mathbf{\hat{\varphi}}=-\sen\varphi\,\mathbf{\hat{x}}+\cos\varphi\,\mathbf{\hat{y}}=\mathbf{\hat{y}}$.


\paragraph{Segunda ilustraci�n.}


Calcular la corriente que atraviesa la superficie $\varphi=\frac{\pi}{4}$ de un cono con densidad de carga $\sigma=\sigma_0\,\frac{r}{R}$ que gira sobre su eje con velocidad angular $\mathbf{\omega}=\omega_0\,\mathbf{\hat{z}}$, como se ilustra en la figura \ref{g_ConCar}. 

\begin{figure}[ht]
	\centering
		\includegraphics{E3g2.2}
	\caption{Cono cargado con movimiento rotacional.}
	\label{g_ConCar}
\end{figure}

La corriente buscada es:
\begin{equation}\label{ec_DefCor2}
		\mathbf{I}=\int\limits_{c}{\mathbf{K}\cdot\mathbf{\hat{t}}\,dr}\;.
\end{equation}

La densidad de corriente superficial es:
\begin{equation}\label{ec_DefK}
	\mathbf{K}=\sigma\,\mathbf{v}.
\end{equation}

La velocidad lineal de cualquier punto de la superficie es:
\begin{equation}\label{ec_DefV}
	 \mathbf{v}=\mathbf{\omega}\times\mathbf{r}=
	 	\left.\omega_0\,\mathbf{\hat{z}}\times r\,\mathbf{\hat{r}}\right|_{\theta=\theta_0}=
	 		\omega_0\,r\sen\theta_0\,\hat{\mathbf{\varphi}}\;.
\end{equation}

Reemplazando \eqref{ec_DefV} y el valor de $\sigma$ en \eqref{ec_DefK}, se obtiene que:
\begin{equation}\label{ec_DefK2}
	\mathbf{K}=\frac{\sigma_0\,\omega_0\,r^2}{R}\,\sen\theta_0\,\hat{\mathbf{\varphi}}\;.
\end{equation}

En \eqref{ec_DefCor2}, $\mathbf{\hat{t}}=\hat{\mathbf{\varphi}}$ y el elemento diferencial de l�nea es $dr$. Reemplazando \eqref{ec_DefK2} y lo anterior en \eqref{ec_DefCor2},
\begin{equation}\label{ec_SolCor2}
		\mathbf{I}=\int\limits_{0}^{R}{\frac{\sigma_0\,\omega_0}{R}\,\sen\theta_0\,r^2\,dr}=
		\frac{\sigma_0\,\omega_0\,R^2\,\sen\theta_0}{3}\;.
\end{equation}




\include{CalculoCamposCH}
\chapter{Interacci�n cl�sica de los campos con con la materia.}


\section{Interacci�n del campo el�ctrico con un diel�ctrico.}


Cuando un campo el�ctrico interact�a con la materia diel�ctrica, �sta responde con un nuevo campo el�ctrico que, en la mayor�a de los casos, va en la direcci�n contraria al campo inicial. Para entender c�mo es este proceso se estudiar� a continuaci�n el fen�meno de la polarizaci�n, que es el nombre con el que se asocian las respuestas el�ctricas de un aislante a la existencia de un campo el�ctrico externo con intensidad menor a la intensidad de ruptura. 

\subsection{Mecanismo f�sicos de la polarizaci�n.}\label{c_MecFispol}


En la interacci�n de un campo el�ctrico con un aislante ocurren varios procesos que se pueden resumir en tres: polarizaci�n electr�nica, polarizaci�n i�nica y polarizaci�n orientacional.


\paragraph{Polarizaci�n electr�nica.}


Suponga que un �tomo neutro diel�ctrico se sumerge en un campo el�ctrico con variaciones temporales y espaciales despreciables a escala at�mica. Al interactuar con el campo, el �tomo se deforma ya que se ejerce una fuerza sobre la nube de electrones que la desplaza, induciendo un momento dipolar por la aparici�n de una distancia no nula entre los centros efectivos de carga positiva (n�cleo) y negativa (nube de electrones), como se ilustra en la figura \ref{g_PolElec}.

\begin{figure}[ht]
	\centering
		\includegraphics{4g7.7}
	\caption{Separaci�n de los centros efectivos de carga. CEN: centro efectivo de carga negativa. CEP: centro efectivo de carga positiva.}
	\label{g_PolElec}
\end{figure}

El momento dipolar inducido es paralelo al campo el�ctrico polarizante. 


\paragraph{Polarizaci�n i�nica.}


Se presenta en mol�culas polares (mol�culas con momento dipolar intr�nseco no nulo) cuando interact�an con un campo el�ctrico de variaciones espacio-temporales despreciables a escala molecular. A toda mol�cula polar se le asocia una distancia del enlace ($d$) definida como
\begin{equation}
	d=\frac{p}{Q_+}\;,
\end{equation}

siendo $p$ la magnitud del momento dipolar y $Q_+$ la carga del ion positivo de la mol�cula. 

Cuando una mol�cula polar se encuentra ante la presencia de un campo el�ctrico con su momento dipolar intr�nseco paralelo a �l, se produce una fuerza sobre el ion positivo que va en direcci�n contraria a la fuerza ejercida sobre el ion negativo. Esto genera un aumento en la distancia del enlace y en consecuencia un aumento del momento dipolar que se puede interpretar como la aparici�n de un momento dipolar inducido que se suma al momento dipolar intr�nseco. 

\begin{figure}[ht]
	\centering
		\includegraphics{4g8.8}
	\caption{Mol�cula de �cido clorh�drico (HCl). El momento dipolar inducido es $p_{ind}=\Delta d\,Q_+$.}
	\label{g_MomDipInd}
\end{figure}



\paragraph{Polarizaci�n orientacional.}


Sup�ngase una distribuci�n de mol�culas polares interactuando con un campo el�ctrico externo. Sobre cada mol�cula aparecer� un torque debido a su momento dipolar intr�nseco igual a $\mathbf{\tau}=\mathbf{p}\times\mathbf{E}$, que intenta alinear el momento dipolar con el campo el�ctrico. El efecto macrosc�pico sobre un conglomerado de mol�culas polares termina siendo la aparici�n de un momento dipolar promedio no nulo que se puede interpretar como un momento dipolar inducido. 

\begin{figure}[ht]
	\centering
		\includegraphics{4g9.9}
	\caption{Conglomerado de mol�culas polares interactuando con un campo el�ctrico. a) Distribuci�n aleatoria con momento dipolar nulo. b) Orientaci�n preferencial de los dipolos en la direcci�n del campo polarizante.}
	\label{g_PolIon}
\end{figure}


\subsection{Primer modelo de la polarizaci�n de un diel�ctrico.}


Considerando los tres mecanismos f�sicos de la polarizaci�n descritos en \ref{c_MecFispol} se puede construir un modelo que prediga con muy buena aproximaci�n la interacci�n cl�sica de campo el�ctricos con los aislantes. N�tese que cada caso en \ref{c_MecFispol} lleva a la aparici�n de un momento dipolar inducido. Al suponer que la materia es una distribuci�n continua de dipolos, se pueden reproducir en un mismo modelo los efectos de los tres mecanismos ya vistos. 


\paragraph{Vector de polarizaci�n.}


Para la cuantificaci�n del fen�meno de la polarizaci�n, se considerar� un volumen $V'$ con una distribuci�n continua de dipolos, tal como se muestra en la figura \ref{g_DistrContDip}. El potencial producido por un diferencial de volumen en $\mathbf{r}$ es:
\begin{equation}\label{ec_DefPotDif}
	d\phi=\frac{d\mathbf{p}\cdot(\mathbf{r}-\mathbf{r'})}
				{4\pi\epsilon_0\,|\mathbf{r}-\mathbf{r'}|^3}\;.
\end{equation}

\begin{figure}[ht]
	\centering
		\includegraphics{4g10.10}
	\caption{Distribuci�n continua de dipolos.}
	\label{g_DistrContDip}
\end{figure}


Se define el \textbf{vector polarizaci�n} como el momento dipolar por unidad de volumen:
\begin{equation}\label{ec_DefVecPol}
	\mathbf{P}=\frac{d\mathbf{p}}{dV'}\;.
\end{equation}
 
Con la definici�n \eqref{ec_DefPotDif} y \eqref{ec_DefVecPol} se obtiene el potencial del volumen polarizado en $\mathbf{r}$:
\begin{equation}\label{ec_DefPotVol}
	\phi(\mathbf{r})=\frac{1}{4\pi\epsilon_0}\iiint\limits_{V'}
			{\frac{\mathbf{P}\cdot(\mathbf{r}-\mathbf{r'})}{|\mathbf{r}-\mathbf{r'}|^3}}\,dV'\;.
\end{equation}
 
Se acostumbra procesar la integral de \eqref{ec_DefPotVol} de forma tal que se expliciten las cargas superficiales y volum�tricas inducidas, as�:
\begin{equation}
	\nabla'\cdot\left( \frac{\mathbf{P}}{|\mathbf{r}-\mathbf{r'}|} \right)=
		\frac{\nabla'\cdot\mathbf{P}}{|\mathbf{r}-\mathbf{r'}|}+
			\mathbf{P}\cdot\nabla'\left( \frac{1}{|\mathbf{r}-\mathbf{r'}|} \right)\;,
\end{equation}

pero 
\begin{equation}\label{ec_Tall1}
	\nabla'\left( \frac{1}{|\mathbf{r}-\mathbf{r'}|} \right)
		=\frac{\mathbf{r}-\mathbf{r'}}{|\mathbf{r}-\mathbf{r'}|^3}\;.
\end{equation}

El procedimiento para calcular \eqref{ec_Tall1} se analiza en el taller de An�lisis vectorial.

En consecuencia,
\begin{equation}\label{ec_Num}
	\iiint\limits_{V'}{\frac{\mathbf{P\cdot(\mathbf{r}-\mathbf{r'})}}{|\mathbf{r}-\mathbf{r'}|^3}\,dV'}=
	\iiint\limits_{V'}{\nabla'\cdot\frac{\mathbf{P}}{|\mathbf{r}-\mathbf{r'}|}\,dV'}\,+\,
	\iiint\limits_{V'}{\frac{-\nabla'\cdot\mathbf{P}}{|\mathbf{r}-\mathbf{r'}|}\,dV'}\;.
\end{equation}

Se definen las siguientes cantidades:
\begin{itemize}
	\item $\rho_p\equiv-\nabla\cdot\mathbf{P}\equiv$ densidad volum�trica de carga de polarizaci�n. 
	
	\item $\sigma_p\equiv\mathbf{P}\cdot\mathbf{\hat{n}}\equiv$ densidad superficial de carga de polarizaci�n.
\end{itemize}

Con lo anterior y utilizando el teorema de la divergencia en \eqref{ec_Num} se obtiene:
\begin{equation}\label{ec_PotSupVolPol}
	\phi(\mathbf{r})=\frac{1}{4\pi\epsilon_0}\oiint\limits_{S'}
		{\frac{\sigma_p\,dS'}{|\mathbf{r}-\mathbf{r'}|}}\,+\,
	\frac{1}{4\pi\epsilon_0}\iiint\limits_{V'}
		{\frac{\rho_p\,dV'}{|\mathbf{r}-\mathbf{r'}|}}\;.
\end{equation}


\paragraph{Interpretaci�n de $\sigma_p$ y $\rho_p$.}


Consid�rese que una distribuci�n \textbf{homog�nea} de dipolos representa un material polarizado. Cuando ocurra la interacci�n con el campo el�ctrico, se puede considerar en una primera aproximaci�n que todos los dipolos se orientan en la direcci�n del campo el�ctrico total. En este escenario ocurre un fen�meno que aparece expl�cito en \eqref{ec_PotSupVolPol}. Una carga inducida se aloja en la superficie fruto de la orientaci�n de dipolos en la frontera (ver figura \ref{g_CarSupVolPol}). 

\begin{figure}[ht]
	\centering
		\includegraphics{4g11.11}
	\caption{a) Distribuci�n homog�nea de dipolos orientados. b) Distribuci�n superficial de carga de polarizaci�n.}
	\label{g_CarSupVolPol}
\end{figure}


T�ngase en cuenta que cada dipolo representa un arreglo de dos cargas puntuales: una positiva y otra negativa. Cuando ocurre la orientaci�n global y teniendo en cuenta que la distribuci�n es continua, hay un efecto de cancelaci�n entre dipolos pr�ximos (la carga positiva de un dipolo se yuxtapone con la carga negativa del dipolo inmediatamente superior) que no ocurre en la frontera, pues all� termina la distribuci�n. 

Haciendo uso del modelo se tiene que: 

\begin{itemize}
	\item La densidad volum�trica del i�simo dipolo es:

\begin{equation}
	\rho_i(\mathbf{r})=q\,\delta(\mathbf{r'}-\mathbf{r_i}-\mathbf{l})-
		q\,\delta(\mathbf{r'}-\mathbf{r_i})\;\medskip.
\end{equation} 

	\item La densidad volum�trica del volumen ser�:
	
\begin{equation}
	\rho(\mathbf{r'})=\sum\limits_{i=1}^{N}
	{q\,\left[\delta(\mathbf{r'}-\mathbf{r_i}-\mathbf{l})-
		\delta(\mathbf{r'}-\mathbf{r_i})\right]}\;\medskip.
\end{equation}

	\item El momento dipolar del volumen es:
	
\begin{equation}
\begin{aligned}
	\mathbf{p}=&\iiint\limits_{V'}{\rho(\mathbf{r'})\,\,\mathbf{r'}\,dV'}=\medskip \\
	&\iiint\limits_{V'}{\sum\limits_{i=1}^{N}
		{q\,\left[\delta(\mathbf{r'}-\mathbf{r_i}-\mathbf{l})-
		\delta(\mathbf{r'}-\mathbf{r_i})\right]}\,\mathbf{r'}\,dV'}=\medskip \\
	&\sum\limits_{i=1}^{N}
		{\left[q\,\delta(\mathbf{r_i}+\mathbf{l})-q\,(\mathbf{r_i})\right]}=\medskip\\
	&\sum\limits_{i=1}^{N}{q\,\mathbf{l}}=N\,\mathbf{p_0}\;.\medskip
\end{aligned}
\end{equation}
\end{itemize}

Siendo $\mathbf{p_0}=q\,\mathbf{l}$ el momento dipolar de un solo dipolo. 

Como se supone una distribuci�n homog�nea de dipolos orientados, la densidad volum�trica de dipolos se puede definir como el n�mero de dipolos por unidad de volumen:
\begin{equation}
	n=\frac{N}{V}\;.
\end{equation}

As�, $\mathbf{p}=n\,V\,\mathbf{p_o}$. Luego,
\begin{equation}
	\mathbf{P}=\frac{d\mathbf{p}}{dV}=\frac{d}{dV}\left[n\,V\,\mathbf{p_o}\right]=n\,\mathbf{p_0}
\end{equation}
es un vector constante. 

Luego $\rho_p=-\nabla\cdot\mathbf{P}=\textbf{0}$, reproduce cuantitativamente la cancelaci�n de la que se hablaba anteriormente. Cuando la distribuci�n de dipolos sea inhomog�nea, es posible que $\rho_p$ no sea nulo, de all� se suele afirmar que $\rho_p$ es una medida de la inhomogeneidad diel�ctrica del material.


\subsection{Vector desplazamiento el�ctrico.}


La aparici�n de $\rho_p$ suscita la pregunta natural de qu� relaci�n tiene con $\rho$, la densidad volum�trica de carga. La respuesta est� inspirada en la clasificaci�n que se puede hacer de la carga, a prop�sito de la aparici�n de la carga de polarizaci�n. 

Se dice que la carga el�ctrica se puede presentar como carga de polarizaci�n (carga inducida por el efecto cooperativo de los dipolos orientados en un diel�ctrico) y carga libre (carga que es susceptible de ser transportada, en el sentido de migrar de �tomo en �tomo, tal como lo hacen los electrones de conducci�n cuando se presenta una corriente directa). 

La anterior informaci�n se condensa en la ecuaci�n:
\begin{equation}\label{ec_DenCarLibPol}
	\rho=\rho_p+\rho_f\;,
\end{equation}

donde $\rho_p$ es la conocida densidad volum�trica de carga de polarizaci�n y $\rho_f$ se define como la densidad volum�trica de carga libre. 

Existe una forma alternativa para diferenciar la carga libre y la carga de polarizaci�n y tiene que ver con la capacidad de trasporte. La carga de polarizaci�n no se transporta, es decir, no hay una corriente material asociada a su aparici�n; mientras la carga libre s� se transporta, de hecho para llegar a donde est� colocada tuvo que existir una corriente material. 

Si en \eqref{ec_DenCarLibPol} se usa la ley de Gauss y la definici�n de la densidad de carga de polarizaci�n, se obtiene:
\begin{equation}\label{ec_PreDespElec}
\begin{gathered}
	\epsilon_0\,\nabla\cdot\mathbf{E}=-\nabla\cdot\mathbf{P}+\rho_f \\
	\nabla\cdot(\epsilon_0\,\mathbf{E}+\mathbf{P})=\rho_f\;.
\end{gathered}
\end{equation}

Se define el desplazamiento el�ctrico $\mathbf{D}$ como:
\begin{equation}\label{ec_DefDespElec}
	\mathbf{D}=\epsilon_0\,\mathbf{E}+\mathbf{P}\;,
\end{equation}

y la ecuaci�n \eqref{ec_PreDespElec} se transforma en:
\begin{equation}\label{ec_PostDespElec}
	\nabla\cdot\mathbf{D}=\rho_f\;.
\end{equation}

La aparici�n del vector $\mathbf{D}$ tiene su justificaci�n por la ecuaci�n \eqref{ec_PostDespElec} que recibe el nombre de la ley de Gauss para el Desplazamiento El�ctrico. \eqref{ec_PostDespElec} informa que una de las fuentes de $\mathbf{D}$ es la carga libre, que es m�s f�cil de manipular en el laboratorio que la carga de polarizaci�n.


\paragraph*{Ley constitutiva para los medios diel�ctricos.}


Una de las dificultades del modelo que se est� utilizando para la polarizaci�n, es que descarga la ``responsabilidad'' del comportamiento el�ctrico del medio en el vector $\mathbf{P}$. Para conocer c�mo es la polarizaci�n en un medio particular hay que apelar a la f�sica estad�stica, cuesti�n que se sale del prop�sito de estas notas de clase. Sin embargo, para la mayor�a de los materiales, existe una relaci�n lineal entre la polarizaci�n y el campo el�ctrico.
\begin{equation}\label{ec_RelPolE}
	\mathbf{P}=\mathbf{P}(\mathbf{E})=\epsilon_0\,\chi_e\,\mathbf{E}\;,
\end{equation}

donde $\chi_e$ es la susceptibilidad el�ctrica, que es una cantidad adimensional y mide la capacidad que tiene un medio de evitar ser penetrado por el campo polarizante: a mayor valor de $\chi_e$, menor campo el�ctrico total.

Utilizando \eqref{ec_RelPolE} en \eqref{ec_DefDespElec} se obtiene:
\begin{equation}
	\mathbf{D}=\epsilon_0\,\mathbf{E}+\epsilon_0\,\chi_e\,\mathbf{E}=
	\epsilon_0\,(1+\chi_e)\,\mathbf{E}=\epsilon_0\,\epsilon_r\,\mathbf{E}=\epsilon\,\mathbf{E}\;,
\end{equation}

de donde se establece por definici�n que:
\begin{itemize}
	\item $\epsilon_r \equiv 1+\chi_e$; permitividad relativa.
	\item $\epsilon \equiv \epsilon_0\,\epsilon_r$; permitividad. 
\end{itemize}

La ecuaci�n $\mathbf{D}=\epsilon\,\mathbf{E}$ se conoce como la ecuaci�n constitutiva para los medios diel�ctricos.


\section{Interacci�n del campo el�ctrico con un buen conductor.} 


Modelo de electrones libres para los metales: aproximaci�n cl�sica. 

Un primer modelo aproximado de c�mo se entiende un conductor desde la perspectiva de la conducci�n el�ctrica tiene los siguientes elementos: 

El conductor se considera como la yuxtaposici�n de una red cristalina y un gas de electrones. La red est� formada por los iones positivos fruto de considerar cada �tomo sin sus electrones de conducci�n y el gas de electrones est� constituido por todos los electrones de conducci�n (electrones libres) del metal, que pueden ``vagar'' por el conductor pero no abandonarlo. 

Se puede demostrar a partir de la mec�nica estad�stica que la rapid�s promedio de los electrones en un metal es del orden de $10^6\,m/s$. Cuando un campo el�ctrico externo es inyectado en un conductor, sobre los electrones libres aparece una fuerza que los mueve en direcci�n contraria al campo. La trayectoria de los electrones no es recta; en ausencia de campo el�ctrico los electrones se mueven de forma completamente aleatoria. 

Al existir un campo el�ctrico aparece una direcci�n de preferencia en sentido contrario al mismo campo aplicado. Se puede aproximar la interacci�n entre electrones libres e iones como un choque inel�stico. La distancia promedio recorrida por los electrones entre colisiones recibe el nombre de \emph{longitud libre media} $\lambda$, y el tiempo promedio entre colisiones se denomina \emph{tiempo libre medio} $\tau$. 

En realidad la interacci�n electr�n-ion no es una colisi�n; se parece m�s al movimiento sugerido en el gr�fico \ref{g_TraIon}. 

\begin{figure}[ht]
	\centering
		\includegraphics{4g12.12}
	\caption{Trayectorias de un electr�n al interactuar (l�nea continua) o no (l�nea punteada) con un ion.}
	\label{g_TraIon}
\end{figure}


Si el ion se considera un objeto puntual, se puede despreciar la trayectoria curva cerca de �l y aproximar el movimiento del electr�n como se muestra en el gr�fico \ref{g_TraNIon}

\begin{figure}[ht]
	\centering
		\includegraphics{4g13.13}
	\caption{Trayectoria con ion puntual.}
	\label{g_TraNIon}
\end{figure}


El movimiento de los electrones en la direcci�n contraria al campo, genera una corriente el�ctrica con una densidad asociada:
\begin{equation}\label{ec_DefDenAso}
	\mathbf{J}=\rho\,\mathbf{v_d}=n\,e\,\mathbf{v_d}\;,
\end{equation}

donde $n$ es el n�mero de electrones libres por unidad de volumen, $e$ es la carga del electr�n y $\mathbf{v_d}$ es la velocidad de deriva: velocidad promedio con que se transporta una hipot�tica carga positiva $-e$ movi�ndose en direcci�n del campo. 

El transporte de carga en la direcci�n del campo se puede entender como una carga que se mueve a ``saltos'', en tramos de longitud $\lambda$ demor�ndose $\tau$ segundos, partiendo del reposo y llegando a una velocidad terminal $\mathbf{v_d}$ por acci�n del campo:
\begin{equation}
	e\,\mathbf{E}=m\,a\,\Rightarrow\,a=\frac{e\,\mathbf{E}}{m}
	\quad,\quad \text{pero}\;\mathbf{v_d}=a\,\tau\;.
\end{equation}

Reemplazando lo anterior en \eqref{ec_DefDenAso} se obtiene:
\begin{equation}
	\mathbf{J}=n\,e\,a\,\tau
	=n\,e\,\tau\,\left(\frac{e\,\mathbf{E}}{m}\right)
	=\frac{n\,e^2\,\tau}{m}\,\mathbf{E}\;.
\end{equation}
Luego
\begin{equation}\label{ec_DefDenCor}
	\mathbf{J}=\sigma\,\mathbf{E}\quad,\quad\text{con}\quad\sigma=\frac{n\,e^2\,\tau}{m}\;.
\end{equation}

$\sigma$ en \eqref{ec_DefDenCor} se conoce como la conductividad y tiene unidades de $\mho/m$. Este primer modelo explica un comportamiento experimental del conductor ampliamente conocido: la relaci�n $V/I$ permanece constante en un amplio rango de valores. 
\begin{equation}
	V=R\,I\quad\text{con}\quad\frac{\partial R}{\partial V}=\frac{\partial R}{\partial I}=0\;.
\end{equation}

Entendi�ndose $V$ como la diferencia de potencial a la que se somete el dispositivo e $I$ la corriente que circula por los bornes. 

 
\section{Interacci�n del campo magn�tico con la materia.}


La \textbf{magnetizaci�n} es el fen�meno que ocurre en la materia cuando �sta interact�a con un campo magn�tico \textit{externo}. Al igual que en el caso de la polarizaci�n, la magnetizaci�n tiene varios mecanismos que sustentan el modelo utilizado para su explicaci�n, como se ver� en la secci�n \ref{sec_MecPol1}. 


\subsection{Modelo f�sico de la magnetizaci�n.}


Para predecir el comportamiento de la materia cuando interact�a con un campo magn�tico, se supone que ella es una distribuci�n continua de dipolos magn�ticos. Sup�ngase que el volumen $V'$ mostrado en la gr�fica \ref{g_RtaMagVolMag1} ha sido magnetizado por un campo que no aparece expl�cito en el dibujo. La respuesta magn�tica producida por el volumen en el lugar donde est� el observador, se puede calcular realizando una suma continua de los aportes de cada diferencial de momento dipolar sobre todo el volumen.
\begin{figure}[ht]
	\centering
		\includegraphics{4g16.16}
	\caption{Respuesta magn�tica de un volumen magnetizado.}
	\label{g_RtaMagVolMag1}
\end{figure}

Un diferencial de volumen magnetizado genera un potencial vectorial magn�tico en el observador:
\begin{equation}\label{ec_mag1}
	d\mathbf{A}(\mathbf{r})=\frac{\mu_0}{4\pi}\,
	\frac{d\mathbf{m}\times(\mathbf{r}-\mathbf{r'})}{|\mathbf{r}-\mathbf{r'}|^3}\;.
\end{equation}

Se define el vector Magnetizaci�n c�mo el momento dipolar por unidad de volumen, as�:
\begin{equation}\label{ec_mag2}
	d\mathbf{M}=\frac{d\mathbf{m}}{dV}\;.
\end{equation}

Al reemplazar \eqref{ec_mag2} en \eqref{ec_mag1} en integrar se obtiene:
\begin{equation}\label{ec_mag3}
	\mathbf{A}(\mathbf{r})=\frac{\mu_0}{4\pi}\,\iiint\limits_{V'}{
	\frac{\mathbf{M}(\mathbf{r'})\times(\mathbf{r}-\mathbf{r'})}{|\mathbf{r}-\mathbf{r'}|^3}\,dV'}\;.
\end{equation}

La integral en \eqref{ec_mag3} se divide en dos t�rminos, aprovechando la identidad vectorial en \eqref{ec_mag4} y utilizando el teorema de la divergencia de Gauss.
\begin{equation}\label{ec_mag4}
	\nabla'\times\left(\frac{\mathbf{M}(\mathbf{r'})}{|\mathbf{r}-\mathbf{r'}|}\right)=
	\frac{\nabla'\times\mathbf{M}(\mathbf{r'})}{|\mathbf{r}-\mathbf{r'}|}+
	\frac{\mathbf{M}(\mathbf{r'})\times(\mathbf{r}-\mathbf{r'})}{|\mathbf{r}-\mathbf{r'}|^3}\;.
\end{equation}

El resultado de esas operaciones matem�ticas queda plasmado en \eqref{ec_mag5}. Obs�rvese que los numeradores de los integrandos tienen unidades de densidad volum�trica y superficial de corriente, respectivamente. 
\begin{equation}\label{ec_mag5}
	\mathbf{A}(\mathbf{r})=\frac{\mu_0}{4\pi}\,\iiint\limits_{V'}{
	\frac{\nabla'\times\mathbf{M}(\mathbf{r'})}{|\mathbf{r}-\mathbf{r'}|}\,dV'}
	+
	\frac{\mu_0}{4\pi}\,\oiint\limits_{S'}{
	\frac{\mathbf{M}(\mathbf{r'})\times\mathbf{\hat{n'}}}{|\mathbf{r}-\mathbf{r'}|}\,dS'}\;.
\end{equation}

Por ello se definen las variables: 
\begin{equation}\label{ec_mag6}
\begin{gathered}
		\mathbf{J_m}(\mathbf{r}) \equiv \nabla\times\mathbf{M} \equiv \text{Densidad volum�trica de 					corriente de magnetizaci�n.} \medskip \\ 
		\mathbf{K_m}(\mathbf{r})=\mathbf{M}(\mathbf{r})\times\mathbf{n} \equiv \text{Densidad superficial de corriente de magnetizaci�n.}
\end{gathered}
\end{equation}


Teniendo en cuenta que
\begin{equation}\label{ec_mag7}
	\mathbf{B}=\nabla\times\mathbf{A}\;,
\end{equation}

el c�lculo del campo magn�tico de respuesta producido por el material magnetizado se realiza mediante:
\begin{equation}\label{ec_mag8}
	\mathbf{B}(\mathbf{r})=\frac{\mu_0}{4\pi}\,\iiint\limits_{V'}{
	\frac{\mathbf{J_m}(\mathbf{r'})\times(\mathbf{r}-\mathbf{r'})}
			{|\mathbf{r}-\mathbf{r'}|^3}\,dV'}
	+
	\frac{\mu_0}{4\pi}\,\oiint\limits_{S'}{
	\frac{\mathbf{K_m}(\mathbf{r'})\times(\mathbf{r}-\mathbf{r'})}
			{|\mathbf{r}-\mathbf{r'}|^3}\,dS'}\;.
\end{equation}

La aparici�n de una corriente de magnetizaci�n con su respectiva densidad volum�trica, suscita la pregunta sobre la relaci�n de esta con la densidad volum�trica $\mathbf{J}$ trabajada hasta ahora. La respuesta est� mediada por la clasificaci�n convencional que se realiza de las corrientes el�ctricas cl�sicas.

Las corrientes se pueden presentar en tres formas: corriente libre, que es la corriente debida al flujo de cargas libres. Se le asocia una densidad volum�trica de carga libre $\mathbf{J_f}$; corriente de magnetizaci�n, corriente que aparece por el efecto cooperativo de los dipolos magn�ticos. Se le asocia una densidad $\mathbf{J_m}$ y la corriente de desplazamiento, es aquella que aparece debida a la variaci�n temporal del desplazamiento el�ctrico.

Convencionalmente se asume que
\begin{equation}\label{ec_mag9}
	\mathbf{J}=\mathbf{J_m}+\mathbf{J_f}\;,
\end{equation}
 
dejando fuera de esta definici�n a la corriente de desplazamiento, dada su importancia en el fen�meno de la radiaci�n. Reemplazando en \eqref{ec_mag9} la definici�n de densidad de corriente de magnetizaci�n \eqref{ec_mag6} y la densidad de corriente total con ayuda de la Ley de Ampere
\begin{equation}\label{ec_mag10}
	\nabla\times\mathbf{B}=\mu_0\,\mathbf{J}\;,
\end{equation}

se obtiene:
\begin{equation}\label{ec_mag11}
	\frac{\nabla\times\mathbf{B}}{\mu_0}=\nabla\times\mathbf{M}+\mathbf{J_f}\quad\Rightarrow\quad
	\nabla\times\left(\frac{\mathbf{B}}{\mu_0}-\mathbf{M}\right)=\mathbf{J_f}\;.
\end{equation}

Se define el vector Intensidad de Campo Magn�tico:
\begin{equation}\label{ec_mag12}
	\mathbf{H}\equiv \frac{\mathbf{B}}{\mu_0}-\mathbf{M}\;.
\end{equation}

As�, \eqref{ec_mag11} termina siendo la Ley de Ampere para $\mathbf{H}$.

En la mayor�a de los materiales hay una relaci�n de directa proporcionalidad entre la Magnetizaci�n y la Intensidad de Campo Magn�tico. La constante de proporcionalidad mostrada en \eqref{ec_mag13} se denomina susceptibilidad magn�tica y es adimensional:
\begin{equation}\label{ec_mag13}
	\mathbf{M}=\mathbf{M}(\mathbf{H})=\chi_m\,\mathbf{H}\;.
\end{equation}

Utilizando \eqref{ec_mag13} en \eqref{ec_mag12} y con las definiciones
\begin{equation}\label{ec_mag14}
	\mu_r \equiv 1+\chi_m\qquad \mu \equiv \mu_0\,\mu_r\;,
\end{equation}

se obtiene la ecuaci�n constitutiva para los medios magn�ticos:
\begin{equation}\label{ec_mag15}
	\mathbf{B}=\mu\,\mathbf{H}\;.
\end{equation}
 

\subsection{Mecanismos f�sicos de la magnetizaci�n.} \label{sec_MecPol1}


\subsubsection{Momento magn�tico orbital de los electrones.}


Los electrones que no tienen completo alg�n nivel y subnivel tienen un momento magn�tico asociado a su momentum angular. La idea consiste en considerar esos electrones como peque�os circuitos de corriente, y el momento magn�tico como el momento dipolar del circuito. 

Debido a la mec�nica cu�ntica, los electrones que llenen \textit{por completo} el grupo de estados con $n$ y $l$ prefijados (siendo $n$ el n�mero cu�ntico principal, y $l$ el momentum angular) tienen un momento nulo, tanto orbital como de esp�n. 

\subsubsection{Momento magn�tico de esp�n de los electrones.}
Los electrones es s� mismos pueden considerarse como peque�os imanes, dotados de un momento dipolar magn�tico. En el caso que un estado est� semilleno, siempre se debe considerar el efecto magn�tico del esp�n. 

\subsubsection{Momento magn�tico total del n�cleo.}
Aunque con los nucleones se puede realizar la misma diferenciaci�n que se realiz� para los electrones (es decir, magnetismo orbital y magnetismo se esp�n), no ser� tomado en consideraci�n ac�, ya que para \textbf{casi todos} los prop�sitos pr�cticos el magnetismo nuclear se puede despreciar\footnote{Naturalmente en la Resonancia Nuclear Magn�tica no se desprecia.}.

Para intentar aclarar el asunto se dan a continuaci�n las relaciones de momento magn�tico para los nucleones:
\begin{equation}
	\frac{m_p}{m_e} \cong 1.3\times10^{-3} \qquad \frac{m_n}{m_e} \cong 1\times10^{-3}\;,
\end{equation}

donde $m_p$ y $m_n$ son los momentos magn�ticos para el prot�n y el neutr�n respectivamente. $m_e$ presenta el momento magn�tico del electr�n. 

\subsection{Interacci�n �tomo y campo magn�tico.} 
Los �tomos, seg�n su comportamiento ante un campo magn�tico externo, se pueden clasificar as�:
\begin{equation*}
	\text{�tomos: }\left\{ 
	\begin{aligned}
	&\text{Diamagn�ticos: �tomos con momento dipolar magn�tico total nulo.}\\
	&\text{Paramagn�ticos: �tomos con momento dipolar magn�tico total no nulo.}\\
	&\text{Ferromagn�ticos: �tomos con momento dipolar magn�tico total "grande``.}
	\end{aligned}
	\right.
\end{equation*}

\subsubsection{El �tomo diamagn�tico.} 
Una representaci�n del �tomo diamagn�tico m�s simple (el helio) es considerar dos electrones orbitando alrededor del n�cleo en direcciones contrarias. 

\begin{figure}[ht]
	\centering
		\includegraphics{4g1.1}
	\caption{�tomo diamagn�tico atravesado por un campo magn�tico entrante.}
	\label{4g1}
\end{figure}

Si el sistema interact�a con un campo magn�tico $\mathbf{B}$ entrando a la p�gina, como se ilustra en la figura \ref{4g1}, sobre cada electr�n aparecer� una fuerza que intenta sacarlo de su �rbita. Suponiendo que esta fuerza \textbf{no} es lo suficientemente intensa para cambiar el momentum angular de los electrones, la �nica forma en que la �rbita seguir� siendo estable es introduciendo una aceleraci�n centr�peta adicional. 

N�tese que sobre el electr�n que gira en direcci�n antihoraria aparece una fuerza magn�tica de la forma $\mathbf{F}=e\,\mathbf{v}\times\mathbf{B}$, y tambi�n una fuerza $\mathbf{F}=-e\,\mathbf{v}\times\mathbf{B}$ para el que gira en direcci�n horaria.

La aceleraci�n centr�peta adicional del primer electr�n ser�:
\begin{equation*}
	e\,v\,B=m_e\,a \,\Rightarrow\, a=\frac{e\,v\,B}{m_e}\;.
\end{equation*}

Esta aceleraci�n es la responsable de un cambio en la rapidez
\begin{equation*}
	a=\frac{v_a^2}{R}=\frac{e\,v\,B}{m_e}\,\Rightarrow\,v_a^2=\frac{e\,v\,B\,R}{m_e}\;.
\end{equation*}

Naturalmente la rapidez total del electr�n antihorario disminuye, por el efecto de $v_a$. El caso para el electr�n horario es an�logo, s�lo que en este caso la velocidad aumenta al intentar conservar la �rbita. 

Llamando $i_h$ la corriente asociada al electr�n horario e $i_a$ a la corriente del electr�n antihorario, la corriente total es 
\begin{equation*}
	i=i_h+i_a\;.
\end{equation*}

En ausencia de campo magn�tico se tiene que $i_h=-i_a\,\Rightarrow\, i=0$. Cuando $B\neq0$, $i_a$ disminuye e $i_h$ aumenta, con lo cual $i\neq0$, y finalmente aparece una corriente inducida que es responsable del momento magn�tico inducido. 

Es importante recalcar que la corriente inducida va en la direcci�n antihoraria, produciendo un momento magn�tico \textbf{que se opone al campo externo}.

�Qu� pasa si $\mathbf{B}$ no entra sino que sale de la p�gina? 

\begin{figure}[ht]
	\centering
		\includegraphics{4g2.2}
	\caption{�tomo diamagn�tico atravesado por un campo magn�tico que sale.}
	\label{4g2}
\end{figure}

La velocidad del electr�n antihorario ahora aumenta mientras que la velocidad del horario disminuye. Con la consabida aparici�n de una corriente total en sentido horario que es la responsable del momento inducido en oposici�n al campo externo. 

A resultados similares se llega si se utiliza la ley de inducci�n de Faraday. Para el caso del campo entrante, el flujo magn�tico pasa de ser nulo a un valor negativo, si se considera la curva l�mite de la superficie orientada en direcci�n antihoraria. 

\begin{figure}[ht]
	\centering
		\includegraphics{4g3.3}
	\caption{Campo magn�tico $\mathbf{B}$ entrante.}
	\label{4g3}
\end{figure}

De la figura \ref{4g3}, se aprecia que 
\begin{equation*}
\begin{gathered}
	\phi_0=\text{ Flujo magn�tico inicial}\\
	\phi_f=\text{ Flujo magn�tico final}\\
	\phi_0=0\;,\;\phi_f\cong-B\,A\;,\\
  E_{ind}\cong-\frac{\Delta\phi_B}{\Delta t}>0\;.
\end{gathered}
\end{equation*}

En consecuencia, aparece una corriente inducida en direcci�n antihoraria, como era de esperarse. 

En el caso contrario,

\begin{figure}[ht]
	\centering
		\includegraphics{4g4.4}
	\caption{Campo magn�tico $\mathbf{B}$ saliendo.}
	\label{4g4}
\end{figure}

De la figura \ref{4g4}, se aprecia que
\begin{equation*}
\begin{gathered}
	\phi_0=0\;,\;\phi_f\cong B\,A\;,\\
  E_{ind}\cong-\frac{\Delta\phi_B}{\Delta t}<0\;.
\end{gathered}
\end{equation*}

Aparece una corriente inducida en direcci�n horaria. 

En todos los casos descritos, el momento dipolar inducido va en una direcci�n tal que se opone al campo que lo produce. En una situaci�n completamente general se puede modelar el comportamiento diamagn�tico como la precesi�n del momento dipolar alrededor del campo magn�tico externo. Cuando un material diamagn�tico se magnetiza, el campo magn�tico total es menor que el campo magnetizante (campo externo) debido a que el material responde con un nuevo campo que se opone al campo externo.

\subsubsection{Paramagnetismo at�mico.} 

Como se dijo ya, cuando el estado con $n$ y $l$ prefijados no est� completamente lleno, el �tomo como un todo tiene un momento dipolar no nulo. 

En presencia de un campo magn�tico externo, sobre el �tomo aparece un torque que intenta alinear el momento magn�tico con el campo: $\mathbf{T}=\mathbf{m}\times\mathbf{B}$. Consid�rese una muestra de �tomos paramagn�ticos. En ausencia de un campo magn�tico externo, la orientaci�n de los �tomos es aleatoria produciendo un momento magn�tico promedio nulo. Cuando un campo interact�a con la muestra, sobre cada �tomo aparece un torque que intenta alinear $\mathbf{m}$ con $\mathbf{B}$. Es apenas obvio que no se logra el alineamiento perfecto si se considera la energ�a t�rmica asociada a cada �tomo, que se manifiesta por su vibraci�n mec�nica. 

\begin{figure}[ht]
	\centering
		\includegraphics{4g5.5}
	\caption{Alineaci�n de dipolos magn�ticos por un campo magn�tico externo.}
	\label{4g5}
\end{figure}

En la ecuaci�n constitutiva:
\begin{equation*}
	\mathbf{B}=\mu\,\mathbf{H}=\mu_0\,\mu_r\,\mathbf{H}=\mu_0\,(1+\chi_m)\,\mathbf{H}\;. 
\end{equation*}

Donde $\chi_m$ es la susceptibilidad magn�tica. 

Cuando el material es diamagn�tico, $\chi_m<0$; si el paramagn�tico, $\chi_m>0$. Normalmente la susceptibilidad magn�tica de un material paramagn�tico es mucho mayor que la de uno diamagn�tico. El efecto diamagn�tico produce cambios fraccionales en $\mu$ del orden de $10^{-8}$ a $10^{-5}$. 

La respuesta diamagn�tica es opacada en materiales que tienen momentos dipolares magn�ticos naturales. Tales materiales tienen una respuesta paramagn�tica, como se sabe, que proviene del momento dipolar de �tomos, mol�culas, defectos cristalinos y electrones de conducci�n. Los momentos dipolares, de nuevo, tienden a alinearse con el campo magn�tico pero no logran un alineamiento perfecto por la actividad t�rmica. 

Se puede demostrar que la magnitud de la magnetizaci�n puede ser expresada en t�rminos de la intensidad de campo magn�tico que act�a sobre los dipolos elementales y la temperatura absoluta:
\begin{equation*}
	M=N\,m_0\,\left[\coth{\left(\frac{m_0\,\mu_0\,H_i}{K_B\,T}\right)}-\frac{K_B\,T}{m_0\,\mu_0\,H_i}\right]\;,
\end{equation*}

donde
\begin{itemize}
	\item $N$: n�mero de dipolos por unidad de volumen.
	\item $m_0$: momento dipolar de un dipolo elemental. 
	\item $K_B$: constante de Boltzman. 
\end{itemize}

En estos materiales $\mathbf{M}$ es paralelo a $\mathbf{H_i}$ y el campo molecular $\mathbf{H_i}$ es pr�cticamente igual al campo macrosc�pico $\mathbf{H}$. Luego, para $\mathbf{H}$ no muy intenso y temperaturas no muy bajas,
\begin{equation*}
	\mathbf{M}=\chi_m\,\mathbf{H}\;.
\end{equation*}

As�,
\begin{equation}
	\chi_m\cong\frac{N\,\mu_0\,m_0^2}{3\,K_B\,T}\quad;\quad\chi_m\approx10^{-5}\;.
\end{equation}


\subsubsection{Medios isotr�picos no lineales.}


Si la temperatura de una substancia paramagn�tica es reducida por debajo de cierto valor que depende de cada material, la magnetizaci�n $\mathbf{M}$ puede ser suficiente para producir el campo necesario para mantener los dipolos alineados, aunque un campo externo no est� presente. El campo molecular $\mathbf{H_i}$ producido por la magnetizaci�n es:
\begin{equation*}
	\mathbf{H_i}=K\,\mathbf{M}\;.
\end{equation*}

De la ecuaci�n para la magnetizaci�n se obtiene la condici�n para la magnetizaci�n espont�nea, despejando $M$:
\begin{equation*}
	M=N\,m_0\,\left[\coth{\left(\frac{m_0\,\mu_0\,K\,M}{K_B\,T}\right)}-\frac{K_B\,T}{m_0\,\mu_0\,K\,M}\right]\;.
\end{equation*}

La temperatura debajo de la cual tales materiales exhiben una magnetizaci�n espont�nea se llama \textit{temperatura de Curie}. Se ha encontrado experimentalmente que $K$, el factor de medida de la interacci�n de los dipolos vecinos, es del orden de $10^{3}$. Si la interacci�n fuese causada �nicamente por interacciones magn�ticas, $K$ ser�a del orden de $\frac{1}{3}$. Otra vez el comod�n de la ignorancia; el valor observado de $K$ se explica desde la mec�nica cu�ntica, ya que hay una interacci�n el�ctrica entre mol�culas, debido a la distorsi�n de las distribuciones de carga en ellas. 

En algunos materiales es energ�ticamente probable que los dipolos magn�ticos se orienten paralelamente a sus vecinos. En estos casos se denomina el comportamiento como \textit{ferromagn�tico}. En otras substancias es favorable energ�ticamente que los dipolos asuman una orientaci�n antiparalela. Este caso se llama \textit{antiferromagnetismo}. En los materiales \textit{ferrimagn�ticos} (ferritas) los dipolos vecinos se alinean antiparalelamente, pero hay diferentes tipos de �tomos y los dipolos no se cancelan. 

Es tambi�n energ�ticamente favorable que un material de estos (ferro, antiferro o ferri) se subdivida en regiones llamadas dominios magn�ticos, donde todos sus dipolos son paralelos o antiparalelos. Esta subdivisi�n minimiza la energ�a total del cristal. Una cantidad macrosc�pica de materiales contiene un gran n�mero de cristalitas (dominios) con orientaci�n aleatoria, dando un efecto macrosc�pico de isotrop�a. Cuando un campo magn�tico es aplicado aparece un torque sobre cada dominio que intenta alinearlo con el campo externo. Este proceso es no lineal y en consecuencia $\mathbf{B}$ no es proporcional a $\mathbf{H}$. 

\begin{figure}[ht]
	\centering
		\includegraphics{4g6.6}
	\caption{Alineaci�n de dipolos magn�ticos por un campo magn�tico externo.}
	\label{4g6}
\end{figure}

De la figura \ref{4g6} se aprecia que:
\begin{itemize}
	\item $B_{sat}$: inducci�n magn�tica de saturaci�n (aunque aumente la intensidad de campo magn�tico, la inducci�n magn�tica de la muestra no supera $B_{sat}$). 
	\item $B_r$: inducci�n magn�tica de remanencia (despu�s de una magnetizaci�n es el campo remanente luego de anular $H$). 
	\item $H_c$: intensidad de campo coercitiva (intensidad de campo magn�tico que debe ser colocado para anular la magnetizaci�n de la muestra).
\end{itemize}

Algunas aplicaciones de los materiales ferromagn�ticos son:
\begin{itemize}
	\item Imanes permanentes. 
	\item Grabado de se�ales.
	\item Apantallamiento magn�tico.
	\item N�cleos para electroimanes, transformadores, motores el�ctricos y generadores.
\end{itemize}

Los materiales ferromagn�ticos pueden ser:
\begin{itemize}
	\item \textbf{Duros}: cuando el campo aplicado es removido, el material queda con una fuerte magnetizaci�n.
	\item \textbf{Suaves}: en una circunstancia similar al material duro, el suave queda con una d�bil magnetizaci�n. 
\end{itemize}

Algunas valores t�picos para los materiales duros son:
\begin{equation*}
\begin{aligned}
	&\text{Acero} & &B_r\approx0.6\,T & &H_c\approx0.0765\,A/m \\
	&\text{Acero cobalto} & &B_r\approx1.08\,T & &H_c\approx0.0980\,A/m
\end{aligned}
\end{equation*}

En los materiales suaves es normal que se presente una inducci�n de saturaci�n grande, peque�a fuerza coercitiva, alta permeabilidad inicial, alta permeabilidad m�xima y peque�as p�rdidas por hist�resis. Un material que se pretende usar como escudo magn�tico debe tener una permeabilidad inicial alta como criterio determinante. 

Algunos valores t�picos para los materiales duros son:
\begin{center}
\begin{tabular}{|c|c|c|c|c|}\hline
	 & Material    & Composici�n    & \multicolumn{1}{m{2.3cm}|}{Permeabilidad\,\,\,\, inicial} & 				
	 						\multicolumn{1}{m{2.3cm}|}{Permeabilidad\,\,\,\, m�xima} \\\hline\hline
	1 & Acero          & 99 Fe          & $2\times10^{2}$       & $6\times10^{3}$ \\ \hline
	2 & Acero puro     & 99.9 Fe        & $2.5\times10^{4}$     & $3.5\times10^{5}$\\ \hline
	3 & 78 permalloy   & 78 Ni, 22 Fe   & $4\times10^{3}$       & $1.0\times10^{5}$\\ \hline
	4 & Superpermalloy & \multicolumn{1}{m{2cm}|}{15 Fe, 79 Ni, 5\,Mo, 0.5\,Mn}  & $9\times10^{4}$       & $1\times10^{6}$ \\ \hline
\end{tabular}
\end{center}

\begin{center}
\begin{tabular}{|c|c|c|}\hline
	  & Fuerza coercitiva & Inducci�n de saturaci�n\\ \hline\hline
	1 & 0.9 & 2.16 \\ \hline
	2 & 0.01 & 2.16 \\ \hline
	3 & 0.05 & 1.05 \\ \hline
	4 & 0.004 & 0.8 \\\hline
\end{tabular}
\end{center}

En metales ferromagn�ticos usados cuando el campo magn�tico cambia en el tiempo (transformadores, generadores), se inducen corrientes de Eddy (corrientes par�sitas). 

Si el material tiene alta conductividad, las corrientes son grandes y las p�rdidas por efecto Joule pueden ser inaceptables. Para reducir este problema se lamina los n�cleos con la direcci�n de las corrientes inducidas perpendicular a la laminaci�n. Sin embargo, estas p�rdidas llegan a ser intolerables para frecuencias de radio y microondas. Es all� donde son importantes las \textbf{ferritas} por su baja conductividad (mucho menor a $10^{-4}\,\frac{\mho}{m}$).

Las ferritas se usan en n�cleos de transformadores de alta frecuencia, filtros, etc. 
\chapter{Condiciones de contorno en las interfases.}


En la realidad se presentan cambios suaves en el promedio de alguna propiedad local en una regi�n determinada. En la idealizaci�n de un mundo continuo (f�sica para los medios continuos) se consideran cambios abruptos de las propiedades, como se ilustra en la figura \ref{g_CamMed}; es decir, $\Delta h$ es nulo. 
\begin{figure}[ht]
	\centering
		\includegraphics{6g1.1}
	\caption{Cambio de medio real e ideal de una propiedad local. En $x=x_0$ est� la interfase.}
	\label{g_CamMed}
\end{figure}

Adem�s:
\begin{itemize}
	\item Las ecuaciones diferenciales no son v�lidas en las interfases entre medios diferentes. Es necesario obtener y aplicar condiciones de contorno que permitan el salto de un medio a otro. 
	\item El procedimiento general consiste en suponer que la transici�n entre medios se produce de forma suave en un intervalo $\Delta h$, se aplica la ecuaci�n integral correspondiente y despu�s se hace tender $\Delta h$ a cero. 
\end{itemize}

Conociendo la divergencia de un campo se puede conocer el comportamiento de las componentes normales. Conociendo el rotacional de un campo se puede conocer el comportamiento de las componentes tangenciales. 


\section{Componentes normales de $\mathbf{A}$.}


Por convenci�n se elige la normal $\mathbf{\hat{n}}$ del medio 1 al medio 2, como se ilustra en \ref{g_DirVecNor}.
\begin{figure}[ht]
	\centering
		\includegraphics{6g2.2}
	\caption{El vector normal: del medio 1 al medio 2.}
	\label{g_DirVecNor}
\end{figure}

\begin{figure}[ht]
	\centering
		\includegraphics{6g3.3}
	\caption{Volumen $V$ encerrado por las superficies de control.}
	\label{g_VolCamMed}
\end{figure}

Se conoce la divergencia de $\mathbf{A}$;
\begin{equation}
	\nabla\cdot\mathbf{A}=d(\mathbf{r})\;.
\end{equation}

Sea $V$ el volumen encerrado por $S_1+S_2+S_{LAT}$, como se ilustra en la figura \ref{g_VolCamMed}, entonces:
\begin{equation}
	\iiint\limits_{V}{\nabla\cdot\mathbf{A}\,dV}=\iiint\limits_{V}{d(\mathbf{r})\,dV}\;.
\end{equation}

Por el teorema de la divergencia, la primera integral es:
\begin{equation}
	\iiint\limits_{V}{\nabla\cdot\mathbf{A}\,dV}=\oiint\limits_{S}{\mathbf{A}\cdot d\mathbf{S}}=
	\iint\limits_{S_1}{\mathbf{A}\cdot d\mathbf{S}}+
	\iint\limits_{S_2}{\mathbf{A}\cdot d\mathbf{S}}+
	\iint\limits_{S_{LAT}}{\mathbf{A}\cdot d\mathbf{S}}\;.
\end{equation}

Cuando $\Delta h \to 0$ ocurre que $S_1\to S$, $S_2\to S$, $S_{LAT}\to S$, $\mathbf{\hat{n}_1}\to-\mathbf{\hat{n}}$, $\mathbf{\hat{n}_2}\to\mathbf{\hat{n}}$. Luego:
\begin{equation}
\begin{gathered}
	\iint\limits_{S_1}{\mathbf{A}\cdot d\mathbf{S}}\to-
			\iint\limits_{S}{\mathbf{A_1}\cdot \mathbf{\hat{n}}\,dS}\medskip  \\
	\iint\limits_{S_2}{\mathbf{A}\cdot d\mathbf{S}}\to 
			\iint\limits_{S}{\mathbf{A_2}\cdot \mathbf{\hat{n}}\,dS}\medskip  \\		
	\iint\limits_{S_{LAT}}{\mathbf{A}\cdot d\mathbf{S}}\to 0\;.	
\end{gathered}
\end{equation}

$\mathbf{A_1}$ y $\mathbf{A_2}$ son lo valores del campo en los medios 1 y 2, respectivamente. 
\begin{equation}
	\boxed{
	\oiint\limits_{S}{\mathbf{A}\cdot d\mathbf{S}} =
	\iint\limits_{S}{\mathbf{\hat{n}}\cdot \left(\mathbf{A_2}-\mathbf{A_1}\right)\, d\mathbf{S}} =
	\lim\limits_{\Delta h\to 0}{\iiint\limits_{V}{\left(\nabla\cdot\mathbf{A}\right)\,dV}}
	}
\end{equation}


\subsection{Para el campo el�ctrico.}


\begin{equation}
	\iint\limits_{S}{\mathbf{\hat{n}}\cdot\left(\mathbf{E_2}-\mathbf{E_1}\right)\,dS}=
	\lim\limits_{\Delta h \to 0}{\iiint\limits_{V}{\left(\nabla\cdot\mathbf{E}\right)\,dV}}=
	\lim\limits_{\Delta h \to 0}{\iiint\limits_{V}{\frac{\rho}{\epsilon_0}\,dV}}\;.
\end{equation}

$\rho$ es la densidad de carga y $\iiint\limits_{V}{\rho\,dV}$ es la carga alojada en el volumen $V$. A medida que $\Delta h$ tiende a cero, el volumen $V$ se asemeja cada vez m�s a la superficie $S$. Se supone entonces que toda la carga del volumen $V$ est� alojada sobre la superficie $S$ ($\rho=\delta(z)\,\sigma$), suponiendo el eje $z$ como eje del cilindro. As�,
\begin{equation}
\begin{gathered}
	\iint\limits_{S}{\mathbf{\hat{n}}\cdot\left(\mathbf{E_2}-\mathbf{E_1}\right)\,dS}=
		\lim\limits_{\Delta h \to 0}{\iiint\limits_{V}{\frac{\delta(z)\,\sigma}{\epsilon_0}\,dV}}=
		\lim\limits_{\Delta h \to 0}{\iint\limits_{S}{\frac{\sigma}{\epsilon_0}\,dS}}\medskip \\
	\iint\limits_{S}{\mathbf{\hat{n}}\cdot\left(\mathbf{E_2}-\mathbf{E_1}\right)\,dS}=	
		\iint\limits_{S}{\frac{\sigma}{\epsilon_0}\,dS}\medskip \\
\end{gathered}
\end{equation}

Como la superficie $S$ es totalmente arbitraria en la interfase, entonces:
\begin{equation}
	\boxed{
	\mathbf{\hat{n}}\cdot\left(\mathbf{E_2}-\mathbf{E_1}\right)=\frac{\sigma}{\epsilon_0}
	\quad \text{�}	\quad
	E_{2n}-E_{1n}=\frac{\sigma}{\epsilon_0}\;.
	}
\end{equation}


\subsection{Para el desplazamiento el�ctrico.}


\begin{equation}
	\iint\limits_{S}{\mathbf{\hat{n}}\cdot\left(\mathbf{D_2}-\mathbf{D_1}\right)\,dS}=
	\lim\limits_{\Delta h \to 0}{\iiint\limits_{V}{\left(\nabla\cdot\mathbf{D}\right)\,dV}}=
	\lim\limits_{\Delta h \to 0}{\iiint\limits_{V}{\rho_f\,dV}}\;.
\end{equation}

$\rho_f$ es la densidad de carga libre y $\iiint\limits_{V}{\rho_f\,dV}$ es la carga libre alojada en el volumen $V$. Con $\rho_f=\delta(z)\,\sigma_f$ se obtiene:
\begin{equation}	
\begin{gathered}
	\iint\limits_{S}{\mathbf{\hat{z}}\cdot\left(\mathbf{D_2}-\mathbf{D_1}\right)\,dS}=
		\iint\limits_{S}{\sigma_f\,dS}\medskip \\
	\boxed{\mathbf{\hat{n}}\cdot\left(\mathbf{D_2}-\mathbf{D_1}\right)=\sigma_f 
	\qquad\text{�}\qquad
	D_{2n}-D_{1n}=\sigma_f}
\end{gathered}
\end{equation}


\section{Componentes tangenciales de $\mathbf{A}$.}


\begin{figure}[ht]
	\centering
		\includegraphics{6g4.4}
	\caption{Componentes tangenciales.}
	\label{g_ComTan}
\end{figure}

Ahora se conoce el rotacional de $\mathbf{A}$:
\begin{equation}
	\nabla\times\mathbf{A}=\mathbf{c}\;.
\end{equation}

Seg�n la figura \ref{g_ComTan}, siendo $S$ la superficie limitada por $L_1$, $L_2$ y $L_{LAT}$, y siendo  $C=L_1+L_2+L_{LAT}$:
\begin{equation}
\begin{gathered}
	\iint\limits_{S}{\left(\nabla\times\mathbf{A}\right)\cdot d\mathbf{S}}=
		\oint\limits_{C}{\mathbf{A}\cdot d\mathbf{r}}\bigskip \\ 
	\oint\limits_{C}{\mathbf{A}\cdot d\mathbf{r}}=
		\int\limits_{L_1}{\mathbf{A}\cdot d\mathbf{r}}+
		\int\limits_{L_2}{\mathbf{A}\cdot d\mathbf{r}}+
		\int\limits_{L_{LAT}}{\mathbf{A}\cdot d\mathbf{r}}\;.
\end{gathered}
\end{equation}

Si $\Delta h \to 0$, se tiene que $L_2 \to L$, $L_1 \to -L$, $L_{LAT} \to 0$:
\begin{equation}
\begin{gathered}
		\int\limits_{L_1}{\mathbf{A}\cdot d\mathbf{r}}=
			-\int\limits_{L}{\mathbf{A_1}\cdot \mathbf{\hat{t}}\,dr}\bigskip \\ 
		\int\limits_{L_2}{\mathbf{A}\cdot d\mathbf{{r}}}=
			\int\limits_{L}{\mathbf{A_2}\cdot \mathbf{\hat{t}}\,dr}\bigskip \\
		\int\limits_{L_{LAT}}{\mathbf{A}\cdot d\mathbf{r}}=0
\end{gathered}
\end{equation}

Entonces:
\begin{equation}
\begin{gathered}
	\oint\limits_{C}{\mathbf{A}\cdot\mathbf{r}} \to 
		\int\limits_{L}{(\mathbf{A_2}-\mathbf{A_1})\cdot\mathbf{\hat{t}}\,dr} \to
		\iint\limits_{S}{(\nabla\times\mathbf{A})\cdot d\mathbf{S}} \medskip \\ 
	\boxed{\int\limits_{L}{(\mathbf{A_2}-\mathbf{A_1})\cdot\mathbf{\hat{t}}\,dt} =
			\lim_{\Delta h\to0}\iint\limits_{S}{(\nabla\times\mathbf{A})\cdot d\mathbf{S}}}
\end{gathered}
\end{equation}


\subsection{Para el campo el�ctrico.}


\begin{equation}
\begin{gathered}
	\int\limits_{L}{(\mathbf{E_2}-\mathbf{E_1})\cdot\mathbf{\hat{t}}\,dr} =
			\lim_{\Delta h\to0}\iint\limits_{S}{(\nabla\times\mathbf{E})\cdot d\mathbf{S}}=0\;.
\end{gathered}
\end{equation}

Como el resultado es para cualquier $L$ sobre $S$, entonces:
\begin{equation}
	(\mathbf{E_2}-\mathbf{E_1})\cdot\mathbf{\hat{t}}=0\;,
\end{equation}

donde $\mathbf{\hat{t}}$ es un vector tangente a $L$ y a su vez a $S$. As�, 
\begin{equation}
	E_{2l}-E_{1l}=0\quad\Rightarrow\quad \boxed{E_{2l}=E_{1l}}
\end{equation}

Lo anterior implica que las componentes tangenciales del campo el�ctrico son continuas al cambia de medio. 


\subsection{Para el desplazamiento el�ctrico.}


\begin{equation}
\begin{gathered}
	\int\limits_{L}{(\mathbf{D_2}-\mathbf{D_1})\cdot\mathbf{\hat{t}}\,dt} =
			\lim_{\Delta h\to0}\iint\limits_{S}{(\nabla\times\mathbf{D})\cdot d\mathbf{S}}=
			\lim_{\Delta h\to0}\iint\limits_{S}{(\nabla\times\mathbf{P})\cdot d\mathbf{S}}=
\end{gathered}
\end{equation}

Si:
\begin{equation}
	\int\limits_{L}{(\mathbf{D_2}-\mathbf{D_1})\cdot\mathbf{\hat{t}}\,dt} =
		\int\limits_{L}{(\mathbf{P_2}-\mathbf{P_1})\cdot\mathbf{\hat{t}},dt}\;,
\end{equation}

entonces:
\begin{equation}
	\boxed{D_{2l}-D_{1l}=P_{2l}-P_{1l}}
\end{equation}


\section{Resumen.}


A continuaci�n se presenta una tabla de las ecuaciones de componentes recurrentes para el c�lculo de las variables electromagn�ticas. A partir de las ecuaciones diferenciales, aplicando un procedimiento similar al visto anteriormente, puede llegarse f�cilmente a las ecuaciones de componentes. 

\begin{table}[ht]
	%\renewcommand{\arraystretch}{2.2}
	\centering
		\begin{tabular}{|c|c|}\hline
			Ecuaci�n diferencial & Ecuaci�n de componentes\\ \hline
			$\nabla\cdot\mathbf{E}=\dfrac{\rho}{\epsilon_0}$ & 
					$\mathbf{\hat{n}}\cdot(\mathbf{E_2}-\mathbf{E_1})=\dfrac{\sigma}{\epsilon_0}$\\ \hline
			$\nabla\times\mathbf{B}
							=\mu_0\,\mathbf{J}+\mu_0\epsilon_0\dfrac{\partial\mathbf{E}}{\partial t}$ & 
					$\mathbf{\hat{n}}\times(\mathbf{B_2}-\mathbf{B_1})=\mu_0\,\mathbf{K}$\\ \hline
			$\nabla\times\mathbf{E}=-\,\dfrac{\partial\mathbf{B}}{\partial t}$ & 
					$\mathbf{\hat{n}}\times(\mathbf{E_2}-\mathbf{E_1})=\textbf{0}$ \\ \hline
			$\nabla\cdot\mathbf{B}=0$ &
					$\mathbf{\hat{n}}\cdot(\mathbf{B_2}-\mathbf{B_1})=\textbf{0}$\\ \hline
			$\nabla\cdot\mathbf{D}=\rho_f$ &
					$\mathbf{\hat{n}}\cdot(\mathbf{D_2}-\mathbf{D_1})=\sigma_f$\\ \hline
			$\nabla\times\mathbf{H}=\mathbf{J_f}+\dfrac{\partial\mathbf{D}}{\partial t}$ & 
					$\mathbf{\hat{n}}\times(\mathbf{H_2}-\mathbf{H_1})=\mathbf{K_f}$\\ \hline
			$\nabla\times\mathbf{D}=-\,\epsilon_0\dfrac{\partial\mathbf{B}}{\partial t}
							+\nabla\times\mathbf{P}$ & 
					$\mathbf{\hat{n}}\times(\mathbf{D_2}-\mathbf{D_1})=
							\mathbf{\hat{n}}\times(\mathbf{P_2}-\mathbf{P_1})$\\ \hline
			$\nabla\cdot\mathbf{H}=-\nabla\cdot\mathbf{M}$ & 
					$\mathbf{\hat{n}}\cdot(\mathbf{H_2}-\mathbf{H_1})=
							-\mathbf{\hat{n}}\cdot(\mathbf{M_2}-\mathbf{M_1})$\\ \hline
			$\nabla\cdot\mathbf{J}+\dfrac{\partial\rho}{\partial t}=0$ & 
					$\mathbf{\hat{n}}\cdot\left
							[\mathbf{J_2}+\epsilon_0\dfrac{\partial\mathbf{E_2}}{\partial t} 
							-\mathbf{J_1}-\epsilon_0\dfrac{\partial\mathbf{E_1}}{\partial t}\right]=0$\\ \hline
		\end{tabular}
	\caption{Ecuaciones de componentes seg�n ecuaciones diferenciales.}
	\label{tab_ec_comp1}
\end{table}

\chapter{Energ�a el�ctrica y magn�tica}

\section{Energ�a electrost�tica.}


El trabajo necesario para conformar una distribuci�n de carga se almacena como energ�a potencial en el campo el�ctrico, porque si se liberan las cargas, este trabajo se podr�a recuperar.


\subsection{Trabajo necesario para conformar una distribuci�n de cargas puntuales.}


Sup�ngase que se construye un sistema de $n$ cargas puntuales, trayendo cada carga desde el infinito hasta su posici�n en la distribuci�n, como se ilustra en la figura \ref{g_DisCar}.

\begin{figure}[hb]
	\centering
		\includegraphics{7g1.1}
	\caption{Distribuci�n de cargas en un universo de permitividad $\epsilon$.}
	\label{g_DisCar}
\end{figure}

Sea $W_i$ el trabajo necesario para traer la carga $q_i$. $W_1=0$ y $W_i=q_i\,\phi_i$ :
\begin{equation}
	W_i=q_i\,\sum\limits_{j=1}^{i-1}{\frac{q_j}{4\pi\epsilon r_{ij}}}
	\quad\text{con}\quad
	r_{ij}=|\mathbf{r_i}-\mathbf{r_j}|\;.
\end{equation}

$\phi_i$ es el potencial en $i$ debido a las $i-1$ cargas existentes. El trabajo total para formar la distribuci�n ser�:
\begin{equation}
	W=\sum\limits_{i=1}^{n}{W_i}=
	\sum\limits_{i=1}^{n}{q_i\,\sum\limits_{j=1}^{i-1}{\frac{q_j}{4\pi\epsilon r_{ij}}}}\;,
\end{equation}

que se puede escribir as�:
\begin{equation}
	\sum\limits_{i=1}^{n}{\sum\limits_{j=1}^{i-1}{\frac{q_i\,q_j}{4\pi\epsilon r_{ij}}}}=
	\sum\limits_{j=1}^{n}{\sum\limits_{i=1}^{j-1}{\frac{q_i\,q_j}{4\pi\epsilon r_{ij}}}}\;.
\end{equation}

Luego:
\begin{equation}\label{ec_Trab1}
	2W=	\sum\limits_{i=1}^{n}{\sum\limits_{j=1}^{i-1}{\frac{q_i\,q_j}{4\pi\epsilon r_{ij}}}}+
	\sum\limits_{j=1}^{n}{\sum\limits_{i=1}^{j-1}{\frac{q_i\,q_j}{4\pi\epsilon r_{ij}}}}\;.
\end{equation}

De la ecuaci�n \eqref{ec_Trab1}, la primera suma representa la suma de los t�rminos de una matriz triangular inferior con $i$ creciendo horizontalmente y $j$ creciendo verticalmente. Cada t�rmino ser�a $\frac{q_i\,q_j}{4\pi\epsilon r_{ij}}$, la segunda es la suma de los t�rminos de la matriz triangular superior que completa la matriz cuadrada para el arreglo.

Luego:
\begin{equation}
		W=\frac{1}{2}\sum\limits_{\substack{i=1\\i\neq j}}^{n}\sum\limits_{\substack{j=1\\j\neq i}}^{n}
		\frac{q_i\,q_j}{4\pi\epsilon r_{ij}}\;.
\end{equation}

El salto al mundo continuo es inmediato.
\begin{equation}
	W=\frac{1}{2}\iiint\limits_{V}{\iiint\limits_{V'}
		{\frac{dq\,dq'}{4\pi\epsilon|\mathbf{r}-\mathbf{r'}|}}}=
		\frac{1}{2}\iiint\limits_{V}{\iiint\limits_{V'}
		{\frac{\rho\,dV\,\rho'\,dV'}{4\pi\epsilon|\mathbf{r}-\mathbf{r'}|}}}\;.
\end{equation}

Tomando la definici�n del potencial electrost�tico, se puede escribir as�:
\begin{equation}
\begin{gathered}
		W=\frac{1}{2}\iiint\limits_{V}{\rho\left[\iiint\limits_{V'}
			{\frac{\rho'\,dV'}{4\pi\epsilon|\mathbf{r}-\mathbf{r'}|}}\right]\,dV} \medskip \\
		W=\frac{1}{2}\iiint\limits_{V}{\rho\,\phi\,dV}\;,
\end{gathered}
\end{equation}

siendo $\phi$ el potencial de la distribuci�n. Como por definici�n la energ�a electrost�tica es el trabajo realizado necesario para conformar la distribuci�n de carga, entonces:
\begin{equation}
	\boxed{U=\frac{1}{2}\iiint\limits_{V}{\rho\,\phi\,dV}}
\end{equation}

$U$ : Energ�a electrost�tica.

Un poco de malabares algebraicos arrojan los siguientes resultados:
\begin{equation}
	\nabla\cdot\left(\phi\,\mathbf{D}\right)=\phi\nabla\cdot\mathbf{D}
			+\left(\nabla\phi\right)\cdot\mathbf{D}=\phi\,\rho_{f}+(-\mathbf{E})\cdot\mathbf{D}\;.
\end{equation}

Integrando sobre un volumen $V$ y aplicando el teorema de Stokes:
\begin{equation}
\begin{gathered}
	\iiint\limits_{V}{\nabla\cdot\left(\phi\,\mathbf{D}\right)\,dV}=
				\iiint\limits_{V}{\phi\,\rho_{f}}
				-\iiint\limits_{V}{\mathbf{E}\cdot\mathbf{D}\,dV} \medskip \\
	\oiint\limits_{S}{\phi\,\mathbf{D}\cdot d\mathbf{S}}=
				\iiint\limits_{V}{\phi\,\rho_{f}\,dV}
				-\iiint\limits_{V}{\mathbf{E}\cdot\mathbf{D}\,dV}\;.
\end{gathered}
\end{equation}

Por la condici�n de regularidad en infinito para $\phi$ y $\mathbf{D}$, si el volumen $V$ tiende a ser el universo, entonces $S$ ser�a la superficie l�mite del universo en donde el producto $\phi\,\mathbf{D}$ es un vector que decrece a raz�n de $\frac{1}{r^3}$, mientras el diferencial de superficie crece al ritmo de $r^2$. Se sigue que la integral en $S$ es nula:
\begin{equation}
		\iiint\limits_{All}{\phi\,\rho_{f}\,dV}=
		\iiint\limits_{All}{\mathbf{E}\cdot\mathbf{D}\,dV}\;.
\end{equation}

Finalmente,
\begin{equation}
	\boxed{U=\frac{1}{2}\iiint\limits_{All}{\rho_f\,\phi\,dV}=
	\frac{1}{2}\iiint\limits_{All}{\mathbf{E}\cdot\mathbf{D}\,dV}}
\end{equation}

N�tese que $All$ denota todo el espacio.


\subsection{Energ�a almacenada en un condensador.}


\begin{figure}[hb]
	\centering
		\includegraphics{7g2.2}
	\caption{Circuito el�ctrico de un condensador carg�ndose.}
	\label{g_ConCar}
\end{figure}

La fuente de voltaje $V_{DC}$ realiza un trabajo para depositar carga positiva en $b$ mientras retira carga de $a$. El trabajo realizado entre el momento en que se conecta el condensador descargado y el momento en que est� completamente cargado es $W=q V$, donde $q$ es la carga suministrada por la fuente y $V$ la diferencia de potencial instantanea que var�a desde $0 V$ hasta $V_{DC}$. El trabajo diferencial realizado cuando se ha suministrado una carga $dq$ es $dW=dq V$. Dado que la capacitancia es constante $C=Q/\Delta \phi$, la carga diferencial suministrada por la fueste se puede escribir en t�rminos del potencial instant�neo de la fuente as�: $dq=CdV$. En consecuencia el trabajo realizado por la fuente es:
\begin{equation}
W=\int\limits_{0}^{V_{DC}} C V dV=\frac{C V_{DC}^2}{2}.
\end{equation}

R�pidamente se puede demostrar que:
\begin{equation}
	U=\frac{Q^2}{2C}=\frac{1}{2}\,Q\,\phi=\frac{1}{2}\,C\,\phi^2\;.
\end{equation}
donde $Q$ es la carga de la placa positiva del condensador, $\phi$ es la diferencia de potencial positiva entre las placas y $C$ la capacitancia, naturalmente.
\subsection{Energ�a de ligadura de un �tomo.}


Si logramos simplificar el �tomo y considerarlo como una nube de electrones esf�rica, uniformemente cargada con una carga puntual (n�cleo) en el origen, como se observa en la gr�fica \ref{g_SimAt}, entonces para encontrar la energ�a, por superposici�n, primero se encontrar� la energ�a de la nube, luego se a�adir� el n�cleo.

\begin{figure}[hb]
	\centering
		\includegraphics{7g3.3}
	\caption{Simplificaci�n del �tomo. La nube de electrones y n�cleo con cargas $-Q$ y $Q$ respectivamente.}
	\label{g_SimAt}
\end{figure}

Para la nube:
\begin{equation}
	\mathbf{E_-}=\left\{
\begin{aligned}
	\frac{-Q}{4\pi\epsilon_0r^2}\,\mathbf{\hat{r}}\quad,\quad\text{si}\quad r>R\,.\medskip \\
	\frac{-Q\,r}{4\pi\epsilon_0R^3}\,\mathbf{\hat{r}}\quad,\quad\text{si}\quad r<R\,.\medskip \\
\end{aligned}
	\right.
\end{equation}

Luego, la energ�a almacenada en la nube de electrones es:
\begin{equation}
\begin{gathered}
	U_-=\frac{1}{2}\,\iiint\limits_{All}{\mathbf{E}\cdot\mathbf{D}\,dV}=\frac{1}{2}\,\epsilon_0\,
		\int\limits_{r=0}^{\infty}\int\limits_{\theta=0}^{\pi}\int\limits_{\varphi=0}^{2\pi}\
		{E^2r^2\sen\theta\,d\theta\,d\varphi\,dr}\medskip \\
	\boxed{U_-=\frac{3Q^2}{20\pi\epsilon_0R}}
\end{gathered}
\end{equation}

Luego se determina el trabajo necesario para colocar el n�cleo en el centro del volumen:
\begin{equation}
\begin{gathered}
	W_{\infty\to r=0}=-Q\,\int\limits_\infty^0{\mathbf{E}\cdot d\mathbf{l}}=
				Q\,\int\limits^\infty_0{\mathbf{E}\cdot dr\,\mathbf{\hat{r}}}=U_+\medskip \\
	U_+=Q\,\left[\int\limits_0^R{\frac{-Q\,r}{4\pi\epsilon_0R^3}
			\mathbf{\hat{r}}\cdot dr\,\mathbf{\hat{r}}}+
		\int\limits_R^\infty{\frac{-Q}{4\pi\epsilon_0r^2}\mathbf{\hat{r}}\cdot dr\,\mathbf{\hat{r}}}
		\right]\medskip \\
	U_+=\frac{-3Q}{8\pi\epsilon_0R}\;.
\end{gathered}
\end{equation}

La energ�a total es $U=U_-+U_+$, luego:
\begin{equation}
	U=\frac{-9Q^2}{40\pi\epsilon_0R}
\end{equation}

Y �obs�rvese que esta energ�a es negativa!


\subsection{Fuerzas sobre conductores.}


Imag�nese una porci�n de un conductor como se muestra en la figura \ref{g_PorCon}.

\begin{figure}[hb]
	\centering
		\includegraphics{7g4.4}
	\caption{Porci�n de un conductor.}
	\label{g_PorCon}
\end{figure}

El diferencial de superficie $dS$ genera un campo $\mathbf{E_{dso}}$ fuera del conductor y $\mathbf{E_{dsi}}$ dentro del conductor. El resto de la carga produce un campo $\mathbf{E_{resto}}$.

Si el conductor est� en condiciones electrost�ticas, el campo en su interior es nulo:
\begin{equation}
	\mathbf{E_{resto}}=-\mathbf{E_{dsi}}\quad\text{y}\quad
	\mathbf{E_{dso}}=-\mathbf{E_{dsi}}=\frac{\sigma}{2\epsilon}\,\mathbf{\hat{n}}\;.
\end{equation}

El campo total justo fuera del conductor es:
\begin{equation}
	\mathbf{E_T}=\mathbf{E_{dso}}+\mathbf{E_{resto}}=2\,\mathbf{E_{resto}}\;.
\end{equation}

La fuerza sobre el $dS$ es $d\mathbf{F}=\mathbf{E_{resto}}\,\sigma\,dS$, luego:
\begin{equation}
\begin{gathered}
	\mathbf{F}=\iint\limits_{S}{\mathbf{E_{resto}}\,\sigma\,dS}\medskip \\
	\boxed{\mathbf{F}=\frac{1}{2}\,\iint\limits_{S}{\frac{\sigma^2}{\epsilon}\,d\mathbf{S}}}
\end{gathered}
\end{equation}


\subsection{Relaci�n fuerza energ�a.}


Se sabe que la energ�a electrost�tica es el trabajo necesario para formar la distribuci�n de carga. Entonces:
\begin{equation}
	U=W=\int\limits_{C}{\mathbf{F_a}\cdot\,d\mathbf{r}}=
	-\int\limits_{C}{q\,\mathbf{E}\cdot\mathbf{r}}\;,
\end{equation}

donde $\mathbf{F_a}$ es la fuerza del agente externo que realiza trabajo. Luego $dU=-\mathbf{F}\cdot d\mathbf{r}$.

Pero $dU=\nabla U\cdot d\mathbf{r}=-\mathbf{F}\cdot d\mathbf{r}$ y como la igualdad se sostiene sin importar la curva:
\begin{equation}
	\boxed{\mathbf{F}=-\nabla U}
\end{equation}

Aqu� $\mathbf{F}$ es la fuerza generada por el campo.


\subsection{Ilustraci�n.}


Se desea calcular la fuerza en un condensador de placas paralelas, como se ilustra en la figura \ref{g_FuerConPla}.

\begin{figure}[hb]
	\centering
		\includegraphics{7g5.5}
	\caption{Condensador de placas paralelas de �rea $A$.}
	\label{g_FuerConPla}
\end{figure}

Se considera el condensador cargado con una carga $Q$. Para la placa inferior:
\begin{equation}\label{ec_FuerAtrac}
		\mathbf{F}=\frac{1}{2}\,\iint\limits_{S}{\frac{\left(Q/A\right)^2}{\epsilon}\,d\mathbf{S}}=
		\frac{1}{2}\,\frac{Q^2}{A^2\epsilon}\,\iint\limits_{S}{d\mathbf{S}}=
		\frac{1}{2}\,\frac{Q^2}{A^2\epsilon}\,A\,\mathbf{\hat{x}}\;.
\end{equation}

La ecuaci�n \eqref{ec_FuerAtrac} da a entender que es una fuerza de atracci�n. Por el lado de la energ�a:
\begin{equation}\label{ec_FuerPla}
\begin{gathered}
	U=\frac{1}{2}\,\frac{Q^2}{C}\quad,\quad
		C=\frac{A\epsilon}{x}\quad\Rightarrow\quad
		U=\frac{1}{2}\,\frac{Q^2}{A\epsilon}\,x \medskip \\
	\mathbf{F}=-\nabla U \quad\Rightarrow\quad
		\mathbf{F}=-\frac{1}{2}\,\frac{Q^2}{A\epsilon}\,\mathbf{\hat{x}}\;.
\end{gathered}
\end{equation}

La �ltima expresi�n de la ecuaci�n \eqref{ec_FuerPla} se interpreta como la fuerza sobre la placa superior.

El signo negativo en $	\mathbf{F}=-\nabla U$ expresa la tendencia natural que tiene el sistema el�ctrico a tratar de disminuir su energ�a potencial, luego s� se presenta confusi�n en el signo de la fuerza; �ste de ser elegido seg�n el criterio anterior.

\section{Energ�a Magn�tica}

Es el trabajo ``irreversible'' realizado por un agente externo para establecer las corrientes en un circuito.
\begin{equation}\label{ec_trabajom_en}
	V=\frac{W}{q}\quad\text{y de manera diferencial}\quad dW=V dq
\end{equation}

Con las siguientes consideraciones:
\begin{equation}\label{ec_trabajom_consideraciones}
	V=-\mathscr{E}\quad\text{;}\quad i=\frac{dq}{dt}
\end{equation}
Se sigue el siguiente desarrollo, donde se considera que $U_B$ es la energ�a magn�tica:
\begin{equation}\label{ec_trabajom1}
	dW=dU_B=\mathscr{E}idt
\end{equation}
\begin{equation}\label{ec_trabajom2}
	dU_B=id\Phi_B\quad\text{con}\quad L=\frac{\Phi_B}{i}\quad\text{Autoinductancia}
\end{equation}
\begin{equation}\label{ec_trabajom3}
	dU_B=iLdi
\end{equation}
\begin{equation}\label{ec_trabajom5}
	U_B=\int\limits_0^I Lidi=\frac{LI^2}{2}
\end{equation}
Y finalmente:
\begin{equation}\label{ec_trabajom6}
	\boxed{U_B=\frac{LI^2}{2}.}
\end{equation}

La ecuaci�n \ref{ec_trabajom6} es la energ�a magn�tica almacenada en un circuito que tiene la corriente de establecimiento $I$ y que tiene por autoinductancia $L$. Existen otras dos expresiones equivalentes a \ref{ec_trabajom6}, si se hacen los reemplazos convenientes al utilizar la relaci�n entre flujo magn�tico enlazado y corriente.

\begin{equation}\label{ec_flujom1}
	\Phi_B\,=LI\quad\text{Donde L es la autoinductancia}
\end{equation}

De all� se pueden obtener las siguientes igualdades:
\begin{equation}\label{ec_flujom2}
	U_B=\frac{LI^2}{2}=\frac{I\Phi_B}{2}
\end{equation}

\subsection{Energ�a magn�tica de una distribuci�n volum�trica de corriente}


El flujo magn�tico que atraviesa una superficie $S$ es:
\begin{equation}\label{ec_flujom3}
	\Phi_B=\iint\limits_{S}\bm {B}\cdot d\bm S
\end{equation}
Aprovechando que todo campo magn�tico es el rotacional del potencial vectorial magn�tico ($\bm B=\bm\nabla\times\bm A$) y haciendo uso del teorema del rotacional de Stokes
\begin{equation}\label{ec_flujom4}
	\Phi_B=\iint\limits_{S}\bm{\nabla}\times\bm{A}\cdot d\bm S = \oint\limits_{C}\bm{A}\cdot d\bm{r}
\end{equation}

Y para la energ�a magn�tica:
\begin{equation}\label{ec_flujom6}
	\boxed{U_B=\frac{I}{2}\oint\limits_{C}\bm{A}\cdot\,d\bm{r}}
\end{equation}

Los elementos diferenciales de corriente son equivalentes entre s�, tal como se ilustra en \ref{ec_flujom7}
\begin{equation}\label{ec_flujom7}
 Id\bm{r}=\bm{I}dr=\bm{K}dS=\bm{J}dv
\end{equation}

La versi�n de \ref{ec_flujom6} para el caso de una distribuci�n volum�trica de corriente es  as�:
\begin{equation}\label{ec_flujom8}
 U_B=\frac{1}{2} \iiint\limits_{V}\bm{A}\cdot\bm{J}dv
\end{equation}
Ahora, considerando la ley de Ampere para el integrando de \ref{ec_flujom8} se puede llegar a la siguiente expresi�n:
\begin{equation}\label{ec_flujom9}
	\bm{A}\cdot\bm{J}=\frac{1}{\mu_0}\bm{A}\cdot\bm{\nabla}\times\bm{B}=
	\frac{1}{\mu_0}\big(\bm{B}\cdot\bm{\nabla}\times\bm{A}-\bm{\nabla}\cdot(\bm{A}\times\bm{B})
	\big)
\end{equation}

Remplazando lo anterior en \eqref{ec_flujom8} e integrando sobre todo el universo se tiene:
\begin{equation}\label{ec_flujom10}
\begin{aligned}
	\boxed{U_B=\frac{1}{2\mu_0}\iiint\limits_{All}B^2dv}\quad \\
	&U_B=\iiint\limits_{All}\frac{\bm{H}\cdot\bm{B}}{2}dv
\end{aligned}
\end{equation}

\subsection{Ilustraci�n}
Encontrar la inductancia por unidad de longitud de un cable coaxial.
%Imagen de el cable coaxial
%Imagen de el cable coaxial
%Imagen de el cable coaxial
%Imagen de el cable coaxial
%Imagen de el cable coaxial

Se supone una corriente uniformemente distribuida $I_0$ en $\hat {z}$ en el conductor interno, e $I_0$ en $-\hat {z}$ en el conductor externo.

La estrategia es la siguiente: aprovechando que el sistema tiene las suficientes simetr�as, se puede calcular tanto $\bm B$ como $\bm H$ en todo punto. Con estos resultados se puede averiguar c�mo es la energ�a magn�tica almacenada en una porci�n de cable de longitud $h$. Finalmente, teniendo en cuenta la ecuaci�n \eqref{ec_flujom10} que relaciona la energ�a magn�tica almacenada en una inductacia y su autoinductancia, se puede obtener la variable pedida.

\begin{equation}\label{eq_ilust1}
\begin{aligned}
	\bm{\nabla}\times\bm{H}=\bm{J}_f\quad \rightarrow
	\iint\limits_{S}\bm{\nabla}\times\bm{H}\cdot\,d\bm{s}=\iint\limits_{S}\bm{J}_F\cdot d\bm{s}
\end{aligned}
\end{equation}

De donde es claro que:
\begin{equation}\label{ec_ejercicio1}
 \oint\limits_C\bm{H}\cdot d\bm{r}=I\quad\text{Forma integral de la ley de Ampere para H}\quad
\end{equation}

$\bullet$ Calculando para $\rho<a$:
%Grafico del corte transversal
%Grafico del corte transversal
\begin{equation}\label{ec_1}
	2\pi\rho H =I'\quad\text{Con}\quad I'=I_0\left(\frac{\rho}{a}\right)^2\quad \rightarrow \bm{H}=\frac{I_0\rho}{2\pi\,a^2}\hat{\bm\varphi}
\end{equation}

$\bullet$ Calculando para $a<\rho<b$:
\begin{equation}\label{ec_2}
	2\pi\rho H=I_0\quad\bm{H}=\frac{I_0}{2\pi\rho}\hat{\bm\varphi}
\end{equation}

$\bullet$ Calculando para $b<\rho<c$:
\begin{equation}\label{ec_3}
	\oint\limits_{C'}\bm{H}\cdot\,d\bm{r}=I_0\frac{\rho^2-b^2}{c^2-b^2}\quad
	\bm{H}=\frac{I_0\,(\rho^2-b^2)}{2\pi\rho\,(c^2-b^2)}\hat{\bm\varphi}
\end{equation}

En este todos los medios son similares al vac�o, luego $\bm{B}=\mu_0\bm{H}$. La energ�a magn�tica se puede escribir como:
\begin{equation}\label{ec_4}
	U_B=\iiint\limits_{V}\frac{\mu_0\,H^2}{2}dv
\end{equation}

Utilizando esta ecuaci�n junto con los resultados anteriores, se llega luego de integrar a:
\begin{equation}\label{ec_5}
\boxed{\frac{L}{h}=\frac{\mu_0}{8\pi}+\frac{\mu_0}{2\pi}\ln\left(\frac{b}{a}\right)+\frac{\mu_0}{2\pi(c^2-b^2)^2}\left[c^4\ln\left(\frac{c}{b}\right)-c^2(c^2-b^2)+\frac{c^4-b^4}{4}\right]}
\end{equation}

La inductancia se puede dividir en inductancia interna e inductancia externa. La primera mide la capacidad de almacenar energ�a en forma de campo magn�tico en el interior de los conductores que forman el inductor. La segunda hace lo propio pero con el la energ�a almacenada fuera de los conductores.

Para el caso del cable coaxial, la inductancia externa mide la capacidad de almacenar energ�a en forma de campo magn�tico en la regi�n exterior a ambos conductores, es decir en $a<\rho<b$.
\begin{equation}\label{ec_6}
	L_{ext}=\frac{\mu_0\,h}{2\pi}\ln\left(\frac{b}{a}\right).
\end{equation}

La inductancia interna del conductor interno es $\displaystyle\frac{\mu_0}{8\pi}$, y no depende del radio de �ste. Algo que podr�a ser sospechoso a primera vista pero que no guarda mayores dificultades si se considera que a mayor inductancia, mayor capacidad de almacenar energ�a en forma de campo magn�tico. Ello implica que, para prop�sitos de comparaci�n, dos inductores con autoinductancias $L_1$ y $L_2$ que satisfagan la relaci�n $L_1<L_2$; a igual corriente la energ�a magn�tica almacenada por el primero es mayor que la correspondiente almacenada por el segundo (siguiendo a \ref{ec_trabajom6}).

\section{Potencia electromagn�tica}

Consid�rese un volumen cargado, sometido a la presencia de los campos $\bm{E}$ y $\bm{B}$.
%Imagen del volumen con las lines de campos
%Imagen del volumen con las lines de campos
%Imagen del volumen con las lines de campos
%Imagen del volumen con las lines de campos
%Imagen del volumen con las lines de campos
%Imagen del volumen con las lines de campos

La fuerza realizada por los campos cuando un agente externo mueve un diferencial de carga $dq$ es
\begin{equation}\label{ec_6}
	d\bm{F}=dq(\bm{E}+\bm{\upsilon}\times\bm{B}).
\end{equation}
Dado que el diferencial de trabajo debido al movimiento diferencial $d\bm{r}$ es $dW=\bm{F}\cdot d\bm{r}$, se puede expresar el trabajo diferencial realizado por los campos cuando un agente externo mueve el $dq$ un diferencial de l�nea $d\bm{r}$ as�:
\begin{equation}\label{ec_7}
  \begin{split}
	d^2 W &=dq(\bm{E}+\bm{v}\times\bm{B})\cdot d\bm{r}\\
	&=dq\bm{E}\cdot d\bm{r}+dq(\bm{v}\times\bm{B}\cdot d\bm{r})\\
	&=\rho_f dV\bm{E}\cdot\bm{v}dt+dq\frac{d\bm{r}}{dt}\times\bm{B}\cdot dr\\
	&=\rho_f\bm{v}\cdot\bm{E}dVdt+id\bm{r}\cdot d\bm{r}\times\bm{B}\\
	&=\bm{J}_f\cdot\bm{E}dVdt
  \end{split}
\end{equation}

% \begin{align}
% 	d^2 W &=dq(\bm{E}+\bm{v}\times\bm{B})\cdot d\bm{r}\\ \nonumber
% 	&=dq\bm{E}\cdot d\bm{r}+dq(\bm{v}\times\bm{B}\cdot d\bm{r})\\ \nonumber
% 	&=\rho_f dV\bm{E}\cdot\bm{v}dt+dq\frac{d\bm{r}}{dt}\times\bm{B}\cdot\,dr\\ \nonumber
% 	&=\rho_f\bm{v}\cdot\bm{E}dVdt+id\bm{r}\cdot d\bm{r}\times\bm{B}\\ \nonumber
% 	&=\bm{J}_f\cdot\bm{E}dVdt
% \end{align}

En \ref{ec_7} debe tenerse en cuenta la relaci�n existente entre la densidad volumetrica de corriente libre y la velocidad $\bm{J}_f=\rho_f\bm{V}$, as� como la propiedad del producto triple $\bm{A}\cdot\bm{B}\times\bm{C}=\bm{A}\times\bm{B}\cdot\bm{C}$.

Finalmente se obtiene \ref{ec_8}, donde se observa que el campo magn�tico \emph{NO HACE TRABAJO} sobre las cargas en movimiento.
\begin{equation}
 \label{ec_8}
	\frac{d^2W}{dt}=dP=\bm{J}_f\cdot\bm{E}dV
\end{equation}

que naturalmente implica:
\begin{equation}\label{ec_potencia}
	\boxed{P=\iiint\limits_{V}\bm{J}_f\,\cdot\bm{E}dv}
\end{equation}

La interpretaci�n que se realiza para la ecuaci�n \ref{ec_potencia} es la siguiente:
\begin{equation}\label{ec_potencia2}
 \bm{J}_f\cdot\bm{E}=\left\{
 \begin{array}{cl}
 >0 \ ; &P\text{ \ se disipada en forma de calor.}\\
 <0 \ ; &P\text{ \ es suministrada y se transforma en potencia electromagn�tica.}
 \end{array}\right.
\end{equation}

\subsection{Teorema de Poynting}

Independiente de si la potencia en \ref{ec_potencia} es mayor o menor que cero, se puede escribir su integrando con ayuda de la ley de Ampere para $\bm H$ as�:
\begin{equation}\label{ec_9}
\begin{split}
  \bm{\nabla}\times\bm{H}&=\bm{J}_f\,+\frac{\partial\bm{D}}{\partial\,t}\\
  \bm{J}_f\cdot\bm{E}=\left(\bm{\nabla}\times\bm{H}-\frac{\partial\bm{D}}{\partial t}\right)\cdot\bm{E}&=\bm{H}\left(\frac{-\partial\bm{B}}{\partial\,t}\right)-\bm{\nabla}\cdot(\bm{E}\times\bm{H})-\frac{\partial\bm{D}}{\partial\,t}\cdot\bm{E}
\end{split}
\end{equation}
Integrando sobre un volumen arbitrario $V$:
\begin{equation}\label{ec_10}
\iiint\limits_{V}\bm{J}_f\cdot\bm{E}= P =-\iiint\limits_{V}\bm{\nabla}\cdot(\bm{E}\times\bm{H})dV-\iiint\limits_{V}\,\left(\bm{H}\cdot\frac{\partial\bm{B}}{\partial t}+\bm{E}\cdot\frac{\partial\bm{D}}{\partial t}\right)dV
\end{equation}
Se puede demostrar f�cilmente que
\begin{equation}\label{ec_11}
% \frac{\partial\bm{H}\bm{B}}{\partial\,t}=\frac{\partial\bm{H}}{\partial\,t}\cdot\bm{B}+\frac{\partial\bm{B}}{\partial\,t}\cdot\bm{H}\quad\text{;}\quad\bm{B}=\mu\bm{H}\\
\bm{H}\frac{\partial\bm{B}}{\partial t}=\frac{1}{2}\frac{\partial\bm H\cdot\bm B}{\partial t}\quad\text{y}\quad\bm{E}\frac{\partial\bm{D}}{\partial\,t}=\frac{1}{2}\frac{\partial \bm{E}\cdot\bm{D}}{\partial t}
\end{equation}
Reemplazando \ref{ec_11} en \ref{ec_10} y aplicando el teorema de la divergencia de Gauss:
\begin{equation}\label{ec_potencia3}
P+\oiint\limits_{S}\bm{E}\times\bm{H}\cdot\,d\bm{S}+\iiint\limits_{V}\frac{\partial}{\partial\,t}\left(\frac{\bm{H}\cdot\bm{B}}{2}+\frac{\bm{D}\cdot\bm{E}}{2}\right)dV=0
\end{equation}
Se realizan las siguientes definiciones:
\begin{equation}\label{ec_potencia4}
\begin{split}
	\mu_E\equiv&\frac{\bm{E}\cdot\bm{D}}{2}\quad\text{Densidad de energ�a el�ctrica}.\\
	\mu_B\equiv&\frac{\bm{H}\cdot\bm{B}}{2}\quad\text{Densidad de energ�a magn�tica}.\\
	\bm{\mathbb S}\equiv&\bm{E}\times\bm{H}\quad\text{Vector de Poynting}\\
	\mu_{EB}\equiv&\mu_E+\mu_B\quad\text{Densidad de energ�a electromagn�tica}.\\
	U_{EB}\equiv&\iiint\limits_{V}\mu_{EB}dV\quad\text{Emerg�a electromagn�tica en}V.
\end{split}
\end{equation}

Finalmente, el teorema de Poynting:
\begin{equation}\label{ec_potencia5}
	\boxed{P+\oiint\limits_{S}\bm{\mathbb S}\cdot d\bm{S}+\frac{dU_{EB}}{dt}=0}
\end{equation}

%%%%%%%%%%%%%%%%%%
%%%%%%%%%%%%%%%%%%
%%%%%%%%%%%%%%%%%%
%%%%%%%%%%%%%%%%%% aqu� voy
%%%%%%%%%%%%%%%%%%
%%%%%%%%%%%%%%%%%%
%%%%%%%%%%%%%%%%%%

\subsection{Ilustraci�n}
El siguiente ejercicio muestra la perdida de la energ�a en el sistema.
%Circuitico de dos capacitores
%Circuitico de dos capacitores
%Circuitico de dos capacitores
%Circuitico de dos capacitores
Se tiene $U_0=\frac{Q^2}{2C}$, y $U_f=\frac{Q^2}{2(2C)}=\frac{Q^2}{4C}$
De esto, se tiene:
\begin{equation}\label{ec_12}
	\Delta\,U_E=\frac{Q^2}{4C}-\frac{Q^2}{2C}=\frac{-Q^2}{4C}
\end{equation}
\begin{equation}
	U_0=2U_f\Rightarrow\,U_0\,>U_F
\end{equation}

Entonces, $?$Que pas� con la energ�a magn�tica$?$
\begin{equation}\label{ec_13}
\begin{aligned}
	P=0\quad\text{;}\quad\frac{dU_EB}{dt}\approx\frac{U_f\,-U_0}{\Delta\,t}<0\\
	&\oiint\limits_{S}\bm{S}\cdot\,d\bm{s}\,>0\quad\bm{S}\text{: densidad de potencia radiada[=]}\frac{W}{m^2}
\end{aligned}
\end{equation}

\subsection{6.3.3 Disipaci�n de potencia en una resistencia.}

%Dibujo de una resistencia
%Dibujo de una resistencia
%Dibujo de una resistencia
%Dibujo de una resistencia
%Dibujo de una resistencia
Se tienen las siguientes consideraciones:
	$\ast$Supongase que un agente externo inyecta una corriente que se distribuye uniformemente en el volumen.
	$\ast$L$>>$R

$\bullet$Entienda $\sigma$ como la conductividad del material: $R=\frac{L}{\sigma\,A}$

Comenzemos con el an�lisis deseado:

	Se supone que $\bm{E}=E_0\hat {z}$, y por lo tanto, $\bm{J}_f=\sigma\,E_0\hat {z}$; si ahora aplicamos \eqref{ec_ejercicio1} 			para una espira de radio $\rho\,<R$ obtenemos lo siguiente:
\begin{equation}\label{ec_14}
	\bm{H}=\frac{I_0\rho}{2\pi\,R^2}\hat{\varphi}
\end{equation}

	Ahora ulitizamos \eqref{ec_potencia4} para encontrar el valor de $\bm{S}$; as�:
\begin{equation}\label{ec_15}
\begin{aligned}
	\bm{S}=\bm{E}\times\bm{H}=E_0\hat{z}\times\frac{I_0\rho}{2\pi\,R^2}\hat{\varphi}\\
	&\bm{S}=\frac{-E_0\,I_0\rho}{2\pi\,R^2}\hat{\rho}
\end{aligned}
\end{equation}

%Dibujito del vector de poynting entrando a la resistencia(corte)
%Dibujito del vector de poynting entrando a la resistencia(corte)

	Calculando la integral cerrada sobre la superficie de la resistencia, tal como se muestra en \eqref{ec_potencia5}:
\begin{equation}\label{ec_16}
\begin{aligned}
\oiint\limits_{S}\bm{S}\cdot\,d\bm{s}=\frac{-E_0\,I_0}{2\phi\,R^2}\iint\limits_{S'}\rho\cdot\,d\bm{s}\hat{\rho}\quad\text{Solo se eval�a la frontera latera ya}\\
	&=\frac{-E_0\,I_0}{2\phi\,R}\iint\limits_{S'}\,ds\hat{\rho}\cdot\hat{\varphi}\qquad\text{ que $\bm{S}\cdot\,d\bm{s}$ es 0 en las tapas.}
	&\oiint\limits_{S}\bm{S}\cdot\,d\bm{s}=-E_0\,I_0\,L
\end{aligned}
\end{equation}

	Es claro que no se tienen campos $\bm{E}$ y $\bm{B}$ variables en el tiempo en este ejercicio, por tanto con \eqref{ec_potencia5} podemos encontrar el valor de P.
\begin{equation}\label{ec_17}
	P=E_0\,I_0\,L
\end{equation}
	Ahora, a modo de prueba, vamos a partir de la ecuaci�n de potencia trabajada en los circuitos el�ctricos:
\begin{equation}\label{ec_18}
	P=IV=I^2\,R=\frac{V^2}{R}
\end{equation}
	Y expresando la resistencia del cilindro en terminos de su conductividad como $R=\frac{L}{\sigma\pi\,R^2}$, tenemos:
\begin{equation}\label{ec_19}
	P=\frac{I_0\,^2\,L}{\sigma\pi\,R^2}=(I_0\,L)\frac{I_0}{\sigma\pi\,R^2}
\end{equation}
	Pero $J_f$ se define como la corriente por unidad de �rea, es decir $\frac{I_0}{\pi\,R^2}$, de all� que:
\begin{equation}\label{ec_20}
	P=\frac{I_0\,LJ_f}{\sigma}=\frac{I_0\,L\sigma\,E_0}{\sigma}=E_0\,I_0\,L
\end{equation}

Lo cual es consecuente con lo que predice el Teorema de Poynting.


\chapter{Campo el�ctrico de movimiento}

En este cap�tulo, vamos a estudiar lo que acontece cuando se cambia el sistema de referencia sobre un fen�meno electromagn�tico.
%Echar mas carreta
%Echar mas carreta
%Echar mas carreta

%Todo estudiante de teor�a electromagn�tiga sabe que:
%\begin{equation}
%	\bm{F}=q[\bm{E}+\bm{V}\times\bm{B}];.
%\end{equation}

Definimos ahora un nuevo sistema de referencia inercial, que se mueve con una carga, con velocidad $Vo\hat{x}$

%Imagen de los sistemas de referencia
%Imagen de los sistemas de referencia
%Imagen de los sistemas de referencia

Para un punto , se tiene que Z=Z' y Y=Y', adem�s que en $t=0$ los centros de coordenadas de ambos sistemas coinciden.

$\bullet$Como los dos sistemas de referencia son inerciales:
\begin{equation}\label{ec_21}
	\bm{a}=\frac{d^2\bm{r}}{dt^2}=\frac{d\bm{v}}{dt}=\frac{d\bm{v'}}{dt}
\end{equation}

$\bullet$ Supongamos $\bm{F}_EB=q[\bm{E}+\bm{V}\times\bm{B}]$ como un invariante Galileano\footnote{Derivado del principio de la relatividad, seg�n el cual las leyes fundamentales de la f�sica son las mismas en todos los sistemas de referencia inerciales.}.

\section{7.1. Ilustraci�n.}

Supongamos un campo magn�tico en $\hat{-x}$, y una carga que se mueve con velocidad $V_0\hat{y}$, como muestra la figura.
%La figurita "animada" de la carga
%La figurita "animada" de la carga
%La figurita "animada" de la carga
Para la carga, se tiene la siguiente fuerza debida a los efectos magn�ticos:
\begin{equation}\label{ec_22}
	\bm{F}=qV_0\,B_0\hat{z}
\end{equation}

Ahora tomemos un marco de referencia diferente, as�:
%La figurita "animada" de la carga2
%La figurita "animada" de la carga2
%La figurita "animada" de la carga2
En este caso, el sistema de coordenadas se mueve con la carga, y por tanto $\bm{V}=0$. Esto trae como consecuencia, que la fuerza debida al campo magn�tico sea 0; pero como ya se hab�a mencionado, los efectos de fuerza sobre la carga no pueden cambiar entre los diferentes sistemas de referencia inerciales. Es por esto, que se debe asociar una fuerza $\bm{F'}$ a la carga, que responda a la fuerza debida al campo magn�tico.
\begin{equation}\label{ec_23}
	\bm{F'}=\bm{F}_B
\end{equation}
Suponemos un campo magn�tico $\bm{E'}$:
\begin{equation}\label{ec_24}
\begin{aligned}
	\bm{F'}=\bm{F}_B\\
	&q[\bm{V}\times\bm{B}]=q\bm{E'}
\end{aligned}
\end{equation}
De donde finalmente:
\begin{equation}\label{ec_25}
	\bm{E}=V_0\,B_0\hat{z}
\end{equation}

Este es el campo el�ctrico de movimiento.

\subsection{7.1.1. Efecto Hall.}

El efecto hall, es el fen�meno que ocurre en un conductor por el que circula una corriente, en presencia de un campo magn�tico perpendicular al movimiento de las cargas.

Al pasar las cargas, estas se ven afectadas por el campo magn�tico, y se van acumulando de uno de los lados del conductor; a medida que se acumulan cargas, se inducen otras de signo opuesto en el otro extremo del conductor produciendo finalmente un campo el�ctrico en la zona afectada por el campo magn�rico.
Cuando el fen�meno se ha estabilizado, las cargas que pasan por la zona afectada por el campo magn�tico, no sienten fuerza alguna, ya que la fuerza magn�tica es compensada con la fuerza el�ctrica.
%Figuritas del efeto hall
%Figuritas del efeto hall
%Figuritas del efeto hall
%Figuritas del efeto hall

De lo anterior se obtiene lo siguiente, analizando el fen�meno desde un sistema de referencia est�tico:
\begin{equation}\label{ec_26}
\begin{aligned}
	\bm{F}_E=q\bm{E}\quad\\bm{F}_B=q\bm{V}\times\bm{B}\\
	&\bm{F}_EB=\bm{0}=\bm{F}_E\,+\bm{F}_B\\
	&\bm{F}_E=-\bm{F}_B\\
	&\bm{E}=-\bm{V}\times\bm{B}=V_0\,B_0\hat{x};.
\end{aligned}
\end{equation}

Ahora haciendo el analisis desde un sistema de referencia centrado en la carga, y con velocidad igual a esta:
\begin{equation}\label{ec_27}
	\bm{E'}=\bm{V}\times\bm{B}=-V_0\,B_0\hat{x}
\end{equation}

Calculando el campo el�ctrico total:
\begin{equation}\label{ec_28}
	\bm{E}_T=\bm{E}+\bm{E'}=\bm{0}
\end{equation}

%?Nombre del capitulo?
%?Nombre del capitulo?
%?Nombre del capitulo?
\chapter{Campos electromagn�ticos y la ecuaci�n de onda}

\section{Corriente de desplazamiento}
Cuando reci�n se plante�, la ley de Ampere arrojaba resultados il�gicos, al abordar ciertos problemas.
Por ejemplo con el caso de un condensador.
$?`$Porqu�?
%Dibujo de un condensador
%Dibujo de un condensador
%Dibujo de un condensador
Aplicando la ley de Ampere sobre una circunferencia alrededor del alambre, se obtiene lo siguiente:
\begin{equation}\label{ec_29}
\begin{aligned}
	\oint\bm{B}\cdot\,d\bm{r}=\mu_0\,I\\
	&\bm{B}=\frac{\mu_0\,I_0}{2\pi\rho}\hat{\varphi}
\end{aligned}
\end{equation}

Pero ahora, si aplicamos la misma ecuaci�n en otra linea de integraci�n, as�:
%Imagen del corte del condensador
%Imagen del corte del condensador
%Imagen del corte del condensador
\begin{equation}\label{ec_30}
\oint\limits_{C}\bm{B}\cdot\,d\bm{r}=0\quad
\end{equation}
Lo cual, esta claramente mal.

Tiempo despu�s lleg� el brit�nico James Clerk Maxwell e hizo el siguiente an�lisis sobre la ley de Ampere:
\begin{equation}\label{ec_31}
\begin{aligned}
 	\bm{\nabla}\times\bm{B}=\mu_0\bm{J}\\
	&\bm{\nabla}\cdot\,[\bm{\nabla}\times\bm{B}]=0=\mu_0\bm{\nabla}\cdot\bm{J};.
\end{aligned}
\end{equation}
De la ley de continuidad se tiene lo siguiente:
\begin{equation}\label{ec_32}
	\bm{\nabla}\cdot\bm{J}=\frac{-\partial\rho}{\partial\,t};.
\end{equation}

Suponiendo que la corriente es continua a�n en el condensador, y continuando con el an�lisis:
\begin{equation}\label{ec_33}
\begin{aligned}
	\bm{\nabla}\times\bm{H}=\bm{J}_f+\bm{J}_d\\
	&\bm{\nabla}\cdot\,(\bm{J}_f+\bm{J}_d)\\
	&\bm{\nabla}\cdot\bm{D}=\rho_f\\
	&\bm{J}_d=\frac{\partial\bm{D}}{\partial\,t}\\
	&\boxed{\bm{\nabla}\times\bm{H}=\bm{J}_f+\frac{\partial\bm{D}}{\partial\,t}}\quad\text{Ley de Ampere-Maxwell para $\bm{H}$}\\
	&\boxed{\bm{\nabla}\times\bm{B}=\mu_0\bm{J}_f+\mu\epsilon\frac{\partial\bm{E}}{\partial\,t}}\quad\text{Ley de Ampere-Maxwell para $\bm{B}$}
\end{aligned}
\end{equation}

Volviendo ahora al ejercicio del condensador
%Otra vista del condensador
%Otra vista del condensador
%Otra vista del condensador

Como la corriente debe ser continua, se sabe que $I_d=I_0$, pero a continuaci�n se har� una breve demostraci�n de ello, y de otros valores importantes.

Para encontrar $I_d$ debemos conocer primero cuanto vale $\bm{D}$:
\begin{equation}\label{ec_34}
\begin{aligned}
	\oiint\bm{D}\cdot\,d\bm{s}=Q_fenc\\
	&Q=\int\limits_{0}^{t}\,Idt=I_0\,t\\
	&\bm{D}=\frac{I_0\,t}{A}\hat{z}\\
	&\frac{\partial\bm{D}}{\partial\,t}=\frac{\partial}{\partial\,t}[\frac{I_0\,t}{A}]\hat{z}=\frac{I_0\,\hat{z}}{A}\\
	&T_d=\iint\limits_{S}\frac{I_0\hat{z}\cdot\,ds\hat{z}}{A}=I_0;.
\end{aligned}
\end{equation}

Ahora, para calcular $\bm{H}$
\begin{equation}\label{ec_35}
\begin{aligned}
	\bm{\nabla}\times\bm{H}=\bm{J}_f+\frac{\partial\bm{D}}{\partial\,t}\\
	&\bm{\nabla}\times\bm{H}=\frac{\partial\bm{D}}{\partial\,t}\quad\text{Dentro del condensador.}\\
	&\oint\bm{H}\cdot\,d\bm{r}=\iint\limits_{S}\frac{\partial\bm{D}}{\partial\,t}\,d\bm{s}\\
	&\bm{H}=\bm{I_0\rho}{2A}\hat{\varphi};.
\end{aligned}
\end{equation}

Y finalmente para el c�lculo de $\bm{S}$
\begin{equation}\label{ec_36}
\begin{aligned}
	\bm{S}=\bm{E}\times\bm{H}=\frac{I_0\,t}{\epsilon_0\,A}\hat{z}\times\frac{I_0\rho}{2A}\hat{\varphi}\\
	&=\frac{-I_0\,^2\,t\rho}{2\epsilon\,A^2}\hat{\rho}\\
	&\bm{S}=\frac{-|\bm{J}_d\,|^2\,t\rho}{2\epsilon_0}\hat{\rho};.
\end{aligned}
\end{equation}

\section{Los campos electromagn�ticos y la ecuaci�n de onda.}

%Carreta
%Carreta
%Carreta
\begin{equation}\label{ec_37}
\begin{aligned}
	\bm{\nabla}\times\,(\bm{\nabla}\times\bm{E})=\frac{-\partial}{\partial\,t}\bm{\nabla}\times\bm{B}\\
\bm{\nabla}\,(\bm{\nabla}\cdot\bm{E})-\nabla\,^2\bm{E}=\frac{-\partial}{\partial\,t}[\mu_0\bm{J}+\mu\epsilon\frac{\partial\bm{E}}{\partial\,t}];.
\end{aligned}
\end{equation}

Si asumimos las condiciones en el vacio:
\begin{equation}\label{ec_39}
	\bm{\nabla}\cdot\bm{E}=\frac{\rho}{\epsilon_0}=0\quad\bm{J}\quad\mu=\mu_0\quad\epsilon=\epsilon_0;.
\end{equation}

Finalmente se obtiene la siguiente ecuaci�n
\begin{equation}\label{ec_ondas1}
	\boxed{\nabla^2\bm{E}=\mu_0\epsilon_0\frac{\partial^2\bm{E}}{\partial\,t^2}}
\end{equation}

Claramente, la ecuaci�n anterior satisface la ecuaci�n de onda, para la cual
\begin{equation}\label{ec_ondas2}
	\bm{V}_prop=\frac{1}{\sqrt{\mu_0\epsilon_0}};.
\end{equation}

Para el vacio, se sabe que $\bm{V}_prop=c$, donde c es la velocidad de la luz.

''La luz es un caso particular de las ondas electromagn�ticas''

\subsection{Soluci�n de la ecuaci�n de onda.}

Partiendo de la ecuaci�n \eqref{ec_ondas1}, se estudia el caso de una onda con simetr�as en $\hat{x}$ y en $\hat{y}$, es decir:
\begin{equation}\label{ec_40}
	\bm{E}=\bm{E}_x\,(z,t)+\bm{E}_y\,(z,t)+\bm{E}_z\,(z,t)
\end{equation}
Ademas, se supone que las ecuaciones que describen el comportamiento del campo son independientes entre s�:
\begin{equation}\label{ec_41}
	\bm{E}\,(z,t)=\Psi\,(z)T(t)
\end{equation}
Reemplazando este resultado en \eqref{ec_ondas1}:
\begin{equation}\label{ec_42}
	T\frac{d^2\Psi}{dz^2\,}=\mu_0\epsilon_0\Psi\frac{d^2\,T}{dt^2}
\end{equation}
\begin{equation}\label{ec_43}
	\frac{1}{\Psi}\frac{d^2\,\Psi}{dz^2\,}=\frac{\mu_0\epsilon_0}{T}\frac{d^2\,T}{dt^2}
\end{equation}
De donde quedan finalmente las siguientes ecuaciones diferenciales, en donde K es una constante positiva
\begin{equation}\label{ec_44}
	\ast\frac{1}{\Psi}\frac{d^2\Psi}{dz^2}=-K^2\\
	\ast\frac{\mu_0\epsilon_0}{T}\frac{d^2\,T}{dt^2}
\end{equation}
Con soluci�n:
\begin{equation}\label{ec_ondas3}
	\Psi=A_z\,e^ikz+B_z\,e^-ikz\quad\,T=A_t\,e^i\omega\,z+B_z\,e^-i\omega\,z
\end{equation}
Donde:
\begin{equation}\label{ec_45}
	\omega=K^2\,V^2
\end{equation}

Finalmente, el campo ser� la suma de todas las posibles soluciones:
\begin{equation}\label{ec_46}
	E(z,t)=\sum\limits_K{(A_z\,e^ikz+B_z\,e^-ikz\,)(A_t\,e^i\omega\,z+B_z\,e^-iwz\,)}
\end{equation}
Al desarrollar los productos de los par�ntesis, se toman solo dos de las soluciones\footnote{Convenci�n para el electromagnetismo}
\begin{equation}\label{ec_47}
	E(z,t)=\sum\limits_K{(A\,e^i(kz+\omega\,t)\,)+(B\,e^i(kz-\omega\,t)\,)}
\end{equation}

\subsection{Soluci�n fasorial}

\begin{equation}\label{ec_48}
\hookrightarrow\bm{E}\,(\bm{r}\,,t)=R_e\,{\underline{\bm{E}}\,(\bm{r})\,e^-iwt\,}\quad\text{Donde $\underline{\bm{E}}\,(\bm{r})$ es el fasor de $\bm{E}$}
\end{equation}
\begin{equation}\label{ec_fasor0}
	\bm{E}_c\,(\bm{r}\,,t)=\underline{\bm{E}}\,(\bm{r})\,e^-i\omega\,t
\end{equation}
Aplicando la ecuaci�n \eqref{ec_ondas1} a la expresi�n anterior, se obtiene el siguiente resultado:
\begin{equation}\label{ec_fasor1}
\frac{\partial\,^2\,\underline{\bm{E}}\,(\bm{r})}{\partial\,z^2}=-\omega\,^2\mu_0\,\epsilon_0\,\underline{\bm{E}}\,(\bm{r})
\end{equation}

Se definen otros valores importantes:\\
	$\ast$ La fase del campo el�ctrico: $\varphi_E\,=Kz+\omega\,t+\varphi_A$\\
	$\ast$ La velocidad de fase\footnote{La velocidad de un punto con fase constante}: $V_ph\,=\frac{dz}{dt}=\frac{-\omega\,}{K}<0$\\
	$\ast$ La constante de propagaci�n ''K''

\subsection{Ilustraciones}

Vamos a analizar en esta parte del documento, la propagaci�n de una onda electromagn�tica plana\footnote{Onda de frecuencia constante, cuyos frentes de onda son planos paralelos de amplitud constante}, en un medio conductor.

Para el analisis se tiene la siguiente soluci�n de onda:
\begin{equation}\label{ec_fasor2}
\frac{\partial^2\,\bm{E}}{\partial\,z^2}=\mu\sigma\frac{\partial\bm{E}}{\partial\,t}+\mu\epsilon\frac{\partial^2\,\bm{E}}{\partial\,t^2}
\end{equation}

Reemplazando \eqref{ec_fasor0} en la ecuaci�n anterior, se obtiene:
\begin{equation}\label{ec_49}
\frac{d^2\,\underline{\bm{E}}\,(z)}{dz^2}=-\emph{i}\mu\sigma\omega\underline{\bm{E}}\,(z)-\mu\epsilon\omega^2\,\underline{\bm{E}}\,(z)
\end{equation}

	Si la onda solo fuese incidente, entonces $\underline{\bm{E}}\,(z)=\underline{\bm{E}}_0\,e^iKz$\\.
	Y si $\bm{E}_0$ no dependiera de z, entonces $\frac{d^2\,\underline{\bm{E}}\,(z)}{dz^2}=-K^2\,\bm{E}_0$
\begin{equation}\label{ec_50}
	\Rightarrow\,K^2\,\underline{\bm{E}}\,(z)=(\mu\epsilon\omega^2\,+\emph{i}\mu\sigma\omega\,)\underline{\bm{E}}\,(z)
\end{equation}
En donde $K^2\,=\mu\epsilon\omega\,^2\,+\emph{i}\mu\sigma\omega$ y $K=\alpha\,+\emph{i}\beta$
De esto se tiene que:
\begin{equation}\label{ec_51}
	\mu\epsilon\omega^2\,=\alpha^2\,-\beta^2\,\qquad\mu\sigma\omega=2\alpha\beta
\end{equation}

Y finalmente se obtiene la siguiente soluci�n:
\begin{equation}\label{ec_fasor3}
	\boxed{\alpha=\sqrt{\frac{\mu\epsilon}{2}}\omega\,(1+\sqrt{1+(\frac{\sigma}{\omega\epsilon})^2\,})^\frac{1}{2}\,}
\end{equation}
\begin{equation}\label{ec_fasor4}
	\boxed{\beta=\sqrt{\frac{\mu\epsilon}{2}}\omega\,(\sqrt{1+(\frac{\sigma}{\omega\epsilon})^2\,}-1)^\frac{1}{2}\,}
\end{equation}


Ahora consideremos una onda plana propagandose en un semiespacio infinito bien conductor:
%Semiespacio infinito bien conductor
%Semiespacio infinito bien conductor
%Semiespacio infinito bien conductor
%Semiespacio infinito bien conductor
%Semiespacio infinito bien conductor



\subsection{El factor de calidad}

Se introduce el valor Q, como el factor de calidad, definido as�:
\begin{equation}\label{ec_52}
	Q=\frac{||J_d||}{||J_f||}=\frac{||\partial\bm{D}/\partial\,t||}{||\sigma\bm E||}=\frac{\epsilon\omega}{\sigma}
\end{equation}

$\ast$Si estamos en un medio parecido al vacio, entonces $\sigma\rightarrow\,0$ y por tanto $Q\gg1$, y para $\alpha$ y $\beta$ se tiene lo siguiente:
\begin{equation}\label{ec_52}
	\alpha\rightarrow\omega\sqrt{\mu\epsilon}\Rightarrow\,V=\frac{\omega}{\alpha}
\end{equation}
\begin{equation}\label{ec_53}
	\beta\rightarrow\,0
\end{equation}

$\ast$Si estamos en un medio buen conductor, entonces $\sigma\rightarrow\infty$ y por tanto Q$<<$1, y para $\alpha$ y $\beta$ se tiene lo siguiente:
\begin{equation}\label{ec_54}
	\alpha=(\frac{\mu\sigma\omega}{2})^{1/2}(1+\frac{Q}{2})
\end{equation}
\begin{equation}\label{ec_55}
	\beta=(\frac{\mu\sigma\omega}{2})^{1/2}(1-\frac{Q}{2})
\end{equation}
\begin{equation}\label{ec_56}
	V=(\frac{2\omega}{\mu\sigma})^{1/2}(1-\frac{Q}{2})
\end{equation}










\end{document}


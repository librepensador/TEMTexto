\chapter{Condiciones de contorno en las interfases.}


En la realidad se presentan cambios suaves en el promedio de alguna propiedad local en una regi�n determinada. En la idealizaci�n de un mundo continuo (f�sica para los medios continuos) se consideran cambios abruptos de las propiedades, como se ilustra en la figura \ref{g_CamMed}; es decir, $\Delta h$ es nulo. 
\begin{figure}[ht]
	\centering
		\includegraphics{6g1.1}
	\caption{Cambio de medio real e ideal de una propiedad local. En $x=x_0$ est� la interfase.}
	\label{g_CamMed}
\end{figure}

Adem�s:
\begin{itemize}
	\item Las ecuaciones diferenciales no son v�lidas en las interfases entre medios diferentes. Es necesario obtener y aplicar condiciones de contorno que permitan el salto de un medio a otro. 
	\item El procedimiento general consiste en suponer que la transici�n entre medios se produce de forma suave en un intervalo $\Delta h$, se aplica la ecuaci�n integral correspondiente y despu�s se hace tender $\Delta h$ a cero. 
\end{itemize}

Conociendo la divergencia de un campo se puede conocer el comportamiento de las componentes normales. Conociendo el rotacional de un campo se puede conocer el comportamiento de las componentes tangenciales. 


\section{Componentes normales de $\mathbf{A}$.}


Por convenci�n se elige la normal $\mathbf{\hat{n}}$ del medio 1 al medio 2, como se ilustra en \ref{g_DirVecNor}.
\begin{figure}[ht]
	\centering
		\includegraphics{6g2.2}
	\caption{El vector normal: del medio 1 al medio 2.}
	\label{g_DirVecNor}
\end{figure}

\begin{figure}[ht]
	\centering
		\includegraphics{6g3.3}
	\caption{Volumen $V$ encerrado por las superficies de control.}
	\label{g_VolCamMed}
\end{figure}

Se conoce la divergencia de $\mathbf{A}$;
\begin{equation}
	\nabla\cdot\mathbf{A}=d(\mathbf{r})\;.
\end{equation}

Sea $V$ el volumen encerrado por $S_1+S_2+S_{LAT}$, como se ilustra en la figura \ref{g_VolCamMed}, entonces:
\begin{equation}
	\iiint\limits_{V}{\nabla\cdot\mathbf{A}\,dV}=\iiint\limits_{V}{d(\mathbf{r})\,dV}\;.
\end{equation}

Por el teorema de la divergencia, la primera integral es:
\begin{equation}
	\iiint\limits_{V}{\nabla\cdot\mathbf{A}\,dV}=\oiint\limits_{S}{\mathbf{A}\cdot d\mathbf{S}}=
	\iint\limits_{S_1}{\mathbf{A}\cdot d\mathbf{S}}+
	\iint\limits_{S_2}{\mathbf{A}\cdot d\mathbf{S}}+
	\iint\limits_{S_{LAT}}{\mathbf{A}\cdot d\mathbf{S}}\;.
\end{equation}

Cuando $\Delta h \to 0$ ocurre que $S_1\to S$, $S_2\to S$, $S_{LAT}\to S$, $\mathbf{\hat{n}_1}\to-\mathbf{\hat{n}}$, $\mathbf{\hat{n}_2}\to\mathbf{\hat{n}}$. Luego:
\begin{equation}
\begin{gathered}
	\iint\limits_{S_1}{\mathbf{A}\cdot d\mathbf{S}}\to-
			\iint\limits_{S}{\mathbf{A_1}\cdot \mathbf{\hat{n}}\,dS}\medskip  \\
	\iint\limits_{S_2}{\mathbf{A}\cdot d\mathbf{S}}\to 
			\iint\limits_{S}{\mathbf{A_2}\cdot \mathbf{\hat{n}}\,dS}\medskip  \\		
	\iint\limits_{S_{LAT}}{\mathbf{A}\cdot d\mathbf{S}}\to 0\;.	
\end{gathered}
\end{equation}

$\mathbf{A_1}$ y $\mathbf{A_2}$ son lo valores del campo en los medios 1 y 2, respectivamente. 
\begin{equation}
	\boxed{
	\oiint\limits_{S}{\mathbf{A}\cdot d\mathbf{S}} =
	\iint\limits_{S}{\mathbf{\hat{n}}\cdot \left(\mathbf{A_2}-\mathbf{A_1}\right)\, d\mathbf{S}} =
	\lim\limits_{\Delta h\to 0}{\iiint\limits_{V}{\left(\nabla\cdot\mathbf{A}\right)\,dV}}
	}
\end{equation}


\subsection{Para el campo el�ctrico.}


\begin{equation}
	\iint\limits_{S}{\mathbf{\hat{n}}\cdot\left(\mathbf{E_2}-\mathbf{E_1}\right)\,dS}=
	\lim\limits_{\Delta h \to 0}{\iiint\limits_{V}{\left(\nabla\cdot\mathbf{E}\right)\,dV}}=
	\lim\limits_{\Delta h \to 0}{\iiint\limits_{V}{\frac{\rho}{\epsilon_0}\,dV}}\;.
\end{equation}

$\rho$ es la densidad de carga y $\iiint\limits_{V}{\rho\,dV}$ es la carga alojada en el volumen $V$. A medida que $\Delta h$ tiende a cero, el volumen $V$ se asemeja cada vez m�s a la superficie $S$. Se supone entonces que toda la carga del volumen $V$ est� alojada sobre la superficie $S$ ($\rho=\delta(z)\,\sigma$), suponiendo el eje $z$ como eje del cilindro. As�,
\begin{equation}
\begin{gathered}
	\iint\limits_{S}{\mathbf{\hat{n}}\cdot\left(\mathbf{E_2}-\mathbf{E_1}\right)\,dS}=
		\lim\limits_{\Delta h \to 0}{\iiint\limits_{V}{\frac{\delta(z)\,\sigma}{\epsilon_0}\,dV}}=
		\lim\limits_{\Delta h \to 0}{\iint\limits_{S}{\frac{\sigma}{\epsilon_0}\,dS}}\medskip \\
	\iint\limits_{S}{\mathbf{\hat{n}}\cdot\left(\mathbf{E_2}-\mathbf{E_1}\right)\,dS}=	
		\iint\limits_{S}{\frac{\sigma}{\epsilon_0}\,dS}\medskip \\
\end{gathered}
\end{equation}

Como la superficie $S$ es totalmente arbitraria en la interfase, entonces:
\begin{equation}
	\boxed{
	\mathbf{\hat{n}}\cdot\left(\mathbf{E_2}-\mathbf{E_1}\right)=\frac{\sigma}{\epsilon_0}
	\quad \text{�}	\quad
	E_{2n}-E_{1n}=\frac{\sigma}{\epsilon_0}\;.
	}
\end{equation}


\subsection{Para el desplazamiento el�ctrico.}


\begin{equation}
	\iint\limits_{S}{\mathbf{\hat{n}}\cdot\left(\mathbf{D_2}-\mathbf{D_1}\right)\,dS}=
	\lim\limits_{\Delta h \to 0}{\iiint\limits_{V}{\left(\nabla\cdot\mathbf{D}\right)\,dV}}=
	\lim\limits_{\Delta h \to 0}{\iiint\limits_{V}{\rho_f\,dV}}\;.
\end{equation}

$\rho_f$ es la densidad de carga libre y $\iiint\limits_{V}{\rho_f\,dV}$ es la carga libre alojada en el volumen $V$. Con $\rho_f=\delta(z)\,\sigma_f$ se obtiene:
\begin{equation}	
\begin{gathered}
	\iint\limits_{S}{\mathbf{\hat{z}}\cdot\left(\mathbf{D_2}-\mathbf{D_1}\right)\,dS}=
		\iint\limits_{S}{\sigma_f\,dS}\medskip \\
	\boxed{\mathbf{\hat{n}}\cdot\left(\mathbf{D_2}-\mathbf{D_1}\right)=\sigma_f 
	\qquad\text{�}\qquad
	D_{2n}-D_{1n}=\sigma_f}
\end{gathered}
\end{equation}


\section{Componentes tangenciales de $\mathbf{A}$.}


\begin{figure}[ht]
	\centering
		\includegraphics{6g4.4}
	\caption{Componentes tangenciales.}
	\label{g_ComTan}
\end{figure}

Ahora se conoce el rotacional de $\mathbf{A}$:
\begin{equation}
	\nabla\times\mathbf{A}=\mathbf{c}\;.
\end{equation}

Seg�n la figura \ref{g_ComTan}, siendo $S$ la superficie limitada por $L_1$, $L_2$ y $L_{LAT}$, y siendo  $C=L_1+L_2+L_{LAT}$:
\begin{equation}
\begin{gathered}
	\iint\limits_{S}{\left(\nabla\times\mathbf{A}\right)\cdot d\mathbf{S}}=
		\oint\limits_{C}{\mathbf{A}\cdot d\mathbf{r}}\bigskip \\ 
	\oint\limits_{C}{\mathbf{A}\cdot d\mathbf{r}}=
		\int\limits_{L_1}{\mathbf{A}\cdot d\mathbf{r}}+
		\int\limits_{L_2}{\mathbf{A}\cdot d\mathbf{r}}+
		\int\limits_{L_{LAT}}{\mathbf{A}\cdot d\mathbf{r}}\;.
\end{gathered}
\end{equation}

Si $\Delta h \to 0$, se tiene que $L_2 \to L$, $L_1 \to -L$, $L_{LAT} \to 0$:
\begin{equation}
\begin{gathered}
		\int\limits_{L_1}{\mathbf{A}\cdot d\mathbf{r}}=
			-\int\limits_{L}{\mathbf{A_1}\cdot \mathbf{\hat{t}}\,dr}\bigskip \\ 
		\int\limits_{L_2}{\mathbf{A}\cdot d\mathbf{{r}}}=
			\int\limits_{L}{\mathbf{A_2}\cdot \mathbf{\hat{t}}\,dr}\bigskip \\
		\int\limits_{L_{LAT}}{\mathbf{A}\cdot d\mathbf{r}}=0
\end{gathered}
\end{equation}

Entonces:
\begin{equation}
\begin{gathered}
	\oint\limits_{C}{\mathbf{A}\cdot\mathbf{r}} \to 
		\int\limits_{L}{(\mathbf{A_2}-\mathbf{A_1})\cdot\mathbf{\hat{t}}\,dr} \to
		\iint\limits_{S}{(\nabla\times\mathbf{A})\cdot d\mathbf{S}} \medskip \\ 
	\boxed{\int\limits_{L}{(\mathbf{A_2}-\mathbf{A_1})\cdot\mathbf{\hat{t}}\,dt} =
			\lim_{\Delta h\to0}\iint\limits_{S}{(\nabla\times\mathbf{A})\cdot d\mathbf{S}}}
\end{gathered}
\end{equation}


\subsection{Para el campo el�ctrico.}


\begin{equation}
\begin{gathered}
	\int\limits_{L}{(\mathbf{E_2}-\mathbf{E_1})\cdot\mathbf{\hat{t}}\,dr} =
			\lim_{\Delta h\to0}\iint\limits_{S}{(\nabla\times\mathbf{E})\cdot d\mathbf{S}}=0\;.
\end{gathered}
\end{equation}

Como el resultado es para cualquier $L$ sobre $S$, entonces:
\begin{equation}
	(\mathbf{E_2}-\mathbf{E_1})\cdot\mathbf{\hat{t}}=0\;,
\end{equation}

donde $\mathbf{\hat{t}}$ es un vector tangente a $L$ y a su vez a $S$. As�, 
\begin{equation}
	E_{2l}-E_{1l}=0\quad\Rightarrow\quad \boxed{E_{2l}=E_{1l}}
\end{equation}

Lo anterior implica que las componentes tangenciales del campo el�ctrico son continuas al cambia de medio. 


\subsection{Para el desplazamiento el�ctrico.}


\begin{equation}
\begin{gathered}
	\int\limits_{L}{(\mathbf{D_2}-\mathbf{D_1})\cdot\mathbf{\hat{t}}\,dt} =
			\lim_{\Delta h\to0}\iint\limits_{S}{(\nabla\times\mathbf{D})\cdot d\mathbf{S}}=
			\lim_{\Delta h\to0}\iint\limits_{S}{(\nabla\times\mathbf{P})\cdot d\mathbf{S}}=
\end{gathered}
\end{equation}

Si:
\begin{equation}
	\int\limits_{L}{(\mathbf{D_2}-\mathbf{D_1})\cdot\mathbf{\hat{t}}\,dt} =
		\int\limits_{L}{(\mathbf{P_2}-\mathbf{P_1})\cdot\mathbf{\hat{t}},dt}\;,
\end{equation}

entonces:
\begin{equation}
	\boxed{D_{2l}-D_{1l}=P_{2l}-P_{1l}}
\end{equation}


\section{Resumen.}


A continuaci�n se presenta una tabla de las ecuaciones de componentes recurrentes para el c�lculo de las variables electromagn�ticas. A partir de las ecuaciones diferenciales, aplicando un procedimiento similar al visto anteriormente, puede llegarse f�cilmente a las ecuaciones de componentes. 

\begin{table}[ht]
	%\renewcommand{\arraystretch}{2.2}
	\centering
		\begin{tabular}{|c|c|}\hline
			Ecuaci�n diferencial & Ecuaci�n de componentes\\ \hline
			$\nabla\cdot\mathbf{E}=\dfrac{\rho}{\epsilon_0}$ & 
					$\mathbf{\hat{n}}\cdot(\mathbf{E_2}-\mathbf{E_1})=\dfrac{\sigma}{\epsilon_0}$\\ \hline
			$\nabla\times\mathbf{B}
							=\mu_0\,\mathbf{J}+\mu_0\epsilon_0\dfrac{\partial\mathbf{E}}{\partial t}$ & 
					$\mathbf{\hat{n}}\times(\mathbf{B_2}-\mathbf{B_1})=\mu_0\,\mathbf{K}$\\ \hline
			$\nabla\times\mathbf{E}=-\,\dfrac{\partial\mathbf{B}}{\partial t}$ & 
					$\mathbf{\hat{n}}\times(\mathbf{E_2}-\mathbf{E_1})=\textbf{0}$ \\ \hline
			$\nabla\cdot\mathbf{B}=0$ &
					$\mathbf{\hat{n}}\cdot(\mathbf{B_2}-\mathbf{B_1})=\textbf{0}$\\ \hline
			$\nabla\cdot\mathbf{D}=\rho_f$ &
					$\mathbf{\hat{n}}\cdot(\mathbf{D_2}-\mathbf{D_1})=\sigma_f$\\ \hline
			$\nabla\times\mathbf{H}=\mathbf{J_f}+\dfrac{\partial\mathbf{D}}{\partial t}$ & 
					$\mathbf{\hat{n}}\times(\mathbf{H_2}-\mathbf{H_1})=\mathbf{K_f}$\\ \hline
			$\nabla\times\mathbf{D}=-\,\epsilon_0\dfrac{\partial\mathbf{B}}{\partial t}
							+\nabla\times\mathbf{P}$ & 
					$\mathbf{\hat{n}}\times(\mathbf{D_2}-\mathbf{D_1})=
							\mathbf{\hat{n}}\times(\mathbf{P_2}-\mathbf{P_1})$\\ \hline
			$\nabla\cdot\mathbf{H}=-\nabla\cdot\mathbf{M}$ & 
					$\mathbf{\hat{n}}\cdot(\mathbf{H_2}-\mathbf{H_1})=
							-\mathbf{\hat{n}}\cdot(\mathbf{M_2}-\mathbf{M_1})$\\ \hline
			$\nabla\cdot\mathbf{J}+\dfrac{\partial\rho}{\partial t}=0$ & 
					$\mathbf{\hat{n}}\cdot\left
							[\mathbf{J_2}+\epsilon_0\dfrac{\partial\mathbf{E_2}}{\partial t} 
							-\mathbf{J_1}-\epsilon_0\dfrac{\partial\mathbf{E_1}}{\partial t}\right]=0$\\ \hline
		\end{tabular}
	\caption{Ecuaciones de componentes seg�n ecuaciones diferenciales.}
	\label{tab_ec_comp1}
\end{table}
